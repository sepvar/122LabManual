\documentclass[twoside,11pt]{article}
\usepackage{html,hmwk,color}
\usepackage{graphicx}

\setlength{\evensidemargin}{0in}
\setlength{\oddsidemargin}{-.15in}
\setlength{\textwidth}{6.6in}
\setlength{\marginparwidth}{.6in}
\setlength{\marginparsep}{.15in}

\setlength{\topmargin}{-0.75in}
\setlength{\headheight}{0.1in}
\setlength{\headsep}{0.25in}
\setlength{\textheight}{9.2in}
\setlength{\footskip}{0.5in}

\newcommand{\blank}{\underline{\hspace{1cm}}}
\newcommand{\markleft}{}
\newcommand{\revised}[1]{\renewcommand{\markleft}{Last Revised: #1}}
\newcommand{\revision}[2]{}
\newcommand{\skipthis}[1]{}

\makeatletter	   % `@' is now a normal "letter' for LaTeX
\renewcommand{\ps@plain}{%
     \renewcommand{\@oddhead}{}%
     \renewcommand{\@evenhead}{\@oddhead}%
     \renewcommand{\@oddfoot}{\hfill\thepage}
     \renewcommand{\@evenfoot}{\thepage\hfill}}
\newcommand{\ps@labstyle}{%\textrm{
     \renewcommand{\@oddhead}{{\color{cyan}{\footnotesize\markleft}}\hfill{\it Lab \rightmark}}%
     \renewcommand{\@evenhead}{{\it Lab \rightmark}\hfill{\color{cyan}{\footnotesize\markleft}}}%
     \renewcommand{\@oddfoot}{\hfill\thepage}
     \renewcommand{\@evenfoot}{\thepage\hfill}
    \def\sectionmark##1{%
      \markright {\MakeUppercase{%
        \ifnum \c@secnumdepth >\m@ne
          \thesection\quad
        \fi
        ##1}}}}
\makeatother     % `@' is restored as a "non-letter" character


\newcounter{dave}
\setcounter{dave}{0}
\newenvironment{question}{\begin{enumerate}\setcounter{enumi}{\thedave}}{\end{enumerate}\setcounter{dave}{\arabic{enumi}}}


\makeindex

\begin{document}
\onecolumn

\setcounter{tocdepth}{1}
\setcounter{page}{1}
\newcounter{jim}

\newcommand{\newpar}{\hspace*{.225in}}

\newcommand{\txtcirc}{\setlength{\unitlength}{.5cm}
\begin{picture}(.4,.5)
\put(.2,.2){\circle{.4}}
\end{picture}}
\newcommand{\txtgalv}[1]{\setlength{\unitlength}{.5cm}
\begin{picture}(1,1)
\put(0,0){#1}
\put(.3,.2){\circle{1}}
\end{picture}}

\newcommand{\JC}[1]{}  %Blank for final version
%\newcommand{\JC}[1]{\marginpar[\tiny\raggedright {#1}]{\tiny\raggedleft
%%{#1}}}
\newcommand{\note}[1]{}  % Blank for final version
%\newcommand{\note}[1]{\marginpar[\tiny\raggedright {#1}\hfill
%%$\Rightarrow$]{\tiny\raggedleft $\Leftarrow$ {#1}}}

\newcommand{\JL}[1]{}  %Blank for final version
%\newcommand{\JL}[1]{\reversemarginpar\marginpar[\tiny\raggedright
%%{#1}]{\tiny\raggedleft {#1}}\normalmarginpar}
\newcommand{\JLS}[1]{}  %Blank for final version
%% FOLLOWING LINE CANNOT BE BROKEN BEFORE 80 CHAR
%\newcommand{\JLS}[1]{\reversemarginpar\marginpar[\tiny\raggedright\vspace{-.75cm} {#1}]{\tiny\raggedleft\vspace{-.75cm} {#1}}\normalmarginpar}

\newenvironment{lablist}{\begin{list}{$\Rightarrow$}{}}{\end{list}}



\title{Elements of Physics II Laboratory Manual}
\author{Dr.\ Joseph Christensen \\ Dr. Wes Ryle \\ Dr. Jack Wells \\ Physics Department, Thomas More College}
\date{} %\small $Revision: 2.0 $}
\maketitle


\pagestyle{plain}

\vfill
\begin{latexonly}
\tableofcontents
%\newpage
\ \vfill
%\listoftables\vfill
%\listoffigures\vfill
\end{latexonly}

\note{\hrule
Labs~\ref{s:spheat} and~\ref{s:laheat} shall be presented orally to the lab class during lab time of the week after Lab~\ref{s:laheat} is performed.  A single group-written report will be due at the time of the presentation.  Also, Lab~\ref{s:MagField} will be peer-reviewed in a double-blind review process.  Students will have a few days to review colleagues work.  Instructor will grade the report as well as the reviewing.
\hrule}
\vfill

\onecolumn

\renewcommand{\thesection}{\Roman{section}}
\renewcommand{\thefigure}{\Roman{section}.\arabic{figure}}
\renewcommand{\theequation}{\Roman{section}.\arabic{equation}}
\renewcommand{\theenumi}{\Roman{section}.\arabic{enumi}}

\section{Uncertainty: A Measure of Precision}

It should be realized that the {\it true value} of a physical quantity is
almost never known.  If several determinations of a quantity are made using
the same apparatus, the results differ for a number of reasons.  Sometimes
the accuracy with which a given measurement can be made is determined by
variations in the thing being measured.  For instance, a number of
measurements
of the diameter of a baseball would probably show that the ball is not a
perfect sphere and consequently the measured values would be distributed over
a range of values.

Sometimes the accuracy with which a measurement can be made is determined by
the accuracy with which the scale on the instrument can be read.  For example,
it is hardly possible to read a meter stick more closely than $\pm$ 0.5mm.
The limits of accuracy may be set either by the precision of the scale of the
instrument or by the ability and/or skill of the observer.  But limits always
exist.

It is also possible to have systematic error due to faulty instruments, for
example, a meter stick which is not exactly one meter long.  Then all
measurements made with the instrument are in error, usually by a constant
factor.

It must be noted that uncertainty is not due to the failure of the observer to
read the instruments correctly.  If the observer records a 99.5 when the
value should have been 89.5, this is not uncertainty, but is a {\it mistake}.

It is always of interest and usually necessary to know just how dependable
are the results of an experiment and it is usually not the absolute
uncertainty
that is important but the percent uncertainty between the measured value
and the \latex{``}\html{"}true" value.  For example, a $1000\unit{km}$ uncertainty in
measuring the
distance from Abilene to Moscow is much worse than a $1000\unit{km}$ uncertainty in
measuring the distance from Abilene to the Sun.

When the true answer is known, the percent error is calculated from
the difference divided by the true value:
%
\[ \%\mbox{-\small error} = {({\mbox{\small true value -- experimental
value}})\over{{\mbox{\small true value}}}} \times 100 \% \]
%
The percent difference is calculated between two experimental values from the
difference divided by the average (which is your best guess at the true
value):
%
\[ \%\mbox{-\small difference} = \left[ \frac{\left(\mbox{\small value \#1 --
value \#2}\right)}
                                                   {\left(\frac{\mbox{\small
value \#1 + value \#2}}{\mbox{\small 2}}\right)} \right] \times 100 \% \]

\subsection{Propagation of Uncertainty}

Along with knowing the {\bf percent difference} between two numbers, it is
also
necessary sometimes to know whether two numbers are {\it consistent}, i.e.,
is it possible for the numbers to be equal to each other if the uncertainties
on the
numbers are taken into account.  For example, suppose the
\latex{``}\html{"}true value" is
20, but your experimental result is $17 \pm 3$.  At first glance we can say
that 17 does not equal 20, but since the actual range of the result is from
14 to 20 (i.e., 17 plus or minus 3), it very well could be equal to 20.  If
this is the case, we say that the experimental result and the true value are
consistent.  If the experimental result was 15 $\pm$ 3, we say it is
inconsistent with the true value.  Therefore, it is essential to know the
uncertainty range (A.K.A.\ margin of error, or error-bars) on your experimental
results.  This is how
you tell whether your answer is \latex{``}\html{"}good enough" or not.

The uncertainty range on an experimental result depends on the uncertainties of
all the
measurements that were made during the lab leading up to this result.  Taking
these various measurement uncertainties and determining the uncertainty range
on the final
answer requires a process known as {\bf Error Propagation}.  One result of
error propagation is that the various experimental uncertainties always combine
to
increase the overall uncertainty.  For example:  suppose measurements of the
length
of two pieces of string are made, with the goal of knowing their combined
length.  If the first piece is measured to be $10 \pm 1 \unit{cm}$ and the second is
measured to be $5\pm1\unit{cm}$, the total length is $15\unit{cm}$, but the overall
uncertainty
is $\pm$ 2 cm.  The 1 cm uncertainty on each measurement added to give a
combined
uncertainty of 2 cm.

Notice that the first string can be no shorter than 9cm and no longer than 11cm
(10 $\pm$ 1 cm).
Similarly, the second string can be no shorter than 4cm and no longer than 6cm
(5 $\pm$ 1 cm).
Therefore, the combination can be no shorter than 13cm and no longer than 17cm:
 15 $\pm$ 2cm.

The above simple example dealt with what will be called the uncertainty.
Another way to express uncertainty is the {\it percent uncertainty}.  This is
equal to the
uncertainty divided by the measurement, times 100\%.  For example, the
percent uncertainty from the above example would be ($\frac{1{\rm cm}}{10{\rm
cm}}\times 100\% = 10\%$)
and ($\frac{1{\rm cm}}{5{\rm cm}}\times 100\% = 20\%$).  In some cases
of error propagation the uncertainties are used and in other cases, the
percent uncertainties are used.  The rules for determining which to use are
given
below:
\begin{enumerate}
\item\label{add}  When two measurements with associated uncertainties are added
or subtracted, the
overall uncertainty is equal to the sum of their uncertainties.
\item\label{multiply}  When two measurements with associated uncertainties are
multiplied or divided, the
overall percent uncertainty is equal to the sum of their percent uncertainty.
\end{enumerate}

%\newpage
\subsubsection*{Example:}

\begin{minipage}[t]{3.2in}

The area and perimeter of a rectangular table are to be calculated.
The table is measured to be 176.7 cm $\pm$ 0.2 cm along one side and
148.3 cm $\pm$ 0.3 cm along the other side.
Because the perimeter is found by adding the sides, rule~\ref{add} is used:
%
\begin{eqnarray*}
P & = & (176.7 {\rm cm}) + (148.3 {\rm cm}) \\&& + (176.7 {\rm cm}) + (148.3
{\rm cm}) \\
P  & = & 650.0 {\rm cm} \\ \\
\Delta P & = & (0.2 {\rm cm}) + (0.3 {\rm cm}) \\&& + (0.2 {\rm cm}) + (0.3
{\rm cm}) \\
\Delta P & = & 1.0 {\rm cm}
\end{eqnarray*}
%
The perimeter is \fbox{$P = 650{\rm cm} \pm 1{\rm cm}$}.
The area of the table is calculated to be (significant digits are underlined)
%
\begin{eqnarray*}
A & = & (176.7 {\rm cm}) \times (148.3 {\rm cm}) = \underline{2620}4.61 {\rm
cm}^2  \\ \\
\Delta A & = & \left(\frac{.2{\rm cm}}{176.7 {\rm cm}} 100\%\right) +
\left(\frac{0.3{\rm cm}}{148.3 {\rm cm}} 100\%\right) \\
\Delta A & = & (.11\%) + (.20\%) = (.31\%)
\end{eqnarray*}
%
\end{minipage}
\hfill
\begin{minipage}[t]{3.2in}
Since $(.31\%)\times (\underline{2620}4.61 {\rm cm}^2) = 82.67 {\rm cm}^2$, we
write the area in a variety of ways:
%
\begin{eqnarray*}
A & = & (2.620\!\times\!10^4 {\rm cm}^2) \pm 0.3\% \\
  & = & (2.620\!\times\!10^4) \pm (0.008\!\times\!10^4) {\rm cm}^2 \\
  & = & 2.620(8)\!\times\!10^4 {\rm cm}^2
\end{eqnarray*}
%
Please be aware that the reason some digits are called insignificant is that
they are insignificant.  For example, when calculating the uncertainty above:
%
\begin{eqnarray*}
  & & (.31548\%) \times (26204.61) = 82.67 \\
  & & (.31\%) \times (26204.61) = 81.23 \\
  & & (.31\%) \times (26200) = 81.22 \\
  & & (.3\%) \times (26204.61) = 78.61 \\
  & & (.3\%) \times (26200) = 78.60
\end{eqnarray*}
%
But, for the uncertainty, we only need the first digit and
all of these round to $80$cm$^2$, giving
\begin{quote}
\fbox{$A = (2.620\!\times\!10^4) \pm (0.008\!\times\!10^4) {\rm cm}^2$}.
\end{quote}
\end{minipage}

%-----------------------------------------------------------------------------------
\onecolumn %========================================
\pagestyle{labstyle}
%\addtocontents{toc}{\newpage}
\addtocontents{toc}{\vspace{8pt}\hrule}
\addcontentsline{toc}{section}{{\large\bf Laboratory Exercises}\hfill\ldots}
\setcounter{section}{0}
\renewcommand{\thesection}{\arabic{section}}
\renewcommand{\thefigure}{\arabic{section}.\arabic{figure}}
\renewcommand{\theequation}{\arabic{section}.\arabic{equation}}
\renewcommand{\theenumi}{\arabic{section}.\arabic{enumi}}
%-------------------------------------------------------------------------------------
%                     *** START LAB INSTRUCTIONS ***
%-------------------------------------------------------------------------------------


%-------------------------------------------------------------------------------------

\onecolumn

\section[Skills Worksheet]{Review the Skills of PHY 121 Lab}

Recall:
\begin{itemize}
\item measurement uncertainty
\item propagation of uncertainty
\item accuracy versus precision
\item \%-error and \%-difference
\item Tabulate data in excel
\item Graph data in excel
\item Linear Regression
\item How to write a Lab Report
\end{itemize}

\newpage

\section{Archimedes' Principle}
\revised{Nov. 23, 2008}


\subsection{Introduction}

Archimedes was a Greek scientist who lived from 287-212 BCE.  As the story goes, the king thought that his crown was not pure gold and asked Archimedes to determine if this was true.   Archimedes had previously observed that a totally submerged object displaces a volume of fluid which is equal to the volume of the object, and deduced what is now called Archimedes' principle: that the buoyant force on the submerged object is equal to the weight of the displaced fluid.
Knowing the weight and density of gold, Archimedes measured the weight of the crown and the weight and volume of the water displaced by the crown, enabling him to determine that the crown was not pure gold.

As any swimmer who has had their head more than about 8 feet deep can tell you, the pressure exerted on the diver by the water increases as the diver swims deeper into the water.  We can express this as: the pressure within a fluid is dependent on the depth at which the object sits ($h$), the gravitational acceleration of the earth ($g$), and the density of the fluid ($\rho_f$): $P=\rho_f gh$.  Because of this, the bottom surface of any object in a fluid will have more pressure on it than the top surface.  This $\Delta P$ gives rise to a net upward force acting on the submerged object: the buoyant force ($F_B$).  The magnitude of the upward force depends on the density of the fluid and the size of the object: $F_B=\rho_f g V_o$, where $V_o$ is \underline{the submerged portion} of the volume of the object.  A partially submerged object has a smaller buoyant force than a completely submerged object.  Interestingly, unless the fluid density depends on the depth of the fluid, the $F_B$ is independent of the depth of the object.

Knowing the buoyant force will help you determine if an object will float or sink.  To determine whether an object will sink or float one should use Newton's second law to determine the direction of the net force.  Assuming the simplest case of a submerged object feeling only its weight $(F_g=m_og = \rho_o g V_o)$ and the buoyant force $(F_B=\rho_f g V_o)$, if the weight of the object is greater than $F_B$ then the net force on the object is down and it will sink.  If $F_B$ is larger than the weight then the net force on the object is up and then it will accelerate up.  However, when an object is floating, it is in equilibrium and the object's weight is equal to $F_B$, which depends on the submerged volume.  While floating, not all of the object is submerged.


\subsection{Experimental Objectives}

After verifying some properties of the buoyant force using Archimedes' principle, each group will predict the maximum amount of cargo (the number of pennies) that a ship (glass beaker) will hold without sinking.

\subsection{Pre-Lab Work}

\begin{lablist}
\item Using the definitions of force as pressure times area, $F=PA$, and pressure, $P=\rho_f g h$, derive the equations for the buoyant force $F_B=\rho_fgV_o$ (if totally submerged) and for the object's weight $F_g=\rho_o gV_o$, where $\rho_o$  is the object's density.
\item Show that although the pressure on an object does depend on depth within the fluid, the $\Delta P = P_{\rm bottom}-P_{\rm top}$ (and therefore $F_B$) is independent of the depth within the fluid.
\item Draw a force diagram for a completely submerged object:
    \begin{description}
    \item[case 1] object's density is greater than that of the fluid's,
    \item[case 2] object's density is less than that of the fluid's.
    \end{description}
\item Using Newton's Second Law show that the object in case 1 will sink, and  that the object in case 2  will accelerate up.
\item Draw a force diagram for a partially submerged object which is floating.
\item Using Newton's Second Law show that if an object has a density $\rho_o=.5\rho_f$  that only 0.5 of the object's volume is submerged.
\item Look up the density of water.
\end{lablist}

\subsection{Procedure}

\subsubsection{Develop Your Understanding}

\begin{lablist}
\item Select an object for which you can easily measure the volume using a caliper.
\item Measure the dimensions and calculate the volume.
\item Measure the mass using the triple beam balance at your table.
\item Calculate the density with uncertainty.
\item Predict (with uncertainty) what buoyant force this object would have if it were completely submerged.
\item Predict (with uncertainty) what a scale would read if this object were completely submerged.
\end{lablist}

\subsubsection{Verify Your Understanding}

\begin{lablist}
\item Place a balance on top of a support rod so that the side with the string dangling can drop into a container of water.
\item Find the mass of your catch beaker -- this should be a graduated cylinder because you will be using this beaker to measure both the volume and the mass of the fluid that overflows from the overflow container.
\item Place an overflow container under the scale so that the string can dip into the container if the lab jack is raised or lowered.
\item Place a catch beaker next to the overflow container and slowly fill the overflow container until it just starts dripping.  Gently tap the overflow container once.  Be very careful not to bump your table after this.
\item As thoroughly as possible, dry off your catch beaker and put it back in place.  If you cannot dry it off completely, you will not be able to accurately complete all of the necessary steps.
\item Attach your object to the string, measure the mass of the object, $m_o$, in air while the object dangles.
\item Raise the lab jack until the object is completely submerged.  {\bf Be sure to catch all of the water that overflows from the container!}
\item The scale should now read a reduced mass, $m_s$, for the submerged object.  Record this value.
\item You now have three ways of calculating the buoyant force:  (Pay attention to which is the most precise and why.)
    \begin{enumerate}
    \item \label{i:deltam} The water (via the buoyant force) supports the difference in weights between not submerged and totally submerged.  Calculate the buoyant force, with uncertainty, as the difference in these \underline{weights}.  Does it agree with your prediction?
    \item \label{i:V_f} Measure the volume of the overflow water.  This is equal to the volume of the object submerged, $V_f = V_o$.  Calculate (with uncertainty) the buoyant force from this volume, $F_B = \rho_f g V_o$.  Does this agree with your prediction?
    \item \label{i:m_f} Note that $\rho_f V_f = m_f$, so instead of measuring $V_f$, we can measure the mass of the overflow water.  The buoyant force can also be found from:  $F_B = m_f g$, the weight of the fluid.  Does this agree with your prediction?
    \end{enumerate}
\end{lablist}

You should ensure that you understand these results {\it with uncertainty} before continuing.  If this doesn't make sense, then check your numbers, check your units, remeasure quantities that you thought you were sure of, and ask your instructor.

\subsubsection{Consider How the Buoyant Force Changes as an Object is Submerged}

For this part, you are going to repeat the previous procedure, with three modifications.  First, you won't be able to measure the volume with a caliper.  Second, based on the previous procedure, we now know which measurement provides the best estimate (smallest uncertainty) for the buoyant force, so you should only need to do one of \ref{i:deltam}, \ref{i:V_f}, or \ref{i:m_f} from the previous section.  Finally, we will consider an object in air, partially submerged, and fully submerged.

\begin{lablist}
\item Select a fairly large object for which you cannot easily measure the volume using a caliper.  This must fit within the overflow container without touching the sides.
\item Refill the overflow container, as before, until is just starts to overflow.
\item As thoroughly as possible, dry off your catch beaker.  If you cannot dry it off, you will have to measure its mass with whatever water is in it and you will not be able to use the previous section technique \ref{i:V_f} because you cannot accurately measure the volume of new water.
\item Attach it to the string hanging from the balance and find the mass in air, $m_o$.
\item {\bf You will need to do the rest of this {\it very} carefully.  Go slowly and do not bump the table.}
\item Raise the lab jack until the object is about half submerged.  Catch the overflow water.
\item Measure the apparent mass of the object and the mass or volume of the overflow water, as appropriate
    \begin{lablist}
    \item Calculate the buoyant force with uncertainty.
    \item Calculate the volume (with uncertainty) of the object that is submerged.
    \end{lablist}
\item Without removing any of the water from the catch beaker, continue to submerge the object and catch the overflow water until the object is completely submerged.
\item Measure the new apparent mass of the object and the mass or volume of the overflow water, as appropriate.
    \begin{lablist}
    \item Calculate the buoyant force with uncertainty.
    \item Calculate the volume (with uncertainty) of the object that is submerged.
    \item Calculate the density (with uncertainty) of the object.
    \end{lablist}
\end{lablist}

\subsubsection{Fill the Cargo-Hold}

You have been provided with a beaker and some pennies.  Use the information in the previous sections to predict the maximum number of pennies that can be gently placed into the beaker without allowing it to sink.  You should measure the mass of the beaker and the dimensions of the beaker.  The volume of a cylinder is $V=\pi r^2 h$.  You should also measure about 30 pennies (all at once) to find an average mass for the pennies.

After you have calculated your prediction with uncertainty, prepare the overflow canister with water and dry the catch beaker.  Invite the instructor to your table and allow the instructor to test your prediction.


\subsection{Questions}

\begin{enumerate}
\item Why does an object weigh a different amount when in air and when submerged in water?
\item If the string was cut on your first object (while submerged), what would be the acceleration of the object?
\item Why is it important to make certain that no air bubbles adhere to the object during the submerged weighing procedures?
\item What would the buoyant force be for an object that was immersed in a fluid with the same density as the object?
\item A floating barge filled with coal is in a lock along the Ohio River.   If the barge accidentally dumps its load into the water, will the water level in the lock rise or fall?
\item Does the mass of displaced water depend on the mass of the totally submerged object or on the volume of the submerged object?
\item There are two identical cargo ships.  One has a cargo of 5 tons of steel and the other has a cargo of 5 tons of styrofoam.  Which ship floats lower in the water?
\end{enumerate}





%-------------------------------------------------------------------------------------

\onecolumn
\section{Standing Waves on a String}
\revised{(Jan, 2013)}
\revision{(Jan, 2013)}{Revised ``connect to the vibrator''.}
\revision{(Jan, 2012)}{Fixed Wavelength plots F vs v^2 to get m/L as slope.  Added comments about stretching affecting density.}
\revision{(Aug, 2011)}{Rearranged equations for clarity.}
\revision{(Jan, 2008)}{Created}
\label{s:standingwaves}

\subsection{Introduction}

A wave is the propagation of a disturbance in a medium.  In a transverse wave, the disturbance of the medium is perpendicular to the direction of the propagation of the wave.  In this experiment, transverse waves will be propagated along a taut flexible string.  The frequency of the waves is determined by the source, the wave generator, which is driving one end of the string.   The speed of the waves, $v$, is determined by the medium itself; namely the linear mass density of the string, $(m/L)$, and the tension in the string which is determined by the static force of a mass suspended from the other end of the string, $F_T$.  This is expressed by Eq.~(\ref{eq:velten}) below.

Upon hitting at the fixed end of the string, the transverse waves will reflect back along the string.  If the end of the string is rigidly held then the reflected waves will be inverted $180^\circ$.  The initial propagated waves and the reflected waves will interfere constructively and destructively.  At particular string tensions and/or wave frequencies, this wave superposition will give rise to waves called standing waves or stationary waves.  There are many different standing waves patterns (normal modes) possible which have different wavelengths.  The different modes are characterized by the number of nodes or antinodes in the wave.  At one of the normal modes of oscillation, the amplitude of oscillation can become rather large, much larger than the amplitude of the original propagated wave.  This phenomenon is called resonance.  In this experiment, each end of the string will be close to a node position.  The fundamental frequency (mode n=1) exhibits a wave pattern with one antinode.  The second harmonic (mode n=2) exhibits a wave pattern with two antinodes, and so forth.

\subsection{Experimental Objectives}

In this experiment, the conditions required for the production of standing waves in a string will be investigated in order to
\begin{lablist}\itemsep 0in
\item[a)] empirically verify the relationship between the frequency ($f$), wavelength $(\lambda)$, and speed of a standing wave $(v)$, for a series of normal modes of oscillation, and
\item[b)] use the two relationships of the speed of the wave
 \begin{eqnarray}
 v & = & \sqrt{\frac{F_T}{m/L}} \label{eq:velten} \\
 v & = & \lambda f \label{eq:velwave}
 \end{eqnarray}
 to empirically relate the string tension $(F_T)$ and the wave velocity $(v)$, for one mode of oscillation, specifically the fundamental mode.
\end{lablist}

\subsection{Pre-Lab Work}

\begin{lablist}\itemsep 0in
\item Define the following terms:   standing wave, node, and antinode.
\item Draw a set of pictures of a standing wave on a string with two fixed ends showing the fundamental frequency (n=1), and the harmonics n=2, 3, 4, and 5.  Label a wavelength, the nodes, and the antinodes on each picture.
\item Given Eq.~(\ref{eq:velten}), predict what will happen to the velocity of the wave as the static hanging mass (providing the tension) increases, decreases, or stays the same.
\item Given Eq.~(\ref{eq:velwave}), how can measurements of $\lambda$ and $f$ be plotted to determine the wave velocity from the graph? (There is more than one way to do this; one way in particular produces a line.)  How should the axes of the graph be labeled?  Explain your reasoning.
%\item Given Eqs.~(\ref{eq:velten}) and~(\ref{eq:velwave}), create an equation that relates the tension, $F_T$ to the frequency, $f$.  How would you graph this to get a line and how would you interpret the slope and intercept?
\end{lablist}

\subsection{Fixed Tension}

In order to verify Eq.~(\ref{eq:velwave}), you can fix $F_T$ and compare multiple $f$ to the corresponding $\lambda$.

\subsubsection{Procedure}

\begin{lablist}
\item If you have not already done so, measure the mass and length of the string in order to calculate the linear mass density ($m/L$).  You might note Question~\ref{q:stretching}.
\item A mechanical vibrator, driven with a (variable frequency) function generator, will be the wave source.  Use a string slightly greater than the length of the table.  Connect one end of the string to a post such that the vibrator can wiggle the string near that end.  Hang the other end over a pulley.  Keeping the string horizontal by adjusting the height of the pulley and the location of connection at the other end.  The height of the vibrator will determine the height you need the ends to be.
\item Set up the system with some particular string tension (like 450 grams).  This tension will be kept constant for this entire part.
\item Produce 5 different normal modes by changing the frequency of the vibrator.  Adjust the standing wave amplitude until it is at its maximum.  Record the corresponding frequency for each normal mode.
\item Measure the corresponding wavelengths for each normal mode, directly from the string with a meter stick.  The vibrator end of the string is not a true node because the string is vibrating a small amount.  Take this into account when measuring the wavelengths.
\item Try touching the string at a node and at an antinode.  What happens to the wave in each case?
\item Optional:  Try measuring the frequency of the vibrator with a strobe lamp.
\end{lablist}

\subsubsection{Analysis}

\begin{lablist}
\item Determine the mathematical relation between the wavelength of the standing wave and the frequency of the vibrator that will produce a straight line ($y=mx+b$) when graphed.
\item Comment on the significance (and units) of the slope and intercept of this graph.  Determine both the precision and the accuracy of the slope and of the intercept; are they consistent with what you expect?  (Your expectation should be based on the ``other'' equation which we are not testing here.)
\item Use one or both of these (slope and intercept) to determine the velocity (with uncertainty) of the standing wave.  Relate this (\%-error) to the expected value based on the tension you chose for your specific string.
\end{lablist}


%\newpage

\subsection{Fixed Wavelength}

In order to verify Eq.~(\ref{eq:velten}), you can fix $\lambda$ and compare multiple $F_T$ to the corresponding $v$.

\subsubsection{Procedure}

\begin{lablist}
\item If you have not already done so, measure the mass and length of the string in order to calculate the linear mass density ($m/L$).  You might note Question~\ref{q:stretching}.
\newpage
\item A mechanical vibrator, driven with a (variable frequency) function generator, will be the wave source.  Use a string slightly greater than the length of the table.  Connect one end of the string to a post such that the vibrator can wiggle the string near that end.  Hang the other end over a pulley.  Keeping the string horizontal by adjusting the height of the pulley and the location of connection at the other end.  The height of the vibrator will determine the height you need the ends to be.
\item Place 150 grams (including the hanger) on the end of the string.  (If you start with too little mass, the string stretches different amounts -- changing the string's effective density -- during the experiment, causing unaccounted for curving in the graph.)
\item Adjust the frequency of the generator until the n=2 mode is observed and is at its maximum amplitude.  Record the frequency and corresponding wave velocity.
\item Repeat the previous step for 6 hanging masses up to about 1200 grams.
\end{lablist}


\subsubsection{Analysis}

\begin{lablist}
\item Determine the mathematical relation between the hanging mass ($F_T$) and the wave velocity ($v$) that will produce a straight line ($y=mx+b$) when graphed.
\item Determine which variables the slope and intercept of this graph should be related to.  Determine both the precision and the accuracy of the slope and of the intercept; are they consistent with what you expect?  (The equation we are testing should imply what to expect.)
\end{lablist}


\subsection{Questions}

\begin{enumerate}
\item What are the necessary and sufficient conditions for the production of standing waves?
\item Is it valid to consider the tightening of the string to be the same as a change of medium?  (Does tightening the string change the way waves move or the way standing waves are produced on the string?) Why or why not?
\item\label{q:stretching} When you measure the length of string, remember that the string will be under tension (pulled on) and may have some ``give'' to it.  It may be possible for it to stretch by nearly 10\%. If you don't account for this (if you measure the unstretched length) will your value of density be incorrect too large or incorrect too small?  Can you account for this either in the value of the length or in the uncertainty of the length.  Will this affect comparison to either of your graphs?
\item Consider the Fixed Tension graph.  We noted that the vibrator end of the string should not be considered to be a node.  This means that the node might be a little in front of the post or a little behind the post and therefore your wavelength values might be a little wrong, but your error bars should already include the amount the values could be off.  Based on the error bars on your Fixed Tension graph, would shifting the wavelength data a little larger or a little smaller have made a difference to the values of slope or intercept that you calculated?  If so, would it have increased the value or decreased the value?  If not, why not?
\item If the tension and the linear density of the string remain constant, but the pulley is shifted further from the post (so that the region which is vibrating lengthens), how is the resonant wavelength affected?  How is the velocity of the wave affected?
\item What was the experimental velocity (and uncertainty) of the waves in part I of the experiment?
\item What will happen to the velocity of the wave if the tension is fixed but the frequency is changed.
\end{enumerate}



%-------------------------------------------------------------------------------------

\onecolumn
\section{Speed of Sound with a Moving Stereo Speaker}
\revised{(Jan, 2006)}
\label{s:sound-tuning-fork}

\subsection{Introduction}

Longitudinal sound waves can be produced by any vibrating source.  The frequency of these waves is determined solely by this vibrating source.  There must also be a medium (solid, liquid, or gas) in order for the sound wave to be propagated.  These waves do not travel through the medium instantaneously.  There is a finite wave speed which is determined by the characteristics of the medium.   The wave speed is not dependent on the source.  If the medium is modified, then the speed of sound is also changed.  For example, the pressure or temperature of a gas or the gas composition itself are factors which the speed of sound in the gas would depend on.  In this experiment, the speed of sound in room temperature air will be determined.

\subsection{Experimental Objectives}

The purpose of this experiment is to determine the speed of sound in air.  The measurements in this experiment will be the distance traveled by the wave and the time taken to travel that distance.  Data will be taken as a function of the distance traveled by the wave.


\subsection{Experimental Procedure}

\begin{lablist}
\item The source of the waves will be a stereo-type speaker, powered by a frequency oscillator (35 Hz square wave).   The waves will move through the normal room air.
\item A microphone will be placed to receive the waves at varied distances:  3 to 30 meters.
\item An oscilloscope will be used to measure the time taken by the wave to travel the given distance.  The output from the oscillator will be connected to the external trigger of the oscilloscope and the signal from the microphone will be connected to channel one of the oscilloscope.
\item There is only one experimental setup.  The total class data set will then be tabulated for each person to analyze.
\end{lablist}

\vspace{6pt}
\hspace{-.7cm}
\begin{minipage}[t]{3in}
\subsection{Analysis}

\begin{lablist}
\item The speed of sound in air can then be calculated from the graph of distance versus time.
\item Carry out the standard linear regression analysis.
\item Carry out appropriate uncertainty analysis for the raw data and show propagation of error analysis for the results.
\item Compare your experimental results to calculations based on the room temperature and the air pressure today.
\end{lablist}
\end{minipage}
%
\hfill
%
\begin{minipage}[t]{3in}
\subsection{Questions}

\begin{enumerate}
\item Does the time of travel of the wave depend on the frequency of the oscillator?
\item Does the speed of the wave depend on the frequency of the oscillator?
\item How does the speed of sound in air vary with the air temperature?   By how much would the results of your experiment change if you conducted the experiment outside today?
\item Should the oscillator period be greater than or less than the time of travel of the wave?  Why?
\item How does the oscilloscope start and stop the timing for this experiment?   What does triggering mean?
\end{enumerate}
\end{minipage}


%====================================================
\twocolumn
\section{Thermal Expansion}
\revised{Jan, 2012}
\revision{Jan, 2012}{Removed the Thought Experiment, which was confusing.  Added explaining how they can guess the material.}
\revision{Feb. 9, 2009}{}
\setcounter{dave}{0}
\label{s:thermal}

\begin{minipage}{\textwidth}
\subsection{Introduction}

You may have noticed that most materials expand as they are heated.
This can be seen in a wide variety of situations:  Your front door
doesn't seal out the colder air, bridges have a gap of a few inches
at each end, stuck jars are easier to open when hot water is run
over them, etc.

\end{minipage}

\subsection{Experimental Objective}

In this lab, we will use the equation for thermal expansion to identify the composition of three different rods.

\subsection{Pre-Lab Questions}
Consider the following situations:
\begin{question}
\item \label{l:length} If a single rod, $1\unit{m}$ in length, expands by $1\unit{mm}$ when heated a specified amount,
how much does a $2\unit{m}$ rod (made of the same material) expand when heated by the same amount as the $1\unit{m}$ rod?

\item \label{l:alpha} If I have a glass jar with a metal lid and heat up both glass and lid by the same amount, do they expand the same amount?
If not, which one expands more?
%\item []
Why is it possible to open the heated jar?

\item \label{l:temp} If my single, $1\unit{m}$ metal rod is heated some amount
and expands $1\unit{mm}$, how much does that same rod expand when
heated by twice the original amount?
\end{question}

\noindent
The answers to these questions show that the expansion of a material depends on it's original size, it's material composition, and the amount that the temperature changes.
The equation for linear thermal
expansion is
%
\begin{equation} \label{eq:thermal}
\bf \Delta L = L_0 \alpha \Delta T
\end{equation}
%
and $\alpha$ (lower-case Greek-symbol alpha) is the coefficient of linear expansion, which is different for each material.

%In this lab, you will be measuring $\alpha$ for three metal rods; you should be able to deduce what those materials are.

Please note,
%\begin{lablist}
%\item
$\Delta L$ is the {\bf absolute expansion},
%\item
$\left(\frac{\Delta L}{L_0}\right)$ is the {\bf relative expansion}
(the expansion relative to the original length), and
%\item
$\left(\frac{\Delta L}{L_0}\times 100\%\right)$ is the {\bf percent expansion}
(the relative expansion written as a percentage).
%\end{lablist}

%\subsubsection{More Pre-Lab Questions}

\newpage\vspace*{1.25in}

In order to experimentally measure the value of $\alpha$, you must measure
the relative expansion and the change in temperature.
%
\begin{question}
\item Based on Eq.~(\ref{eq:thermal}), what are the units of $\alpha$?
\item For reliable temperature measurements, the entire rod should be at the same temperature.  This requires a stable temperature environment.  List three temperatures that can be conveniently maintained.
%\item What would be a convenient initial temperature, given that you need to measure the length of the rod with a meter stick?
\item \label{l:phase} Ice can be colder than $0\unit{^\circ C}$ and steam can be warmer than $100\unit{^\circ C}$.  How can we use either (specifically steam) to ensure a stable temperature environment?
\item If we immerse our meter stick into the steam bath with the rod to measure the expansion, then we \underline{cannot} trust the reading of the meter stick to accurately determine the new length of the rod\ldots WHY NOT?  What does this require of our measuring device?
\item  We have an apparatus which will allow steam to enter at one end and water to drain at the other.  At what temperature do you expect the rod to be when it is immersed in this steam-water mixture?
\item With several groups creating steam in the same room, why should we worry about the value of ``Room Temperature'' when we make a second or third measurement?
\end{question}

\subsection{The Experiment}

Using room temperature and the water-steam temperature interface, experimentally determine
the coefficient of linear expansion for three different types of metals.
Based on the color, weight, and density, guess the metal composition.
Carry out an error analysis and use your error-bars to verify the material of each rod.

\begin{htmlonly}
%
\setlength{\unitlength}{.5cm}
\begin{figure}[b]
\begin{center}
\begin{picture}(25,10)
\put(7,7){water \& steam environment}
\put(2,5){\line(1,0){20}}
\put(2,10){\line(1,0){20}}
\put(2,5){\line(0,1){5}}
\put(22,5){\line(0,1){5}}
\put(-.5,8){rod}
\put(-2,6){(immobile)}
\put(2,7.5){\oval(2,1)[l]}
\put(22,7.5){\oval(2,1)[r]}
\put(1.25,2){water $\downarrow$}
\put(3,3){\line(0,1){2}}
\put(3,3){\line(1,0){1}}
\put(4,3){\line(0,1){2}}
\put(18.25,13){steam $\downarrow$}
\put(20,10){\line(0,1){2}}
\put(20,12){\line(1,0){1}}
\put(21,10){\line(0,1){2}}
\put(23.05,8){\line(1,0){1}}
\put(23.05,8){\line(0,-1){5.65}}
\put(24.05,8){\line(0,-1){7.35}}
\put(23.55,1.5){\circle{.25}}
\put(23.05,2.35){\line(-1,0){9.15}}
\put(24.05,0.65){\line(-1,0){10.15}}
\put(13.05,1.5){\line(1,1){.85}}
\put(13.05,1.5){\line(1,-1){.85}}
\put(12.9,.5){\line(0,1){2}}
\multiput(12.9,.5)(0,.2){11}{\line(-1,0){.25}}
\multiput(12.9,.5)(0,1){3}{\line(-1,0){.5}}
\end{picture}
\caption{Apparatus for measuring the coefficient of linear thermal expansion.}
\label{f:thermal}
\end{center}
\end{figure}
%
\end{htmlonly}




%========================================
\onecolumn

\section{Specific and Latent Heats of Solids}
\revised{Jan, 2010}
\revision{Jan 7, 2010}{Added the Pre-Lab and uncertainty worksheet, must re-print the prelab page with \ans redefined in order to see the answers.}
\revision{Jan, 2009}{}
\label{s:spheat}

When multiple substances at different temperatures are placed in thermal contact,
the hotter substances lose energy (heat) while the colder ones gain energy (heat)
until thermal equilibrium is reached.  It is assumed that the sum of the heat lost
by the warmer objects is equal to the sum of the heat gained by the others:
%
\begin{equation}\label{eq:spheat}
 -Q_{\rm lost} = Q_{\rm gained}
\end{equation}

Notice that Eq.~(\ref{eq:qmct}) says that when you heat the shot,
it not only ``gets warmer'' (increases temperature), but it
{\sl also}\/ ``heats up'' (gains energy).  These are
\underline{not the same}, but they are very closely related.
The relation between the heat
content (or internal energy) and the change of temperature is
%
\begin{equation}\label{eq:qmct}
 Q = m\, c \, \Delta T
\end{equation}
%
where $m$ is the mass of the substance, $\Delta T$ is the change in the
temperature, and $c$ is the specific heat.  (The product $mc$ is also known
as the ``water equivalent.'') The specific heat of a substance is
a measure of the molecular activity within the material.  Measurements of
specific heats have played an important role in helping us to understand the
nature of matter.

Furthermore, the relation between the heat content and a change of phase is
\begin{equation}\label{eq:laheat}
Q = mL
\end{equation}
%
where $L$ is called the latent heat.  There is a latent heat associated with each phase change.
\begin{itemize}
\item {\bf Latent heat of fusion, $L_f$}, refers to the heat associated with the liquid/solid phase change.
\item {\bf Latent heat of vaporization, $L_v$}, refers to the heat associated with the vapor/liquid phase change.
\end{itemize}

For example, given one ice cube ($5\unit{g}$) at $-10^\circ\,{\rm C}$, which is to
be warmed to $30^\circ\,{\rm C}$, we must {\bf warm}\/ the ice to the melting point,
{\bf melt}\/ it, and then {\bf warm}\/ the {\it melted ice} some more.  (There are three separate terms.)  The heat required to do so is
\begin{eqnarray*}
Q & \!\! = \!\! & m c_{\rm ice} \, \Delta T + m L_f + m c_{\rm water} \, \Delta T \\
Q & \!\! = \!\! & (5\,{\rm g})(c_{\rm ice})[(0^\circ\,{\rm C}) - (-10^\circ\,{\rm C})]
      + (5\,{\rm g})L_f
      \\ &&
      + (5\,{\rm g})(c_{\rm water})[(30^\circ\,{\rm C}) - (0^\circ\,{\rm C})]
\end{eqnarray*}
%
This expression gives the amount of heat required $(Q_{\rm gained})$ regardless of how the ice is
warmed.  It does address the heat lost $(Q_{\rm lost})$ by whatever warmed the ice.

%========================================
%\onecolumn

\section*{Specific Heat Pre-Lab Exercise}
%\revised{Feb 23, 2009}
\setcounter{dave}{0}


% NO ANSWER
\newcommand{\ans}[2]{#2}
% PRINT ANSWER
%\newcommand{\ans}[2]{{\color{blue}{#1}}}

\newcommand{\relunc}[1]{\displaystyle \frac{\delta #1}{#1}}

The point of this worksheet is to remind you how to use your measurements and calculations to create an intelligent analysis.  This exercise walks through the calculation for the Specific Heat lab and helps you decide how to analyze the data.
%
When you calculate the specific heat based on measurements from the lab, you will use Eq.~(\ref{eq:spheat}):
%
\begin{eqnarray*}
- Q_{\rm brass} & = & Q_{\rm Al cup} + Q_{\rm water} \\
-m_b c_b \Delta T_b & = & m_a c_a \Delta T_a + m_w c_w \Delta T_w
\end{eqnarray*}
%
solving this for $c_b$, we get:
%
\[ c_b = \frac{m_a c_a \left(T_f - T_{ia}\right) + m_w c_w \left(T_f - T_{iw}\right)}{-m_b \left(T_f - T_{ib}\right)} \]
%

\fbox{\begin{minipage}{\textwidth}
The ``error analysis'' for this lab is somewhat more involved than most of the previous labs.
In order to ensure your familiarity with the process, there is a Pre-Lab worksheet appended to the end of the lab manual.  It is the last page so that you can tear it off without ruining the rest of the manual.  Please use the formula above and the worksheet as a guide to calculate the specific heat, $c$ of brass based on the measurements provided on the worksheet. \\

After you have completed the calculations, you should answer the pre-lab questions below in your lab notebook.  These are written specifically to point out the comparisons that you will need to make when you write up your analysis of the data.
\end{minipage}
}





%\newpage
\subsection*{Pre-Lab Questions}
\begin{question}
\item When calculating $Q_w$, does one of the terms, $\Delta T_w$, $m_w$, or $c_w$, have a significantly larger relative uncertainty?  If so, what is it about that measurement that made that uncertainty large while the others were small?  Is this uncertainty improvable?
\item When calculating $Q_a$, does one of the terms, $\Delta T_a$, $m_a$, or $c_a$, have a significantly larger relative uncertainty?  If so, what is it about that measurement that made that uncertainty large while the others were small?  Is this uncertainty improvable?
\item When calculating the numerator, does one of the terms, $Q_w$ or $Q_a$, have a significantly larger uncertainty?  If so, which measurement caused that uncertainty to be large?  (See questions 1 and 2.)
\item When calculating the denominator, does one of the terms, $\Delta T_b$ or $m_b$, have a significantly larger relative uncertainty?  If so, which measurement caused that uncertainty to be large?  Is this uncertainty improvable?
\item Which term, $(Q_w+Q_a)$ or $m_b \Delta T_b$, had the largest relative uncertainty?  What is it about that measurement that made that uncertainty large while the others were small?  Is this uncertainty improvable?
\item Which measurement ultimately caused the size of $\delta c$ to be as large as it was?  Is there a way to reduce the relative uncertainty by planning the experiment differently?  For example, should you use a different mass of water or of metal?  Should you start at a different temperature?  Should the final temperature be higher or lower?  How might one accomplish this?
\item When you do any experiment, a significant component of the analysis should be devoted to the question: For each measurement, if you measured the value to be higher than the true value, then how does this affect the final result?  To that end, for this case, answer the following:
    \begin{enumerate}
    \item If all masses are skewed high by the scale, then is $c$ affected?
    \item If all temperatures are skewed high by the thermometer, then is $c$ affected?
    \item If some heat escaped into the atmosphere, lost to the experiment, then how does this affect the final temperature of the mixture and how does this affect the final value of $c$?
    \end{enumerate}
\end{question}


\newpage

\section*{Laboratory Experiment}


%\vfill

\subsection{Experimental Objectives}

First, determine the composition of a metallic cylinder based on its measured specific heat capacity.
Second, determine the latent heat of fusion for water to verify that the equations provided describe the process.

%\vfill

%\newpage

\subsection{The Specific Heat Capacity of an Unknown Metal}

The following considerations should allow you to experimentally determine (with
uncertainty) the specific heat of the metal and, from that, the composition of the metal.
Use these questions to determine your procedure, with notes about where to be careful.
Verify the procedure with the professor before beginning the experiment.
%
\begin{enumerate}
\item We would like the metal to be at some warm, stable, and uniform temperature.  Can you imagine what convenient temperature we can warm the metal to which satisfies these three conditions?  Hint: In order for the cylinder to be at a uniform temperature, it must sit in this environment for some short time; since we don't know the temperature of a burner, we can't just place the metal on the burner itself.
\item We would like the water to be at some cool, stable, and uniform temperature.  Can you imagine what convenient temperature we can maintain for the water which satisfies these three conditions?  HINT: The container for the water must also be in thermal equilibrium with the water before the unknown metal is added.
\item[] {\sc \bf Assume}: The aluminum cup is always at the same temperature as the water.
\item The specific heat is an inherent property of a material.  Since the cup is aluminum, you can easily find it's specific heat with a CRC, or a textbook, or it may also be stamped onto your equipment.
\item When the metal and water (with container) are placed in thermal contact, they must be isolated from the external environment.  With what equipment will this be done?  (See your lab table.)
\item If some hot metal is held in the air, it cools down.  What happens to the heat when hot metal is exposed to the air?  Can we easily measure the heat lost to the air?  How will this affect your experimental technique?
\item Think about Eq.~(\ref{eq:qmct}) and the likely values of $c$ for the water and for the unknown metal.  If there are equal masses of water and metal, then where (roughly) do you expect the final temperature to be  (closer to $T_i$ for water or $T_i$ for metal)?  What if you have more water?
\item If the final temperature is far from room temperature and the calorimeter is not well insulated, then where will the actual final temperature be?  If the final temperature value is wrong in this way, then will your calculated $c$ for the metal be wrong too high or wrong too low?
\end{enumerate}
\setcounter{dave}{\arabic{enumi}}

\newpage

\subsection{The Latent Heat of Fusion for Water}

The following considerations should allow you to experimentally determine (with
uncertainty) the latent heat of water.
Use these questions to determine your procedure, with notes about where to be careful.
Verify the procedure with the professor before beginning the experiment.
%
\begin{enumerate}\setcounter{enumi}{\arabic{dave}}
\item We would like the water to be at some warm, stable, and uniform temperature.  Can you imagine what convenient temperature we can maintain for the water which satisfies these three conditions?
\item We would like the ice to be at some cool, stable, uniform, and known temperature.  Can you imagine what convenient temperature we can maintain for the ice which satisfies these three conditions?  HINT: Most freezers are colder than freezing.
\item[] Again, {\sc \bf assume}: The aluminum cup is always at the same temperature as the water.
\item When the ice and water (with container) are placed in thermal contact, they must be isolated from the external environment.  With what equipment will this be done?  (See your lab table.)
\item You will need to know the mass of the ice added.  If you take the time to weigh it on a scale, it will melt.  How can we determine the mass of the ice added without explicitly placing it on the scale before adding to the water?
\item Based on the expected value of the latent heat of fusion for water, how much ice should you use compared to the amount of water that you have?
\item When you add the ice, the $mL$ term should only include the mass of the ice that will melt in the water.  If you add water with your ice, then this value of $m$ will not be accurate.  How does this fact affect your procedure for adding ice?
\item If the final temperature is far from room temperature and the calorimeter is not well insulated, then where will the actual final temperature be?  If the final temperature value is wrong in this way, then will your calculated $L$ be wrong too high or wrong too low?
\end{enumerate}
\setcounter{dave}{\arabic{enumi}}








\onecolumn%========================================

\section{Electric Field Lines}
\revised{(March, 1997)}
\label{s:EField}

\subsection{Introduction}

The effects of electric charges (positive and negative) can be seen in many electronic devices, like the radio.  The effects of static electricity can be seen when clothing is pulled out of a dryer on a winter day.  There is a force exerted on one charge by another charge and this can be either attractive or repulsive.  This force is called the Coulomb force and is named after Charles Coulomb (1736-1806).

Physics relies on abstractions (new quantities and new names), with pictures to help convey information and ideas.  The English scientist Michael Faraday (1791-1867) introduced the concept of lines of force, a force field, as an aid in visualizing the interactions of charges.  These lines of force are a mental abstraction, but they can be visualized with the use of iron filings placed near a charge or a group of charges.  These lines of force convey a picture of the interaction between one charge and another.  The iron filings will align themselves when in the presence of a single charge.  This conceptualizes the idea of the electric field strength E (electric force per unit charge) because it is convenient to know the electric force per unit charge at any point in space due to a nearby set of electric charges.   The electric field strength though can not be easily measured with a meter.

It requires work to move a charge against an electric field.  The ratio of the work done to the charge strength is called the potential difference (voltage) and it is measured in units of joules/coulomb and is called a volt.   This potential difference is easily measured with a voltmeter.  If the charge is moved along a path perpendicular to the electric field lines then there is no work done, it takes zero energy.  This is because there is no force component in the direction of the path.  The potential (voltage) is then constant along paths which are perpendicular to the field lines.  Such paths are called equipotential lines.  These equipotential lines can be measured with a simple voltmeter, and then from these the electric field lines can be deduced.

\subsection{Experimental Objectives}

In this experiment, you will map out the equipotential lines for a couple of given electric field configurations and determine the pattern of the electric lines of force for these configurations.  You will then give a physical explanation of the terms ``electric field'' and ``electric potential'' and their relationship.

\subsection{Pre-Lab Work}

\begin{lablist}
\item Define Coulomb's law.
\item Give a definition of electric field, both mathematical and pictorial.
\item Show a picture of the electric field near:
    \begin{enumerate}
    \item[a)] a point charge,
    \item[b)] two equal and opposite point charges, and
    \item[c)] two equal positive point charges.
    \end{enumerate}
\item Show from the above three cases (pictures), places where the electric field is zero.
\item Show in a picture, the electric field inside and outside of  a positively charged conductor (show that electric field lines will start from the surface of the charged conductor).                      .
\item Give an argument why two electric field lines can never cross.
\item Give an argument why an electric field line is perpendicular to the equipotential line.
\end{lablist}

\subsection{Procedure}

\begin{lablist}
\item A special conducting plate with metal terminals will serve as the charge configuration.  This plate should be fastened beneath the field-mapping board, without observing the specific charge configuration.
\item An electric field is produced when a power supply (battery) is connected between the two terminals (points X and Y).  Set the power supply at 10 volts.
\item Points of equal potential (voltage) are found using a movable U-shaped probe which is connected to a voltmeter.  Plot a series of points on your own graph paper which are at a constant potential (an equipotential line).  Then repeat this for different potentials in steps of one volt.   In this way the entire field is explored.
\item Obtain an equipotential map and the electric field lines map (on graph paper) for  three different charge configurations.   Be sure to indicate the direction of the electric field lines on your maps.
\end{lablist}

\subsection{Analysis}

\begin{lablist}
\item Discuss the relationships between the voltage measurements and the electric field lines, and between the electric field lines and the charge configuration.  Estimate the magnitude of the electric field at a few points on each map.   Please write the electric field intensities in units of volts/meter.
\item Discuss any irregularities in the field patterns which you have found.
\item Discuss the spacing and what it represents, for the equipotential lines, and for  the electric field lines.
\item Indicate on your maps the location of the positive and negative charge distributions and their approximate shapes.
\end{lablist}

\subsection{Questions}

\begin{enumerate}
\item Can equipotential lines cross?   Can electric field lines cross?   Explain.
\item Why are there direction arrows on electric field lines but not on equipotential lines?
\item Why are the lines of force always perpendicular to the equipotential lines?
\item How is the electric potential affected by an insulator, and  by a conductor, when they are placed in the electric field?
\item How is the electric field affected inside an insulator, and inside a conductor, when they are placed in the field?
\end{enumerate}




\onecolumn%========================================
\section{Internal Resistance of Batteries}
\revised{Jan. 8, 2009}
\label{s:intr}
\setcounter{dave}{0}

\subsection{Experimental Objective}

In this experiment, we would like to consider the characteristics of a voltage source.

\subsection{Introduction}

A voltage source, $V$, is anything that has a potential difference across two terminals: A liquid battery (such as in a car), a dry cell battery, a power plant (accessible through the wall socket), etc.  An ideal voltage source, ${\cal E}$, will provide the same amount of voltage regardless of how much current is drawn from it.  Physical voltage sources will approximate this to various degrees.

In order to determine how ideal a voltage source is, we will draw more and more current and measure how much voltage it is able to supply.  If we attach a large resistance to the voltage source, then it will resist the current and we will only draw off a small current.  If we attach a small resistance to the voltage source, then it will draw a large current.

Next week, you will consider the details of that relationship.  Your resistors should already be ordered largest to smallest; don't worry about their actual values, but please try to keep them in order.  Each of the resistors has multiple color bands on it.  Please be diligent about recording the colors in order on each resistor.  When you are finished with the lab, you should be able to sort them largest to smallest.



\subsection{The Equipment}\label{ss:equipment}

In the diagrams,
\txtgalv{A} is an ammeter, which measures the current in amps (A), milliamps (mA), or microamps ($\mu$A).  These measure the \underline{current through} a wire and must be placed in series with the circuit element.  Two circuit elements are ``in series'' if all of the current which goes through one also goes through the other.  Current is the flow of charge (an amount of material); it does not diminish as it passes through the circuit elements.

In the diagrams, \txtgalv{V} is a voltmeter, which measures the voltage in volts (V). These measure the \underline{voltage across} (the potential difference from before to after) a circuit element and must be placed in parallel with the circuit element.  Two circuit elements are ``in parallel'' if the current gets split between one and the other.  Voltage is related to the energy of the charges; it does diminish (or increase) as it passes through the circuit elements.

Figure~\ref{f:R1}
%
\setlength{\unitlength}{.5cm}
\begin{figure}[bh]
\begin{center}
\begin{picture}(18,6)
%
\put(0,6){Either}
%
\put(2,2.55){\circle{.4}}
\put(2,3.25){\line(0,-1){.5}}
\put(0.5,3.5){$V$}
\put(1.75,3.25){\line(1,0){.5}}
\put(1.5,3.5){\line(1,0){1}}
\put(1.75,3.75){\line(1,0){.5}}
\put(1.5,4){\line(1,0){1}}
\put(2,4){\line(0,1){.5}}
\put(2,4.7){\circle{.4}}
%
\put(5.5,-.2){\circle{.4}}
\put(5.5,.5){\line(0,-1){.5}}
\put(6.1,1.5){$R$}
\multiput(5,0.725)(0,.5){4}{\line(4,1){1}}
\multiput(6,0.5)(0,.5){4}{\line(-4,1){1}}
\put(5.5,2.475){\line(0,1){.5}}
\put(5.5,3.175){\circle{.4}}
%
\put(5.5,3.175){\circle{.4}}
\put(5.5,3.375){\line(0,1){.5}}
\put(5.5,4.375){\circle{1}}
\put(5.2,4.175){A}
\put(5.5,4.875){\line(0,1){.5}}
\put(5.5,5.575){\circle{.4}}
%
\put(2,-.2){\line(0,1){2.55}}
\put(2,-.2){\line(1,0){3.3}}
\put(2,5.575){\line(0,-1){.675}}
\put(2,5.575){\line(1,0){3.3}}
%
\put(7.5,-.2){\line(0,1){2.2}}
\put(7.5,-.2){\line(-1,0){1.8}}
\put(7.5,2.5){\circle{1}}
\put(7.2,2.3){V}
\put(7.5,5.575){\line(0,-1){2.575}}
\put(7.5,5.575){\line(-1,0){1.8}}
%
%
\put(9,6){Or}
%
\put(11,2.55){\circle{.4}}
\put(11,3.25){\line(0,-1){.5}}
\put(9.5,3.5){$V$}
\put(10.75,3.25){\line(1,0){.5}}
\put(10.5,3.5){\line(1,0){1}}
\put(10.75,3.75){\line(1,0){.5}}
\put(10.5,4){\line(1,0){1}}
\put(11,4){\line(0,1){.5}}
\put(11,4.7){\circle{.4}}
%
\put(14.5,-.2){\circle{.4}}
\put(14.5,.5){\line(0,-1){.5}}
\put(15.1,1.5){$R$}
\multiput(14,0.725)(0,.5){4}{\line(4,1){1}}
\multiput(15,0.5)(0,.5){4}{\line(-4,1){1}}
\put(14.5,2.475){\line(0,1){.5}}
\put(14.5,3.175){\circle{.4}}
%
\put(14.5,3.175){\circle{.4}}
\put(14.5,3.375){\line(0,1){.5}}
\put(14.5,4.375){\circle{1}}
\put(14.2,4.175){A}
\put(14.5,4.875){\line(0,1){.5}}
\put(14.5,5.575){\circle{.4}}
%
\put(11,-.2){\line(0,1){2.55}}
\put(11,-.2){\line(1,0){3.3}}
\put(11,5.575){\line(0,-1){.675}}
\put(11,5.575){\line(1,0){3.3}}
%
\put(16.5,-.2){\line(0,1){1.7}}
\put(16.5,-.2){\line(-1,0){1.8}}
\put(16.5,2.0){\circle{1}}
\put(16.2,1.8){V}
\put(16.5,3.175){\line(0,-1){0.675}}
\put(16.5,3.175){\line(-1,0){1.8}}
%
\end{picture}
\caption{Determining Ohm's Law with a single resistor.}
\label{f:R1}
\end{center}
\end{figure}
%]
shows two configurations of ammeter and voltmeter.  The small circles are the connections for the wires.
In this lab, we are trying to measure the current drawn from the battery and the voltage output by the battery.
On the left, the voltmeter is in parallel with the battery, but the ammeter is not in series with the battery.
($V$ is correct, but $I$ is too small.)
On the right, the ammeter is in series with the battery, but the voltmeter is not in parallel with the battery.
($I$ is correct, but $V$ is too small.)
You will be making a mistake either way, but hopefully neither is {\sl too wrong}.
For only your largest and smallest resistors, measure this both ways and see how large the effect is.
(It is incorrect to use the voltage from one circuit and the current from the other.)



\subsection{The Experiment}

You should consider (one at a time, of course) three voltage sources: a ``good'' dry-cell battery, a ``bad'' dry-cell battery, and a dc-power supply.  To determine the characteristics of each battery, you will need to build the circuit in Figure~\ref{f:R1} using some resistor, and then replace the resistor with 10 different resistance values, in order to vary the amount of current drawn.  Measure the current (from the ammeter \txtgalv{A}) and the terminal voltage (from the voltmeter \txtgalv{V}) for each of these resistors.  Figure out how the terminal voltage is related to the current drawn by graphing the voltage $V$ versus the current $I$ and considering the equation of the best trendline.


\subsection{The Analysis}

Comment on the following in your lab notebook, and use this information to enhance your analysis of the data.
%
\begin{enumerate}\setcounter{enumi}{\thedave}
\item Does the good battery provide a constant voltage?  If so, tell the instructor to find smaller resistors for you. If not, try to figure out what is happening.
\item Discuss the units and the interpretation of the slope and intercept for each graph.
\item Compare the values of slope and intercept from the three graphs.  It may be that they should not have the same value; but correlate the values to the voltage sources.  If they have comparable voltages, then graph them on the same chart.
\end{enumerate}
\setcounter{dave}{\arabic{enumi}}
%

On your graph, extrapolate the trendline to determine what would happen if you had zero resistance.  ({\bf Do not} actually use zero resistance; that would use up the battery ``juice" {\bf very} quickly.)
%
\begin{enumerate}\setcounter{enumi}{\thedave}
\item Which portion of the graph is large $R$ and small $R$?  How do you know?
\item In principle, if there were absolutely no resistance to the flow of current, how much current would flow?
\item Based on your graph, if you were to allow the resistance to be zero in the circuit, what would happen to the current?  (This is called the ``short-circuit current.'')
\item Can you draw any conclusion about the resistance that is internal to the battery?  (Be as quantitative as possible.)
\item Based on your graph, what characteristic distinguishes a ``good'' battery from a ``bad'' battery?
\end{enumerate}
\setcounter{dave}{\arabic{enumi}}
%

\vfill


\onecolumn%========================================

\section[Ohm's Law]{Using Ohm's Law to Determine Equivalent Resistance}
\revised{Jan, 2006}
\label{s:Req}
\setcounter{dave}{0}

\subsection{Experimental Objective}

In this lab, after empirically verifying Ohm's Law, you would like to empirically determine the equivalent resistance of two resistors in series and of those two resistors in parallel.

\subsection{Introduction}

When current moves through a wire due to an electrical potential difference (a voltage),
it is literally electric charges falling through the wire due to an electrical field.
This is completely analogous to gravitational charges (masses) falling through the air due to a gravitational field.  Different types and sizes (gauges) of wire resist this current by different amounts.  Ohm's Law describes (for some materials) just how this resistance affects a current for a specified voltage, i.e., it relates the current to the voltage.  Using the equipment in the lab, you too will be able to discover Ohm's law!  Hooray!
Furthermore, we can investigate the effect of {\sl multiple}\/ resistors on a current in some voltage.

For each of the cases outlined below, we will measure at least eight voltage values (from a dc-power supply) and the corresponding current for a specific resistor or combination of resistors.

\subsection{Pre-Lab}

Answer the following questions before you come into lab.
%
\begin{enumerate}\setcounter{enumi}{\thedave} \itemsep 0in
\item Look up Ohm's Law.  If you plot $V$ versus $I$, what do you expect the graph to look like?
\item What units should the slope and intercept have?
\item What values do you expect the slope and intercept to have?
\item Draw one circuit diagram each for resistors in series and for resistors in parallel.
\item Look up the resistor color code in your textbook.
\end{enumerate}
\setcounter{dave}{\arabic{enumi}}
%


\subsection{The Equipment}\label{ss:equipment2}

In the diagrams,
\txtgalv{A} is an ammeter, which measures the current in amps (A), milliamps (mA), or microamps ($\mu$A).  These measure the \underline{current through} a wire and must be placed in series with the circuit element.  Two circuit elements are ``in series'' if all of the current which goes through one also goes through the other.  Current is the flow of charge (an amount of material); it does not diminish as it passes through the circuit elements.

In the diagrams, \txtgalv{V} is a voltmeter, which measures the voltage in volts (V). These measure the \underline{voltage across} (the potential difference from before to after) a circuit element and must be placed in parallel with the circuit element.  Two circuit elements are ``in parallel'' if the current gets split between one and the other.  Voltage is related to the energy of the charges; it does diminish (or increase) as it passes through the circuit elements.

Figure~\ref{f:R1}
%
\setlength{\unitlength}{.5cm}
\begin{figure}[bh]
\begin{center}
\begin{picture}(18,6)
%
\put(0,6){Either}
%
\put(2,2.55){\circle{.4}}
\put(2,3.25){\line(0,-1){.5}}
\put(0.5,3.5){$V$}
\put(1.75,3.25){\line(1,0){.5}}
\put(1.5,3.5){\line(1,0){1}}
\put(1.75,3.75){\line(1,0){.5}}
\put(1.5,4){\line(1,0){1}}
\put(2,4){\line(0,1){.5}}
\put(2,4.7){\circle{.4}}
%
\put(5.5,-.2){\circle{.4}}
\put(5.5,.5){\line(0,-1){.5}}
\put(6.1,1.5){$R$}
\multiput(5,0.725)(0,.5){4}{\line(4,1){1}}
\multiput(6,0.5)(0,.5){4}{\line(-4,1){1}}
\put(5.5,2.475){\line(0,1){.5}}
\put(5.5,3.175){\circle{.4}}
%
\put(5.5,3.175){\circle{.4}}
\put(5.5,3.375){\line(0,1){.5}}
\put(5.5,4.375){\circle{1}}
\put(5.2,4.175){A}
\put(5.5,4.875){\line(0,1){.5}}
\put(5.5,5.575){\circle{.4}}
%
\put(2,-.2){\line(0,1){2.55}}
\put(2,-.2){\line(1,0){3.3}}
\put(2,5.575){\line(0,-1){.675}}
\put(2,5.575){\line(1,0){3.3}}
%
\put(7.5,-.2){\line(0,1){2.2}}
\put(7.5,-.2){\line(-1,0){1.8}}
\put(7.5,2.5){\circle{1}}
\put(7.2,2.3){V}
\put(7.5,5.575){\line(0,-1){2.575}}
\put(7.5,5.575){\line(-1,0){1.8}}
%
%
\put(9,6){Or}
%
\put(11,2.55){\circle{.4}}
\put(11,3.25){\line(0,-1){.5}}
\put(9.5,3.5){$V$}
\put(10.75,3.25){\line(1,0){.5}}
\put(10.5,3.5){\line(1,0){1}}
\put(10.75,3.75){\line(1,0){.5}}
\put(10.5,4){\line(1,0){1}}
\put(11,4){\line(0,1){.5}}
\put(11,4.7){\circle{.4}}
%
\put(14.5,-.2){\circle{.4}}
\put(14.5,.5){\line(0,-1){.5}}
\put(15.1,1.5){$R$}
\multiput(14,0.725)(0,.5){4}{\line(4,1){1}}
\multiput(15,0.5)(0,.5){4}{\line(-4,1){1}}
\put(14.5,2.475){\line(0,1){.5}}
\put(14.5,3.175){\circle{.4}}
%
\put(14.5,3.175){\circle{.4}}
\put(14.5,3.375){\line(0,1){.5}}
\put(14.5,4.375){\circle{1}}
\put(14.2,4.175){A}
\put(14.5,4.875){\line(0,1){.5}}
\put(14.5,5.575){\circle{.4}}
%
\put(11,-.2){\line(0,1){2.55}}
\put(11,-.2){\line(1,0){3.3}}
\put(11,5.575){\line(0,-1){.675}}
\put(11,5.575){\line(1,0){3.3}}
%
\put(16.5,-.2){\line(0,1){1.7}}
\put(16.5,-.2){\line(-1,0){1.8}}
\put(16.5,2.0){\circle{1}}
\put(16.2,1.8){V}
\put(16.5,3.175){\line(0,-1){0.675}}
\put(16.5,3.175){\line(-1,0){1.8}}
%
\end{picture}
\caption{Determining Ohm's Law with a single resistor.}
\label{f:R1}
\end{center}
\end{figure}
%]
shows two configurations of ammeter and voltmeter.  The small circles are the connections for the wires.
In this lab, we are trying to measure the current through the resistor and the voltage across the resistor.
On the left, the ammeter is in series with the resistor, but the voltmeter is not in parallel with the resistor.
($I$ is correct, but $V$ is too large.)
On the right, the voltmeter is in parallel with the resistor, but the ammeter is not in series with the resistor.
($V$ is correct, but $I$ is too large.)
You will be making a mistake either way, but hopefully neither is {\sl too wrong}.
For only one of your resistors, measure the data both ways and see how large the effect is.
(It is incorrect to use the voltage from one circuit and the current from the other.)


\subsection{The Experiment}

Please note Section~\ref{ss:equipment2} regarding the equipment as arranged in Figure~\ref{f:R1}.
%
For one of your resistors arranged according to one of the circuits in Figure~\ref{f:R1}, measure the current (from the ammeter \txtgalv{A}) for several arbitrary and somewhat-evenly-spaced voltage values (as measured from \txtgalv{V}).  Verify Ohm's Law by graphing the voltage $V$ versus the current $I$ and considering the equation of the best trendline.  Repeat this for the second resistor.  For at least one of your resistors, repeat this for the other configuration in Figure~\ref{f:R1}.  Answer the first set of questions in Sec.~\ref{ss:ohm-questions}.

% Edited

Build the circuit in Figure~\ref{f:series} using your two resistors.  Measure $I$ from \txtgalv{A} for several $V$.  Graph the voltages $V_1$, $V_2$, and $V_t$ on the same graph, all  versus the current $I$ through the resistor. Consider the equations of the best trendlines.
Answer questions~\ref{q:series-eq}, \ref{q:series-bigsmall}, and \ref{q:series-relation} about this data and graph.

Build the circuit in Figure~\ref{f:parallel} using your two resistors.
Measure $I$ from each \txtgalv{A} ($I_1$, $I_2$, and $I_t$) for several $V$.
All on the same graph, plot the voltage $V$ versus each current.
Consider the equations of the best trendlines.
Answer questions~\ref{q:parallel-eq}, \ref{q:parallel-bigsmall}, and \ref{q:parallel-relation} about this data and graph.
%
\begin{figure}[t]
\begin{minipage}{3in}
\setlength{\unitlength}{.5cm}
\begin{center}
\begin{picture}(12,8)
%
\put(2,2.55){\circle{.4}}
\put(2,3.25){\line(0,-1){.5}}
\put(0.5,3.5){$V$}
\put(1.75,3.25){\line(1,0){.5}}
\put(1.5,3.5){\line(1,0){1}}
\put(1.75,3.75){\line(1,0){.5}}
\put(1.5,4){\line(1,0){1}}
\put(2,4){\line(0,1){.5}}
\put(2,4.7){\circle{.4}}
%
\put(5.5,3.175){\circle{.4}}
\put(5.5,3.875){\line(0,-1){.5}}
\put(6.1,4.875){$R_1$}
\multiput(5,4.1)(0,.5){4}{\line(4,1){1}}
\multiput(6,3.875)(0,.5){4}{\line(-4,1){1}}
\put(5.5,5.85){\line(0,1){.5}}
\put(5.5,6.55){\circle{.4}}
%
\put(5.5,-.2){\circle{.4}}
\put(5.5,.5){\line(0,-1){.5}}
\put(6.1,1.5){$R_2$}
\multiput(5,0.725)(0,.5){4}{\line(4,1){1}}
\multiput(6,0.5)(0,.5){4}{\line(-4,1){1}}
\put(5.5,2.475){\line(0,1){.5}}
\put(5.5,3.175){\circle{.4}}
%
% redo this horizontal
\put(3.1,6.55){\circle{.4}}
\put(3.8,6.55){\line(-1,0){.5}}
\put(4.3,6.55){\circle{1}}
\put(4.0,6.35){$A$}
\put(5.3,6.55){\line(-1,0){.5}}
\put(5.5,6.55){\circle{.4}}
%
\put(2,-.2){\line(0,1){2.55}}
\put(2,-.2){\line(1,0){3.3}}
\put(2,6.55){\line(0,-1){1.65}}
\put(2,6.55){\line(1,0){.9}}
%
\put(10.5,-.3){\line(0,1){2.3}}
\put(10.5,-.3){\line(-1,0){4.8}}
\put(10.5,2.5){\circle{1}}
\put(10.2,2.3){$V_t$}
\put(10.5,6.65){\line(0,-1){3.65}}
\put(10.5,6.65){\line(-1,0){4.8}}
%
\put(8.0,3.275){\line(0,1){1.6}}
\put(8.0,3.275){\line(-1,0){2.3}}
\put(8.0,5.375){\circle{1}}
\put(7.7,5.175){$V_1$}
\put(8.0,6.45){\line(0,-1){0.675}}
\put(8.0,6.45){\line(-1,0){2.3}}
%
\put(8.0,-.1){\line(0,1){1.6}}
\put(8.0,-.1){\line(-1,0){2.3}}
\put(8.0,2.0){\circle{1}}
\put(7.7,1.8){$V_2$}
\put(8.0,3.075){\line(0,-1){0.575}}
\put(8.0,3.075){\line(-1,0){2.3}}
%
\end{picture}
\caption{Using Ohm's Law with multiple resistors in series to find an equivalent resistance.}
\label{f:series}
\end{center}
\end{minipage}
%
\hfill
%
\begin{minipage}{3in}
\setlength{\unitlength}{.5cm}
\begin{center}
\begin{picture}(18,8)
%
\put(2,2.55){\circle{.4}}
\put(2,3.25){\line(0,-1){.5}}
\put(0.5,3.5){$V$}
\put(1.75,3.25){\line(1,0){.5}}
\put(1.5,3.5){\line(1,0){1}}
\put(1.75,3.75){\line(1,0){.5}}
\put(1.5,4){\line(1,0){1}}
\put(2,4){\line(0,1){.5}}

\put(2,4.7){\circle{.4}}
%
\put(7.5,-.2){\circle{.4}}
\put(7.5,.5){\line(0,-1){.5}}
\put(8.1,1.5){$R_1$}
\multiput(7,0.725)(0,.5){4}{\line(4,1){1}}
\multiput(8,0.5)(0,.5){4}{\line(-4,1){1}}
\put(7.5,2.475){\line(0,1){.5}}
\put(7.5,3.175){\circle{.4}}
%
\put(7.5,3.175){\circle{.4}}
\put(7.5,3.375){\line(0,1){.5}}
\put(7.5,4.375){\circle{1}}
\put(7.1,4.175){$A_1$}
\put(7.5,4.875){\line(0,1){.5}}
\put(7.5,5.575){\circle{.4}}
% redo this horizontal
\put(5.1,5.575){\circle{.4}}
\put(5.8,5.575){\line(-1,0){.5}}
\put(6.3,5.575){\circle{1}}
\put(5.9,5.375){$A_t$}
\put(7.3,5.575){\line(-1,0){.5}}
\put(7.5,5.575){\circle{.4}}
%
\put(10.5,-.2){\circle{.4}}
\put(10.5,.5){\line(0,-1){.5}}
\put(11.1,1.5){$R_2$}
\multiput(10,0.725)(0,.5){4}{\line(4,1){1}}
\multiput(11,0.5)(0,.5){4}{\line(-4,1){1}}
\put(10.5,2.475){\line(0,1){.5}}
\put(10.5,3.175){\circle{.4}}
%
\put(10.5,3.175){\circle{.4}}
\put(10.5,3.375){\line(0,1){.5}}
\put(10.5,4.375){\circle{1}}
\put(10.1,4.175){$A_2$}
\put(10.5,4.875){\line(0,1){.5}}
\put(10.5,5.575){\circle{.4}}
%
\put(2,-.2){\line(0,1){2.55}}
\put(2,-.2){\line(1,0){5.3}}
\put(2,5.575){\line(0,-1){.675}}
\put(2,5.575){\line(1,0){2.85}}
\put(7.7,-.2){\line(1,0){2.55}}
\put(7.7,5.575){\line(1,0){2.55}}
%
\put(12.5,-.2){\line(0,1){2.2}}
\put(12.5,-.2){\line(-1,0){1.8}}
\put(12.5,2.5){\circle{1}}
\put(12.2,2.3){V}
\put(12.5,5.575){\line(0,-1){2.575}}
\put(12.5,5.575){\line(-1,0){1.8}}
%
%
\end{picture}
\caption{Using Ohm's Law with multiple resistors in parallel to find an equivalent resistance.}
\label{f:parallel}
\end{center}
\end{minipage}
\end{figure}


\subsection{Questions}\label{ss:ohm-questions}

All of the following questions should be considered in your lab notebook and used to enhance the analysis of your report.

\bigskip
\noindent
Regarding the individual resistors, answer the following questions:
%
\begin{enumerate}\setcounter{enumi}{\thedave} \itemsep 0in
\item Based on the color codes, do your resistor values agree with your graphs?
\item At the end of lab, the instructor will show you how to use the multimeter as an ohm-meter.  Measure the resistance of your resistors and compare this value to the graphs.
\item Did the two configurations in Figure~\ref{f:R1} give different values?
\end{enumerate}
\setcounter{dave}{\arabic{enumi}}

\bigskip
\noindent
Regarding the series resistors, answer the following questions:
%
\begin{enumerate}\setcounter{enumi}{\thedave} \itemsep 0in
\item The series graph should have a plot of the individual resistors as well as the series combination.  Do the individual series resistors on this graph agree with the individual values that you found on the plot in the first part?  Are they too large or too small?
\item\label{q:series-eq} Use Ohm's Law to decide on the single equivalent resistor which could replace your two resistors without changing the current-voltage relationship.
\item\label{q:series-bigsmall} Is the equivalent resistor larger than or smaller than the individual resistances?
\item\label{q:series-relation} Determine a relationship between your resistors, $R_1$ and $R_2$, and the equivalent resistance, $R_{\rm eq}$.
\end{enumerate}
\setcounter{dave}{\arabic{enumi}}
%
Regarding the parallel resistors, answer the following questions:
%
\begin{enumerate}\setcounter{enumi}{\thedave} \itemsep 0in
\item The parallel graph should have a plot of the individual resistors as well as the parallel combination.  Do the individual series resistors on this graph agree with the individual values that you found on the plot in the first part?  Are they too large or too small?
\item\label{q:parallel-eq} Use Ohm's Law to decide on the single equivalent resistor which could replace your two resistors without changing the current-voltage relationship.
\item\label{q:parallel-bigsmall} Is the equivalent resistor larger than or smaller than the individual resistances?
\item\label{q:parallel-relation} Determine a relationship between your resistors, $R_1$ and $R_2$, and the equivalent resistance, $R_{\rm eq}$.  (Hint: It may help to consider reciprocals of certain values; it may also help to look it up in your textbook.)
\end{enumerate}
\setcounter{dave}{\arabic{enumi}}



\onecolumn%========================================

\section{RC Circuits}
\revised{Feb, 2008}
\label{s:RCdetailed}

\subsection{Background}

When you connect a resistor and capacitor in series with a power supply, the power supply will provide current that adds charge to the capacitor.  This is called an RC circuit.  The resistor will regulate the amount of current (the rate that charge builds).  The value of the capacitance (measured in Farads) will determine the amount of charge per volt that the capacitor is designed to hold.  1 Farad equals 1 Coulomb / volt (1 F = 1 C/v).  When you turn off the power supply, the capacitor will release the stored charge (it will ``discharge''), generating a current in the other direction.

A square wave generator produces a cyclic input that is on, then off, then on, etc.  This will provide a convenient method of turning the power supply on and then off, repeatedly, so that you can observe the effect of charging a capacitor and then discharging a capacitor.

\subsection{Objectives}
\begin{lablist}
\item To verify that the voltage of a capacitor increases as it charges and decreases as it discharges and to determine the equation of that increase or decrease.
\item To determine the relationship between the resistor and capacitor values and the time constant (a measure of the rate of decay) for a charging or discharging capacitor in an RC circuit.
\end{lablist}

\subsection{Procedure}

\begin{lablist}
\item BEFORE constructing the circuit, use a multimeter to measure the resistance ($R$) and capacitance ($C$) of the two elements that you will use in your circuit.
    % Use $C \approx 14\unit{\mu F}$ and $R \approx 3\unitk\ohm}$.
    Compute the time constant, $\tau = RC$.  Record this value.   For convenient measurement, we would like the period of the wave to be about 10 time constants ($T=10 RC$).   So, set the frequency of the function generator to be $f = \txtfrac{1}{T} = \txtfrac{1}{(10 RC)}$.
\item Using the function generator, take a data set (through the Pasco 750 interface) of the square wave input voltage (by itself) versus time.   While viewing this, set the maximum voltage to about $V_{\rm max} = 5$volts and adjust the dc offset to set the lower voltage to zero volts.  Notice that the square wave alternates between a non-zero value (your 5 volts) and a zero value.  When the square wave has a non-zero value (5 volts), it will charge the capacitor.  When the square wave has zero value, it will allow the capacitor to discharge.
\item Set up the resistor and the capacitor in series with the square wave generator.  Some capacitors are polar and must be connected in a particular direction.  Check to see if there are $+$ and $-$ signs on the capacitor; if so, ask your instructor about this.
\item Using the Pasco 750 interface, measure both the voltage across the capacitor ($V_C$) and the voltage across the resistor ($V_R$).  Set the sample rate to the equivalent of 250 samples per period ($\txtfrac{250}{[10 RC]}\unitfrac{samples}{sec}$).  Let it run for 2 full periods, so that you can select the best starting point.  In ``Data Studio,'' create and print a plot of the square wave versus time, $V_C$ versus time, and $V_R$ versus time.
\item For either charging or discharging, transfer the data sets $V_C$ versus time and $V_R$ versus time to Excel.  The time values should correspond in these two data sets.
\end{lablist}

\subsection{Analysis}

\begin{lablist}
\item Using Data Studio, empirically measure the time constant for charging and for discharging so that you can use this number in the analysis:
    \begin{description}
    \item[Charging:] When a capacitor is charging, the time constant corresponds to the amount of time that it takes to increase the voltage to 63.2\% of the $V_{\rm max}$.  (This is $\left[1-\frac{1}{e}\right] V_{\rm max}$.)  ``Data Studio'' has a graph of voltage versus time; find the data point that corresponds to this voltage and time value.
    \item[Discharging:] When a capacitor is discharging, the time constant corresponds to the amount of time that it takes to decrease the voltage to 36.8\% of the $V_{\rm max}$.  (This is $\frac{1}{e} V_{\rm max}$.)  ``Data Studio'' has a graph of voltage versus time; find the data point that corresponds to this voltage and time value.
    \end{description}
\item Compare (\%-difference) your empirically determined time constant for charging to your empirically determined time constant for discharging.
    \begin{itemize}
    \item Are they consistent with each other?  Which one is larger?
    \item The function generator has an internal resistance.  This resistance is also in series with the circuit when the capacitor is charging.  In each case (charging and discharging), solve $\tau = RC$ for R to determine the resistance of that circuit.  Which is larger $R_{\rm charging}$ or $R_{\rm discharging}$?  Why?
    \item It is possible to deduce the resistance of the function generator from the information in the previous question.
    \end{itemize}
\end{lablist}

\subsubsection{``Charging'' Analysis}

\begin{lablist}
\item The resistor and capacitor are in series, so the voltage across each should always add to the voltage of the function generator.  With the Excel data, calculate the average and standard deviation of $V_R+V_C$ for the charging time interval and compare this to $V_{\rm max}$.
\item Capacitors build charge exponentially over time.  Linearize your data by modifying the equation below with the natural log function {\it as we have done in previous labs}.
    \[ V(t) =  V_0 \left[1 - e^{-t/\tau} \right] \]
    Deduce what should be plotted on the horizontal and vertical axes as well as what the slope and intercept should be.  Plot this and do a regression.
    \begin{itemize}
    \item Does this time constant match the value for charging that you empirically determined in Data Studio?  Should it?  (If so, calculate the \%-difference.)
    \item Does this time constant match the $\tau = RC$ value that you found from the multimeter measurement of R and C? Should it?  (If so, calculate the \%-difference.)
    \end{itemize}
\item Calculate the true capacitance of your capacitor using the time constant determined from your regression analysis along with the correct value of the resistance, including both the ohmmeter measurement of your resistor and the internal resistance of the function generator.
\end{lablist}

\subsubsection{``Discharging'' Analysis}

\begin{lablist}
\item The resistor and capacitor are in series, so the voltage across each should always add to the voltage of the function generator.  With the Excel data, calculate the average and standard deviation of $V_R+V_C$ for the discharging time interval and compare this to zero.
\item Capacitors discharge exponentially over time.  Linearize your data by modifying the equation below with the natural log function {\it as we have done in previous labs}.
    \[ V(t) =  V_0 \, e^{-t/\tau} \]
    Deduce what should be plotted on the horizontal and vertical axes as well as what the slope and intercept should be.  Plot this and do a regression.
    \begin{itemize}
    \item Does this time constant match the value for discharging that you empirically determined in Data Studio?  Should it?  (If so, calculate the \%-difference.)
    \item Does this time constant match the $\tau = RC$ value that you found from the multimeter measurement of R and C? Should it?  (If so, calculate the \%-difference.)
    \end{itemize}
\item Calculate the true capacitance of your capacitor, again, using the time constant determined from your regression analysis along with the correct value of the resistance, including only the ohmmeter measurement of your resistor.
\end{lablist}



\onecolumn%========================================

\section{Magnetic Field Demonstrations - An Exercise}
\revised{Apr 6, 2009}
\label{l:Magneticdemo}
\setcounter{dave}{0}

The lab room has been set up with five lab stations; each involves a short experiment designed to demonstrate the characteristics of the magnetic field for various objects.  Your group will move from table to table and carry out each of these experiments, making observations and collecting data as necessary.  Listed below are questions which you should attempt to answer at each station.  Instead of a lab report, you will need to write a half-page summary on each station.  It should be thorough enough to explain the phenomenon to a colleague who missed today's lab.

You should form into four groups so that there is always an empty station.  Take notes in your lab notebook, but then write up and turn in a formal document that discusses the relevant ideas for each station.
%\noindent
\\
\rule{4in}{1pt}

\begin{center}
{\bf CAUTION:} In all experiments, note that electronic equipment such as power supplies and computers produce magnetic fields that might impact your experiment.  You should not measure your magnetic fields near other electronic devices.  You should also be aware that strong magnetic fields can erase credit cards and other magnetic storage devices.
\end{center}
\rule{4in}{1pt}

\subsection{Data Studio's Magnetic Field Sensor}
\label{ss:fieldsensor}

To use the magnetic field sensor, turn on Data Studio, plug in the sensor, and activate the sensor in the usual manner.  At the tip of the arm on the sensor are two white dots, one on the end of the tip and one on the side of the tip.  Notice also the control buttons on the handle of the sensor.  The first button has two settings: marked $^\rightarrow_\rightarrow$ and $\uparrow\uparrow$ and these are labeled [Radial/Axial].  The Radial setting measures magnetic field pointing into the side of the arm of the sensor, at the location of the white dot.  The Axial setting measures magnetic field pointing into the end of the sensor, at the location of the white dot.  Negative means that the magnetic field is out of the sensor instead of into the sensor.

When changing between Axial and Radial, the sensor must be tared.  If you tare it while it is pointing in the direction of some magnetic field, then it cannot\footnote{This is exactly analogous to tarring a scale with a cup on it; the scale only reads the mass of the material added to the cup after the tarring, not the mass of the cup itself.} measure that magnetic field.

\subsubsection{Measuring the $B$ of a Permanent Magnet}

Determine the expected direction of the magnetic field.  Set the sensor to either Radial or Axial, according to how you expect to hold it, and tare it.  Verify that the scale is set to [$1\times$] and plot in Gauss or Tesla, as appropriate.  Place the tip of the sensor such that the white dot on the side (or top as necessary) is near the magnet pointing in the direction of the expected magnetic field.

\subsubsection{Measuring the $B$ of a Wire Loop}

Use the right-hand rule to determine the expected direction of the magnetic field.  Set the sensor to Axial and tare it.  Verify that the scale is set to [$1\times$] and plot in Gauss or Tesla, as appropriate.  Place the tip of the sensor such that the white dot on the end is as close to the center of the wire loop as possible and points as perpendicular as possible to the plane of the coil.

If there is a cross-field (a field perpendicular to the field of the coil), then you might also try to measure that by removing the sensor from the vicinity of the coil, setting it to Radial, re-taring it, and then replacing the sensor into the center of the coil with the white dot on the side pointing in the direction of the cross field.

\subsubsection{Measuring the $B$ of a Solenoid}

Use the right-hand rule to determine the expected direction of the magnetic field.  Set the sensor to Axial and tare it.  Verify that the scale is set to [$1\times$] and plot in Gauss or Tesla, as appropriate.  Place the tip of the sensor such that the white dot on the end is as close to the center of the solenoid as possible and points as close as possible along the axis of the solenoid.

You might also try to measure the radial component to the field by removing the sensor from the vicinity of the coil, setting it to Radial, re-taring it, and then replacing the sensor into the center of the solenoid with the white dot on the side pointing outwards towards the coil.

\subsubsection{Measuring the $B$ of a Long Straight Wire}

Use the right-hand rule to determine the expected direction of the magnetic field.  Set the sensor to Radial and tare it.  Verify that the scale is set to [$1\times$] and plot in Gauss or Tesla, as appropriate.  Place the tip of the sensor such that the white dot on the side is in the vicinity of the wire pointing in the direction of the expected magnetic field.

If there is a cross-field (a field perpendicular to the field of the wire), then you might also try to measure that by removing the sensor from the vicinity of the wire, setting it to Radial, re-taring it, and then replacing the sensor into the center of the coil with the white dot on the side pointing in the direction of the cross field.
%\noindent
\\
\rule{4in}{1pt}




\subsection{Magnetic Field of a Permanent Magnet - Iron Filings}

You should have two cylindrical magnets, a coffee filter, a stiff plastic plate, some paper clips and nails, and a salt shaker of iron filings available at this station.  Each magnet is labeled with either ``N'' or ``S'' to indicate the north or south pole of the magnet, respectively.

\subsubsection{Magnets and Paper Clips}

Consider how magnets and paper clips interact.  In each of the following, you may replace the paper clips with nails if you like.
\begin{question}
\item Describe any interaction you experience when you touch any pair of the paper clips together in the various possible orientations.
\item Describe any interaction you experience when you touch the magnets together in the various possible orientations.
    \begin{enumerate}
    \item Describe how you can pick one magnet up with the other.
    \item Describe how you can knock one magnet over with the other (without making contact).
    \end{enumerate}
\item Describe any interaction you experience when you touch one of the magnets together with any one of the paper clips in the various possible orientations.
    \begin{enumerate}
    \item Pick up one paper clip {\it with the narrow end} of one of the magnets so that the clip hangs straight out from the magnet.  Then, while these are still in contact, touch a second paper clip with the end of the first clip.  Now, while carefully holding the first clip near the magnet, carefully remove the magnet from the first clip.  {\it After showing the result to your instructor}, describe and explain the result on the two paper clips.
    \item Predict what will happen if you brought either end of the magnet slowly towards the lower end of the dangling paper clip.  Verify your predictions and provide an explanation for how and why your predictions were correct or incorrect.  (Be sure to include comments about the orientation of the poles of the magnets as well as the poles of the paper clips.)
    \end{enumerate}
\item Explain why it is possible to stick either end of a paper clip to either end of a magnet, but you cannot stick one end of a magnet to either end of another magnet.
%\item Explain how it can be that in some cases (you should note which) the orientation matters and in other cases (you should note which) the orientation does not matter.  Part of this answer should discuss whether or not all objects have a north and south pole.
\end{question}

\subsubsection{Magnets and Iron Filings}

Iron filings act similarly to the paper clips.  The iron filings are small enough that they allow us to ``map'' the magnetic field of an object.  Each individual filing will align itself with the magnetic field.

Place the coffee filter on top of the clear, plastic plate.  Sprinkle some iron filings on the coffee filter.  Note:  We are using a coffee filter because its shape allows us to minimize spilling the filings on the table, or worse, the floor.  {\bf Be \underline{very} careful not to spill the filings.}  Have one person hold the plate up with both hands high enough for another person to move one of the cylindrical magnets beneath it while you all watch what happens to the iron filings.  Meanwhile, somebody else should place one of the cylindrical magnets underneath the sheet of paper so that everybody can observe the patterned displayed by the filings.
\begin{question}
\item Determine if you can make any interesting patterns in the filings by placing or moving the magnet under the filings.  When you think you have a good sense of the patterns, try to draw or describe some of what you found.
\item Without the magnet, try to smooth out the filings.  (If you can't smooth them out well, then carefully hold the salt-shaker over the coffee filter and unscrew the cap, pour the filings back in, replace the cap, and re-sprinkle more filings onto the filter.)  Predict what you should see if you had two magnets under the filings each with the North pole facing the other magnet.  What if both South poles were facing each other?  What if one North and one South pole faced the other?
\item While one person holds the plate with the filter and filings, have two people each hold a magnet under the filings as described in the previous question.  Verify your predictions and provide an explanation for how and why your predictions were correct or incorrect.
\end{question}
%\noindent
\rule{4in}{1pt}


\subsection{Magnetic Field of a Permanent Magnet - Field Sensor}

You should have a small, four-post, wooden stand with a hole in the top and a wooden cylinder taped to a ruler on a lab-jack.  The wooden equipment was hand-made and is delicate; {\bf please be careful when handling the apparatus}.

\begin{question}
\item Use the field sensor (as described in Sec.~\ref{ss:fieldsensor}) set to Axial to measure (and record) the strength of the magnetic field {\it due to the permanent magnet} at various positions from from the magnet.  It will be useful to set the sensor $10.0\unit{cm}$ from the magnet and make a measurement at each $1.0\unit{cm}$ interval as you move towards the magnet.
\item Repeat this for the Radial setting.
\item Tabulate your results (you may also graph it {\it if you like}) and discuss the manner in which the field depends on distance.  Is it linear?  Polynomial?  something else?
\end{question}
%\noindent
\rule{4in}{1pt}




\subsection{Magnetic Field of the Earth - Tangent Galvanometer}

This station has a loop of wire connected to a power supply.  There is also a compass needle balancing on a needle point located at the center of the loop.  Verify that the power supply is connected to the wire.  Without turning on the power supply, predict which way current will flow based on the connection.  With the power supply off (not merely turned down), carefully and gently align the loop so that it rests along the line that the compass wants to rest.  {\bf Before you turn on the power supply}, turn the voltage dial down to zero and the current dial up to some medium value.  The HI-LO setting on the current dial should be set to LO.
When current begins to flow through the coil, a magnetic field is generated whose direction can be found from the right-hand rule and whose magnitude can be calculated from
%
\begin{equation}
\label{eq:Bloop}
B_c = \frac{\mu_0 I N}{2 R}
\end{equation}
%
where $B_c$ is the coil's magnetic field strength, $I$ is the
current (in amps), $\mu_0 = 4\pi\ten{-7}\unitfrac{N}{A^2}$, $N$ is the
number of turns of the wire in the coil, and $R$ is the radius of the coil (which you will have to measure).

\begin{question}
\item Using components, resultants, magnitudes and directions, devise a technique to combine the controllable magnetic field due to the coil with the uncontrollable, but fixed, magnetic field of the earth in such a way as to calculate the magnitude of the Earth's magnetic field.  Several hints are given below.
    \begin{enumerate}
    \item If the field due to the loop were the only field present, which way would the compass needle point after you turned on the current?  What would happen if you changed the direction of the current?
    \item If there were two magnetic fields present, along which (if either) would a compass point?
    \item Can you control the direction of any magnetic field present?  Why might you want the wire to line up with the direction that the compass points before turning on the current?  How are the directions of the components of a vector ``aligned''?
    \item Can you control (and know the value of) the magnitude of any magnetic field present?  What would the compass needle do as you slowly turn this up from zero to some very large value?
    \item Can you relate components to resultants and vice versa?
    \end{enumerate}
\item Turn on the power and very slowly turn up the voltage while watching the ammeter; the ammeter should not go above $4$ or $5\unit{mA}$.  Describe the effect this has on the compass needle.  Predict and explain what \underline{would} happen if you swap the direction, but not the magnitude of the current.  {\bf Do not actually swap the direction of the current at this point.}
\item Let the compass needle settle into whatever orientation it likes while the power continues to be supplied to the circuit.  While you wait, record the value of the current and use the field sensor (as described in Sec.~\ref{ss:fieldsensor}) to measure (and record) the strength of the magnetic field {\it due to the loop} at the center of the compass.  After the compass needle settles in, place a protractor on the top of the wire loop and measure (and record) the angle of the deflection.  Without turning off the power supply, unplug the cables that provide current to the circuit {\it from the power supply, not from the wire loop, which should not be bumped}.  Plug the cables in the other way to flip the current. Describe what happens to the compass.  Remeasure (and record) the current, the magnetic field, and the angle of deflection for this direction.  Calculate the magnetic field of the Earth based on the average measured B and the average deflected angle.  Despite the large uncertainty, compare it to the known value.
\end{question}
%\noindent
\rule{4in}{1pt}







\subsection{Magnetic Field of a Solenoid}

Before you turn on the power supply, set the current all the way down and the voltage up at some intermediate point.  The HI-LO setting on the current dial should be set to LO.

\subsubsection{The Small Solenoid}

\begin{question}
\item Without turning on the power supply, connect the positive and negative terminals of the power supply to the terminals of the smaller solenoid.  Note the direction of current flow and predict the direction of the magnetic field.
\item\label{q:Solenoid} Use the field sensor (as described in Sec.~\ref{ss:fieldsensor}) set to Axial to measure (and record) the strength of the magnetic field {\it due to the solenoid}.  Insert the sensor as far along the axis of the solenoid as possible and try to keep your hand as steady as possible. Turn on the power supply and slowly turn up the current to some intermediate value.  Move the sensor around a little inside the solenoid and discuss the results.
\item Repeat the previous step with the sensor set to Radial.  What variables does the magnetic field due to a solenoid depend on?  Does it follow the predicted equation for a solenoid?
\end{question}
%
Turn the power supply off.  Disconnect the wire from the solenoid.

\subsubsection{Measurements with the Large Solenoid}

Before connecting the large solenoid, measure the number of turns per length, $\displaystyle n = \frac{N}{L}$.  Do not measure the full number of loops nor the full length.  Find some portion of the coil and measure the number of coils and the length in that small sample size.  When you calculate $n$, notice that the solenoid is wrapped four coils deep, so multiply your $N$ by $4$.
In this problem, you can measure $B$ and $I$ in order to find $\mu_0$ to verify the equation of a solenoid.

\begin{question}
\item Connect the power supply in series with an ammeter and then the solenoid before completing the circuit back to the power supply.  Be sure that the ammeter reads several amps.  Insert the field sensor set to the appropriate settings as close as possible to the axis of the solenoid.  Predict what you will measure for the Radial and Axial measurements above.
\item Measure the current from the ammeter, the magnetic field from the sensor and calculate a value for $\mu_0$ based on the equation for a solenoid.   Compare this to the accepted value in your book.  Draw a conclusion about the equation of a solenoid.
\end{question}
%\noindent
\rule{4in}{1pt}

\subsection{Magnetic Field of a Wire}

Before you turn on the power supply, set the current all the way down and the voltage up at some intermediate point.  The HI-LO setting on the current dial should be set to HI.  (You will be using a current of about $8\unit{A}$.)

Notice that the wire connects are exposed.  Be careful not to touch the connections when the current is on.

\begin{question}
\item Without turning on the power supply, connect the positive and negative terminals of the power supply to opposite ends of one of the three wires inside the {\sc Romex}$^{\rm tm}$ cable.  Be sure, based on color, to connect them to {\it the same} wire.  Straighten a long section of the wire as well as possible.  Note the direction of current flow and predict the direction of the magnetic field.
\item\label{q:BoverRa} Use the field sensor (as described in Sec.~\ref{ss:fieldsensor}) set to Radial to measure (and record) the strength of the magnetic field {\it due to the long straight wire}.  The sensor should lie along the direction of the wire and must be touching the wire to make any measurement.  Turn on the power supply and turn up the current to about $8\unit{A}$.  (You may change the current in order to see the effect of current on the size of the magnetic field.) Move the sensor to different radii and discuss the results.  Place the sensor close to the wire and slowly vary the strength of the current using the knob on the power supply.  Discuss the results.
\item\label{q:BoverRb}  Repeat the previous step with the sensor set to Axial.
\item Turn the power supply off.  Disconnect the wire from the power supply.  Connect the red terminal to one side of the white wire and the black terminal to the black wire so that both connections are at the same end.  On the far end, connect the black and white wires (this is where the lamp goes and you should not actually do this at home where there is {\it significantly} more current able to flow!  {\it Really}!).  Predict what you will measure for the Radial and Axial measurements above.
\item Repeat Questions~\ref{q:BoverRa} and~\ref{q:BoverRb} for the two wire case.  Verify your predictions and provide an explanation for how and why your predictions were correct or incorrect.
\end{question}






%\onecolumn%========================================

%\revised{Apr, 2009}
\twocolumn[\section{Reflection and Refraction at a Plane Surface}
\revised{Spring, 2012}
\revision{2012}{clarified, added some PASCO, revised the pins.}
\revision{Apr, 2009}{}
\label{s:refraction}
\setcounter{dave}{0}

\subsection{Introduction}

	Sir Isaac Newton developed a particle theory of light (essentially photons) in order to use geometry to explain two commonly observed optical properties of light: reflection and refraction.  Those interested in optics in Newton's time had observed that whenever light is incident upon any surface, some of the light is reflected off from the surface and some of the light is transmitted through the surface.  When the incident light approaches along a line normal to the surface, it continues along its straight-line path.  However, when the incident light approaches at an angle as in Fig.~\ref{f:refraction}, then the light changes direction.  The transmitted light is said to refract.  The property that determines the amount of the refraction is called the {\it index of refraction} and is denoted by $n$.  By convention, the angles for the incident, the reflected, and the refracted light are all measured from the line normal to the surface. \vspace{3pt}]
\revised{Apr, 2009}

Since the reflected light never sees the material with a different index of refraction, the reflection follows the law of reflection, which states that the angle of the incident ray is equal to the angle of the reflected ray:
\vspace{-3pt}
%
\[ \theta_{\rm incident} = \theta_{\rm reflected}. \vspace{-3pt}
\]
%
The transmitted light, on the other hand, enters the material, which has a different index of refraction and follows the law of refraction or Snell's Law, named for Willebrord Snell (1591-1626).  This law is expressed in terms of the sine of the incident and refracted angles, $\theta_i$ and $\theta_r$:
%
\vspace{-3pt}
\[ n_i \sin\theta_i = n_r \sin\theta_r \vspace{-3pt}
\]
%
where $n_i$ is the index of refraction for the incident medium, and $n_r$ is the index of refraction for the refracted material.
%
\begin{figure}[t]
\begin{center}\setlength{\unitlength}{1cm}
\begin{picture}(5,6)
% Glass
\put(0,0){\line(0,1){3}}
\put(5,3){\line(-1,0){5}}
\put(5,3){\line(0,-1){3}}
% light rays
\put(2,3){\line(-2,3){2}}
\put(2,3){\color{blue}{\line(1,-4){.5}}}
\put(2,3){\color{green}{\line(2,3){2}}}
% normal lines
\multiput(2,1)(0,.6){8}{\line(0,1){.3}}
% label angles
\put(1.9,0.8){\small$\theta_{\rm refract}$}
\put(1.2,4.3){\small$\theta_i$}
\put(2.2,4.3){\small$\theta_{\rm reflect}$}
% label indexes
\put(4,1){$n_r$}
\put(4,3.5){$n_i$}
\end{picture}
\caption{Some of the incident light from the upper left reflects to the upper right and the rest refracts into the material to the lower right.}
  %The refracted light is no longer traveling along the same line of sight as it was originally.}
\label{f:refraction}
\end{center}
\end{figure}
%

%\vspace{11pt}
\noindent
\begin{minipage}{\textwidth}
\vspace{10pt}
\ \ \ The index of refraction determines the amount of refraction because it is a measure of the speed that light travels through the medium: $n = c/v$, where $c=2.998\ten{8}\unitfrac{m}{s}$ is the speed of light in a vacuum and $v$ is the speed light travels through the medium.  The index of refraction for a vacuum is therefore identically 1.  It happens that the index of refraction for air is close enough to 1 that we will not be able to measure a difference at our level of precision.

\vspace{-10pt}
\subsection{Experimental Objectives}

\vspace{-3pt}
\noindent
\begin{lablist}
\item Experimentally verify the law of reflection from a plane surface by tracing the paths of the incident and the reflected light rays.
\item Experimentally verify the law of refraction from a plane surface by tracing the paths of the incident and the refracted light rays to calculate $n$ for the glass,
\item Experimentally verify the law of refraction from a plane surface by predicting the path a ray of light will emerge from a prism for a given incident angle, and
\item Experimentally determine the critical angle for a piece of glass.
\end{lablist}

\end{minipage}

\newpage
\ \vfill \


\twocolumn[\subsection{Pre-Lab Work}

\begin{question}
\item Since light travels faster in a vacuum than in any other material, $c>v$, determine if $n$ has a maximum or a minimum value and what that might be.
\item Consider Snell's Law.
    \begin{enumerate}
    \item If it were possible to create two different materials with the same index of refraction, so that $n_i = n_r$, then comment on the relationship between the incident and refracted angles.
    \item When light passes from a material with a small index of refraction to a material with a large index of refraction so that $n_i<n_r$, will $\theta_i$ larger than or smaller than $\theta_r$?  Draw an approximate diagram indicating if the light bends {\it towards} the normal, as in Fig.~\ref{f:refraction} above, or if it bends {\it away from} the normal.
    \end{enumerate}
\item Critical Angle:  It is possible, for the case where light is bent away from the normal (becomes more parallel to the surface), that it can be bent all the way over to actually being parallel to the surface.  When this happens, the light is not transmitting into the new material and cannot be seen from the other side.
    \begin{enumerate}
    \item It is only possible to have a critical angle for either $n_i > n_r$ or $n_i < n_r$.  For which of these cases is it possible?
    \item Using Snell's Law above, derive an equation for the incident critical angle (the refracted angle is $90^\circ$), in terms of the two indices of refraction for the two media.
    \end{enumerate}
\end{question}


\subsection{Procedure} ]

\subsubsection{Refraction at a plane surface}

Place the square piece of glass on the cardboard that has been covered with a piece of paper.
Orient the frosted sides of the glass towards the short sides of the paper.
Draw the outline of the glass on the paper and be careful to not move the glass.
(You may want to use a pair of pins at the corners to fix the glass in place.)

Plug in and set up the PASCO light box to emit a single ray of light.  Place the light box so that its single ray of light shines through the glass.  Notice that the path the light follows through the glass is like the solid line in Fig.~\ref{f:refract-square} following points 1, 2, 3, and 4.
Notice that the path the light follows above the glass is like the dashed line in Fig.~\ref{f:refract-square} following points 1, 2, $3'$, and $4'$.

To mark the incident path of the light, stick a pin in the paper at location 1, closest to the light box.
You should see the shadow of the pin in the light beam.  Place {\sc pin 2} in the shadow of {\sc pin 1}.
You should also see the shadow of the pin in the exiting light beam.  Place {\sc pin 3} and {\sc pin 4} so that each lines up with shadows of the previous pins.

%
\begin{figure}[p]
\begin{center}\setlength{\unitlength}{.6cm}
\begin{picture}(9,10)
% Glass
\put(3,3){\line(1,0){5}}
\put(3,3){\line(0,1){5}}
\put(8,8){\line(-1,0){5}}
\put(8,8){\line(0,-1){5}}
% light rays
\put(5,8){\line(-2,3){2}}
\put(5,8){\line(1,-5){1}}
\put(6,3){\line(2,-3){2}}
% normal lines
\multiput(5,6)(0,.6){8}{\line(0,1){.3}}
\multiput(6,1)(0,.6){8}{\line(0,1){.3}}
% Pins
\put(4.5,8.75){\circle*{.2}}
\put(3.5,10.25){\circle*{.2}}
\put(6.5,2.25){\circle*{.2}}
\put(7.5,0.75){\circle*{.2}}
% Apparent pins
\multiput(5.5,3.75)(-1.125,1.6875){5}{\line(-2,3){.5}}
\put(2.5,8.25){\circle{.2}}
\put(1.5,9.75){\circle{.2}}
% label angles
\put(6.4,1){\small$\theta_a$}
\put(5.6,4.8){\small$\theta_g$}
\put(4.9,5.8){\small$\theta_g$}
\put(4.2,9.3){\small$\theta_a$}
% label indexes
\put(7,6){$n_g$}
\put(7,8.5){$n_a$}
\put(3.5,2.2){$n_a$}
% label pins
\put(4.4,8.2){\small $3$}
\put(3.3,9.7){\small $4$}
\put(6.7,2.2){\small $2$}
\put(7.7,0.7){\small $1$}
\put(2.2,7.6){\small $3'$}
\put(1.1,9.1){\small $4'$}
\end{picture}
\caption{Square Glass Ray Diagram.  The light travels from pin 1 past pin 2 through the glass to pins 3 and 4.  At the first boundary, air is the incident material and glass is the refracting material.  On the second, glass is the incident material and air is the refracting material.}
\label{f:refract-square}
\end{center}
\vspace{-30pt}
\end{figure}
%

Once you and all of your partners are satisfied with the location of the pins, verify that the glass has its outline drawn on the paper and remove the glass from the paper.  You may also turn off the light box.  Use a straight-edge to draw a straight line through each pair of pins up to the edge of the lens.  Use a straight-edge to draw another line inside the glass connecting where the pin-lines meet the surface.  Use the protractor to measure all incident and refracted angles.  Look up the index of refraction of air and of glass and notice that they have a range of values.  Select a reasonable value for $n_{\rm air}$ and use Snell's law to determine the index of refraction for the glass.  This is your first verification of Snell's Law.

Notice that the angle decreases ($\theta_a>\theta_g$) when the light
passes from small $n$ (air) to large $n$ (glass).  This light is bent towards the normal.
Similarly, when the light passes from large $n$ (glass) to small $n$ (air), the
angle increases ($\theta_g<\theta_a$).  This light is bent away from the normal.


\vfill

\subsubsection{Reflection at a single plane surface}

Carefully replace the square piece of glass onto your diagram and turn the light box back on.  Carefully re-align the light box to shine along the path that it previously followed.  Now you should also notice that a portion of the beam reflects off of each interface.  For the first boundary (after {\sc pin 3}), place two pins along the beam that is reflected off of the glass.  When you are satisfied that it is aligned as well as possible, turn off the light box and remove the glass.  Measure the reflected angle and verify the law of reflection.

\vfill

\subsubsection{Predicting the Exiting Ray}

Use the thin triangular prism in this part.  Place the first two pins as indicated in
Fig.~\ref{f:refract-triangle}, which starts the drawing that you will need to complete.
I have drawn the incident ray, the dashed normal line, and the refracted ray.
You will need to figure out where the light hits the second boundary, draw a line normal to the second surface, and then use Snell's law to determine the direction that the light beam exits from the prism.
\vfill
\newpage
\begin{description}
\item [Step 1)] Decide on a value for the incident angle.  Assuming that the index of refraction of this glass is the same as what you calculated for the square glass, use Snell's law to determine the refracted angle.
\item [Step 2)] Notice that the light ray inside the prism forms a triangle with the apex of the prism.  Since the angles of any triangle add to $180^\circ$, you can use the first refracted angle to determine the second incident angle.  (Recall that a normal line makes a $90^\circ$ with the surface.)
\item [Step 3)] Use Snell's Law to determine the exit angle.
\end{description}

Show your prediction to the instructor.  Set up the triangular prism and the light box to verify your prediction.  Your actual result will probably be noticeably different than your prediction.  When you make your measurements your angles
will not exactly match those you used in your prediction and the glass may have a different index of refraction.  Use your measurements to determine the index of refraction for this piece of glass and verify that it is a reasonable value as your second verification of Snell's Law.

\vfill
%
\begin{figure}[h]
\begin{center}\setlength{\unitlength}{.7cm}
\begin{picture}(10,8)(0,0)
% Glass
\put(3,0){\line(0,1){5}}
\put(8,2.5){\line(-2, 1){5}}
\put(8,2.5){\line(-2,-1){5}}
% light rays
\put(6,3.5){\line(-2,3){3}}
\put(6,3.5){\line(1,-5){.25}}
% normal lines
\multiput(5,1.5)(.75,1.5){5}{\line(1,2){.4}}
% Pins
\put(5.5,4.25){\circle*{.2}}
\put(4.5,5.75){\circle*{.2}}
% label angles
\put(5.9,4.65){\small$\theta_a$}
\put(5.6,2.3){\small$\theta_g$}
% label indexes
\put(3.4,4){$n_g$}
\put(7,3.5){$n_a$}
\put(7,1){$n_a$}
% label pins
\put(5,4.2){\small $1$}
\put(4,5.7){\small $2$}
\end{picture}
\caption{Triangular Glass Ray Diagram}
\label{f:refract-triangle}
\end{center}
\vspace{-10pt}
\end{figure}
%
\vfill

\onecolumn \noindent

\subsubsection{The Critical Angle}

Again consider the prism.  Snell's Law makes an interesting prediction for large angle incident rays.  This is the effect that allows optical fibers to carry signals without loss (unlike electrical cables which have resistance that degrades a signal).  It is possible for light in a material with a high index of refraction that is incident upon a material with a low index of refraction to have total internal reflection -- In our case, this means that the light does not refract out of the glass!  To do this calculation you can repeat the steps of your previous calculation in the reverse order.
\begin{description}
\item [Step 3)] Set your exit angle (the second refraction angle) to be $90^\circ$.  Use Snell's Law to determine the second incident angle (inside the glass).  This is the {\it critical angle}.  Now we need to figure out how to set the equipment to produce this angle.
\item [Step 2)] Notice that the light ray inside the prism will again form a triangle with the apex of the prism.  Since the angles of any triangle add to $180^\circ$, you can use the second incident angle to determine the first refracted angle.  (Recall that a normal line makes a $90^\circ$ with the surface.)
\item [Step 1)] Since you know the index of refraction of this glass, use Snell's Law and the first refracted angle to determine the first incident angle.
\end{description}


Set the light box off of the cardboard and paper so that you can rotate the cardboard to easily change the incident angle.  Set the incident ray to be incident on one side at about $\!\!\txtfrac{1}{4}$ of the way from the apex.  If you set it right, then when you turn on the light you should see the light refracted off of the first surface, but then not exit the glass at the second surface.  This may not actually be the case.  Either way, slowly rotate the paper (and the prism) until you can see an exiting beam and then slowly rotate it again until the refracted ray is refracted by {\it just} $90^\circ$, the smallest angle that makes the light not exit the prism.   The critical angle is now the incident angle at the second surface of the prism.  Experimentally determine the critical angle for this piece of glass.  Compare the predicted value of the critical angle to the measured value of the critical angle as your third and final verification of Snell's Law.


\subsection{Questions}

\begin{question}
\item Explain why a plane mirror reverse left and right.   It will help to draw a ray diagram that replaces a the pins with a wider object that has a clear left side and right side.
\item What happens to the speed that light travels through a medium a greater index of refraction?
%\item Under what conditions is the angle of refraction greater than the angle of incidence?
%\item What can be concluded about the refracted ray and also the reflected ray when the angle of incidence is greater than the critical angle?
\end{question}




\onecolumn%========================================

\section{Simple Lenses}
\revised{Aug, 2011}
\label{s:SimpleLens}
\setcounter{dave}{0}

Anybody who has looked through glasses, microscopes, telescopes,
a magnifying glass, or even a window has experienced a lens.  We see
images through lenses.

\subsection{Objectives}
Through the various arrangements suggested below, you should be able to\ldots
\begin{lablist}
\item \ldots figure out the definition of the following terms based on the images you observe
\begin{itemize}
\item magnified versus minified
\item upright versus inverted
\item real versus virtual
\end{itemize}
\item \ldots figure out the relationships between the measurable quantities:
\begin{itemize}
\item the magnification, defined in terms of the image height and the object height: $\displaystyle M = \frac{h_i}{h_o}$
\item the image distance (from the lens) $q$
\item the object distance (from the lens) $p$
\item the focal length, $f$.
\end{itemize}
\end{lablist}

\subsection{Procedure with Questions for the Analysis}

\begin{center}
\fbox{\begin{minipage}{5in}
The data you take in Section~\ref{i:measure} will also be used in Section~\ref{i:focal}, so tabulate it in a coherent and clear format.
\end{minipage}}
\end{center}

\subsubsection{Defining Your Terms}

As a group, take one of the three lenses and your white screen into a dark room so that you can see a brighter room through a doorway.  Have one person in the group go stand in the bright room and move around while another person looks through the lens.  Then trade places until everybody has a chance to look through the lens.
\begin{question}
\item As a group, decide how to describe the image in the terms above:  Magnified or minified?  upright or inverted?  real or virtual?
\item While still in the darker room, notice that if you hold up your lens in front of a white screen (or a tee shirt) so that the lens is between the screen and the person in the other room, the image of the brighter room appears on the screen. (Have somebody dance and jump in the bright room while you watch the image on the screen.)
\end{question}
Take that same lens back into the room and have each person view this text through the lens.
\begin{question}
\item As a group, decide how to describe the image in the terms above:  Magnified or minified?  upright or inverted?  real or virtual?
\item Notice this time that the text appears to be on the same side of the lens as the actual text is.  There is no place that you can put your white screen to have the words from the page appear on the screen.
\end{question}
Compare and contrast the image of the room to the image of the text.
\begin{question}
\item If you had to choose between the names real and virtual, which image would you call real?  which would you call virtual? why?
\item Decide if either of the images change based on the location of the lens relative to the object being viewed.
\end{question}


\subsubsection{Quantify the Magnification}
\label{i:measure}
In order to make precise measurements, attach the screen to one end of the optical bench and the light source to the other side.  The illuminated cross-hairs on the light source will be the object observed.
Place a converging lens\footnote{Converging lenses, also called convex lenses, bulge in the middle.  Diverging lenses, also called concave lenses, bulge at the edges.} on the optical bench between the
object and the screen.  Keep the screen and the light source fairly far apart (the convenient distance will depend on which lens you are using). Adjust the position of the lens until a clear image is formed on the screen.
%
\begin{question}
\item Describe the image formed in the terms defined above.
\end{question}
%
Measure the distances between the lens and the image (image distance, $q$), and between the lens and the object (object distance, $p$).
Measure the height of the image on the screen $h_i$ and the height of the object on the light source $h_o$.
(If the image is inverted, then $h_i$ is a negative value.)
Calculate the magnification factor of the image, $\displaystyle M = \frac{h_i}{h_o}$.  \\

\noindent
For this same lens, without changing the position of the screen or the light source, find a second image that has a different description using the defined terms above.  Again, measure $p$, $q$, $h_i$, and $h_o$ and calculate $M$.
%
\begin{question}
\item Describe this second image in the terms defined above.
\end{question}

\noindent
Repeat this for each of the other two lenses.
\begin{question}
\item Determine the relationship between $M$, $q$, and/or $p$.  Verify that your relationship works (to about two significant figures) for all six data sets separately.
\end{question}

% View some text through a lens.  Describe the image.
% Measure the object distance and compare it to the focal length.
% Determine the magnification by replacing the text with a ruler.
% Compare the image of the hash marks on the ruler with the actual object hash marks on the ruler.
% Use this comparison to directly measure the magnification.
% Compare magnification/minification, real/virtual, upright/inversion between this situation and that of the far away object.


\subsubsection{Quantify the Focal Length}
\label{i:focal}
The lens equation,
$\displaystyle \frac{1}{p} + \frac{1}{q} = \frac{1}{f}$,
shows the relationship
between the image distance, $q$, object distance, $p$, and the focal length of the
lens, $f$.   Using your data from part~\ref{i:measure}, determine
the focal length of each lens.  Each lens should have two values for $f$ (one for each image).
\begin{question}
\item For each lens separately, compare the average of these numbers to the accepted value.
\end{question}
%
The focal length of a lens may also be determined by forming the
image of a very distant object on a screen.  In this case, the object
distance, $p$, becomes very large and therefore $1/p$ becomes very small.  When
this is the case, the lens equation may be written as $1/q = 1/f$, or more conveniently, $q=f$.
Using a distant object in an adjacent room, measure the image distance and thereby determine the focal length for each lens.
\begin{question}
\item Compare this with your results for $f$ from the data in part~\ref{i:measure}.
\item Describe the image formed by this distant object.
\end{question}
%
In principle, an object needs to be infinitely far away for this second method to be true.  In practice, the object only needs to be ``far enough.'' Determine, theoretically or experimentally, how far an object needs to be to get an accurate measurement of the focal length in this manner.
If the day is sunny, {\bf under strict instructor supervision} take a lens outside and, using the sun as ``an infinitely far away object,'' measure the focal length by forming an image of the sun on your lab notebook.  Take a minute or so to really visualize the sun on your paper.  Predict the results.


%
%\item  The focal length of a lens is related to the radius of curvature
%of each of its surfaces and the index of refraction of the glass by the
%lens makers equation:
%%
%\[\frac{1}{f} = (n - 1) \left(\frac{1}{r_1} + \frac{1}{r_2}\right)\]
%%
%Using the spherometer, determine the radius of curvature of the
%surfaces of a larger lens and compute the value of the index of refraction of
%the glass.  Note: for the spherometer, use the relationship that r=d$^2$/(2h)
%for the following diagram:

%\end{enumerate}



%\newpage \ \
%========================================
\onecolumn

\pagestyle{plain}

\section*{Specific Heat Pre-Lab}
\revised{Feb 23, 2009}
\setcounter{dave}{0}


% NO ANSWER
%\newcommand{\ans}[2]{#2}
% PRINT ANSWER
%\newcommand{\ans}[2]{{\color{blue}{#1}}}

%\newcommand{\relunc}[1]{\displaystyle \frac{\delta #1}{#1}}

Use this worksheet to calculate the value of specific heat value based on the given pre-lab data.  This sheet will help you keep track of the uncertainty. Answer the questions from the pre-lab in your notebook.
We will use the notation: subscript $c$ for ``cup'', subscript $s$ for ``stir rod'', subscript $a$ for ``aluminum'', subscript $b$ for the metal block, and subscript $w$ for the water.

In lab, you are to calculate the specific heat, $c_b$:
%
\[ c_b = \frac{m_a c_a \left(T_f - T_{ia}\right) + m_w c_w \left(T_f - T_{iw}\right)}{-m_b \left(T_f - T_{ib}\right)} \]
%
We will consider a careful examination of the ``error analysis'' based on the {\bf measurements} provided:
%
%\begin{table}[bhpt]
%\caption{
(The measured data has been entered.  You should fill in the relative uncertainty.)
%}\label{t:c-measured}
\[ \begin{array}{crlll}
m_c = & 60.0 & \pm\ .1 & \unit{g} & \underline{\frac{.1\unit{g}}{60.0\unit{g}} = 0.17\%} \\ %\ans{0.17\%}{\blank} \\
m_s = &  3.2 & \pm\ .1 & \unit{g} & \ans{3.1\%}{\blank} \\
m_a = m_c + m_s = & 63.2 & \pm\ .1 & \unit{g} & \ans{.16\%}{\blank} \\
m_w + m_a = & 152.3 & \pm\ .1 & \unit{g} & \ans{0.066\%}{\blank} \\
m_b = & 74.8 & \pm\ .1 & \unit{g} & \ans{.13\%}{\blank} \\ \\
T_{ia} = T_{iw} = & 22.3 & \pm\ .1 & \unit{^\circ C} & \ans{.45\%}{\blank} \\
T_{ib} = & 99.4 & \pm\ .1 & \unit{^\circ C} & \ans{.10\%}{\blank} \\
T_f    = & 27.0 & \pm\ .1 & \unit{^\circ C} & \ans{.37\%}{\blank} \\ \\
c_a = & 0.91 & \pm\ .01 & \unitfrac{kJ}{kg \cdot K} & \ans{1.1\%}{\blank} \\
c_w = & 4.186 & \pm\ .001 & \unitfrac{kJ}{kg \cdot K} & \ans{.024\%}{\blank}
\end{array} \]
%\end{table}
%
When you add or subtract, then add the uncertainties (and then find the relative uncertainty).
When you multiply or divide, then add the relative uncertainties (and then find the uncertainty).
%You can find the mass of the water by subtracting the mass of the cup from the mass of the ``cup with water''.
%You should assume that the aluminum cup is always in thermal equilibrium with the water.

To get a value for $c$, we need the numerator, $Q_w+Q_a$, and the denominator, $m_b \Delta T_b$.
For the numerator, we need $Q_w$ and $Q_a$.
To get $Q_w$ we need $m_w$ and $\Delta T_w$ because we have a value for $c_w$.
To get $Q_a$, we have $m_a$ and $c_a$, and we need $\Delta T_a$.
For the denominator, we have $m_b$ and need $\Delta T_b$.
\[ \begin{array}{crlllc}
m_w = m_{\rm a + w} - m_a = & \ans{89.1}{\blank} & \pm\ \ans{.2}{\blank} & \unit{g} & \ans{.22\%}{\blank} & \delta m_w = \delta m_{(a+w)} + \delta m_a \\[.1cm]
\Delta T_a = \Delta T_w = & \ans{4.7}{\blank} & \pm\ \ans{.2}{\blank} & \unit{^\circ C} & \ans{4.26\%}{\blank} & \delta \Delta T = \delta T_f + \delta T_i \\[.1cm]
\Delta T_b = & \ans{-72.4}{\blank} & \pm\ \ans{.2}{\blank} & \unit{^\circ C} & \ans{.276\%}{\blank} & \delta \Delta T = \delta T_f + \delta T_i
\end{array} \]
For the numerator, then
\[ \begin{array}{crllll}
Q_w = m_w c_w \left(T_f - T_{iw}\right) = & \ans{1753}{\blank} & \pm\ \ans{79}{\blank} & \unit{J} & \ans{4.5\%}{\blank} &
\relunc{Q} = \relunc{m_w} + \relunc{c} + \relunc{(\Delta T)} \\[.5cm]
Q_a = m_a c_a \left(T_f - T_{ia}\right) = & \ans{270}{\blank} & \pm\ \ans{15}{\blank} & \unit{J} & \ans{5.5\%}{\blank} &
\relunc{Q} = \relunc{m_a} + \relunc{c} + \relunc{(\Delta T)} \\[.5cm]
Q_w + Q_a = & \ans{2023}{\blank} & \pm\ \ans{94}{\blank} & \unit{J} & \ans{4.7\%}{\blank} & \delta ({\rm numerator}) = \delta Q_w + \delta Q_a
\end{array}
\]
and for the denominator,
\[ \begin{array}{crllll}
m_b \left(T_f - T_{ib}\right) = & \ans{5416}{\blank} & \pm\ \ans{22}{\blank} & \unit{g \cdot K} & \ans{.41\%}{\blank} &
\relunc{(m_b\, \Delta T)} = \relunc{m_b} + \relunc{(\Delta T)}
\end{array}
\]
Therefore:
\[ \begin{array}{crllll}
\displaystyle
c_b = \frac{Q_w + Q_a}{-m_b \left(T_f - T_{ib}\right)} = & \ans{.374}{\blank} & \pm\ \ans{.019}{\blank} & \unitfrac{kJ}{kg \cdot K} & \ans{5.1\%}{\blank} &
\relunc{c} = \relunc{({\rm numerator})} + \relunc{(m_b \, \Delta T)}
\end{array}
\]


%========================================
\onecolumn

\section*{Specific Heat Worksheet for Your Results}
\revised{Feb 23, 2009}
\setcounter{dave}{0}

Use this worksheet to calculate your specific heat value based on your measured data.  This sheet will help you keep track of the uncertainty and the questions from the pre-lab exercise should help you write your analysis.
We will use the notation: subscript $c$ for ``cup'', subscript $s$ for ``stir rod'', subscript $a$ for ``aluminum'', subscript $b$ for the metal block, and subscript $w$ for the water.

In lab, you are to calculate the specific heat, $c_b$:
%
\[ c_b = \frac{m_a c_a \left(T_f - T_{ia}\right) + m_w c_w \left(T_f - T_{iw}\right)}{-m_b \left(T_f - T_{ib}\right)} \]
%
We will consider a careful examination of the ``error analysis'' based on the {\bf measurements} provided:
%
\[ \begin{array}{crlll}
m_c = & \blank & \pm\ \blank & \unit{g} & \blank \\
m_s = & \blank & \pm\ \blank & \unit{g} & \blank \\
m_c + m_s = & \blank & \pm\ \blank & \unit{g} & \blank \\
m_w + m_c = & \blank & \pm\ \blank & \unit{g} & \blank \\
m_b = & \blank & \pm\ \blank & \unit{g} & \blank \\ \\
T_{iw} = & \blank & \pm\ \blank & \unit{^\circ C} & \blank \\
T_{ib} = & \blank & \pm\ \blank & \unit{^\circ C} & \blank \\
T_f    = & \blank & \pm\ \blank & \unit{^\circ C} & \blank \\ \\
c_a = & \blank & \pm\ \blank & \unitfrac{kJ}{kg \cdot K} & \blank \\
c_w = & \blank & \pm\ \blank & \unitfrac{kJ}{kg \cdot K} & \blank
\end{array} \]
%
When you add or subtract numbers, you should add the uncertainties (and then find the relative uncertainty).
When you multiply or divide numbers, you should add the relative uncertainties (and then find the uncertainty).
%You can find the mass of the water by subtracting the mass of the cup from the mass of the ``cup with water''.
%You should assume that the aluminum cup is always in thermal equilibrium with the water.

To get a value for $c$, we need the numerator, $Q_w+Q_a$, and the denominator, $m_b \Delta T_b$.
For the numerator, we need $Q_w$ and $Q_a$.
To get $Q_w$ we need $m_w$ and $\Delta T_w$ because we have a value for $c_w$.
To get $Q_a$, we have $m_a$ and $c_a$, and we need $\Delta T_a$.
For the denominator, we have $m_b$ and need $\Delta T_b$.
\[ \begin{array}{crlllc}
m_w = m_{\rm cup + water} - m_c = & \blank & \pm\ \blank & \unit{g} & \blank & \delta m_w = \delta m_{(c+w)} + \delta m_c \\[.1cm]
\Delta T_w = & \blank & \pm\ \blank & \unit{^\circ C} & \blank & \delta \Delta T = \delta T_f + \delta T_i \\[.1cm]
\Delta T_b = & \blank & \pm\ \blank & \unit{^\circ C} & \blank & \delta \Delta T = \delta T_f + \delta T_i
\end{array} \]
For the numerator, then
\[ \begin{array}{crllll}
Q_w = m_w c_w \left(T_f - T_{iw}\right) = & \blank & \pm\ \blank & \unit{J} & \blank &
\relunc{Q} = \relunc{m_w} + \relunc{c} + \relunc{(\Delta T)} \\[.5cm]
Q_a = m_a c_a \left(T_f - T_{ia}\right) = & \blank & \pm\ \blank & \unit{J} & \blank &
\relunc{Q} = \relunc{m_a} + \relunc{c} + \relunc{(\Delta T)} \\[.5cm]
Q_w + Q_a = & \blank & \pm\ \blank & \unit{J} & \blank & \delta ({\rm numerator}) = \delta Q_w + \delta Q_a
\end{array}
\]
and for the denominator,
\[ \begin{array}{crllll}
m_b \left(T_f - T_{ib}\right) = & \blank & \pm\ \blank & \unit{g \cdot K} & \blank &
\relunc{(m_b\, \Delta T)} = \relunc{m_b} + \relunc{(\Delta T)}
\end{array}
\]
Therefore:
\[ \begin{array}{crllll}
\displaystyle
c_b = \frac{Q_w + Q_a}{-m_b \left(T_f - T_{ib}\right)} = & \blank & \pm\ \blank & \unitfrac{kJ}{kg \cdot K} & \blank &
\relunc{c} = \relunc{({\rm numerator})} + \relunc{(m_b \, \Delta T)}
\end{array}
\]



%\end{document}


%\end{document}




\onecolumn%========================================

\section{Measuring Planck's Constant}
\revised{Spring, 2011}
\label{s:Planck}
\setcounter{dave}{0}

\vfill
\centerline{This lab description will be provided before you do the experiment.}
\vfill












\section{Laboratory Instructions}

The laboratory work in General and University Physics is designed to:
\begin{enumerate}
\item  Acquaint you with laboratory techniques,
\item  Correlate the academic work of the classroom with actual laboratory
experience,
\item  Develop problem-solving skills,
\item  Give experience drawing scientific conclusions, and
\item  Show the close relationship between science and mathematics and to
help the student gain facility using scientific analysis.
\end{enumerate}

The laboratory work is scheduled as given, please
{\bf study the procedure and theory before coming to the laboratory}.
Reference materials, laboratory manuals, the text, etc.\ should be used
freely.  There will be a weekly pre-lab quiz over this material.

All data taken in the laboratory must be re\-cord\-ed on laboratory note paper
in a neat, orderly fashion with proper headings for rows and columns of tables.
The correct units must always be given.  It is your responsibility
to turn in the carbon copy of your data sheet at the end of each lab period.

\subsection{The Report}\note{too long?}

A formal report is required of each student for each experiment.  Reports should be
written for a fellow student with similar class experience but who did not
experience this experiment.

The report is due
at lecture-time four days from the date of the experiment.  Reports may not
be accepted if turned in later than this date unless a satisfactory (to the
instructor) reason is given; late reports are penalized by a reduced grade.
They must be typed with double-spacing and will be judged on the basis of
English usage as well as on scientific content.  The report should
contain the following:

\begin{description}
\item[(Title)] Names of author {\it and co-experimenters}, name of experiment, date
experiment was performed.

\item[Abstract:] An abstract is a brief summary of your report; although
it accompanies your report, {\bf it is not considered part of
your report and should not be referred to by any other section of the report}.
The abstract should start with a simple statement of the objectives of the
experiment, followed by a simple statement what you measured in order to
achieve that objective.  Finally, you should {\it quantitatively} indicate how
successful your experiment was by citing the numerical results (with uncertainty)
which provide the evidence for your conclusion.  You may also include a relevant
percent difference.

{\it The point of the abstract is to tell someone who is
already familiar with the concept of your experiment how your specific
experiment went.}


\item[Apparatus:] A list of the equipment used in the experiment including
a description of any new or unfamiliar pieces or of unusual uses of some
familiar piece.  Diagrams should be given wherever possible.   Record serial
numbers or other identification when possible.

{\it The point of this record
is to allow the later verification of your measurements.}

\item[Theory:] Although the abstract captured the essence of the concept
explored by this exercise, this section should explain the underlying concepts.
It should be written {\bf to an audience} with the background knowledge of one of
your fellow students who has been in class with you, but was unable to attend
this particular lab exercise.  You can treat ideas from class as familiar when
you reference them, but some of these may still need an explanation to connect
them to this experiment.  You should {\bf begin by} stating the objectives of the
experiment, the principle which you are trying to verify.  This might not
be as simply stated as in the abstract since the goal here is to explain,
rather than merely state, how it is that the principle can be proven.
Once you have stated the objectives, you will {\it briefly} give the experiment
a conceptual context by discussing the physical laws and concepts involved
in reaching your objectives, by defining new terms, and by characterizing
physical laws and relevant models.  Equations used should be derived in order
to relate the form of the equation that expresses the principle (as presented in class)
to the form of the equation that will be useful for your calculation.
All variables should be identified and it is important to relate the
equations used to the physical situations that they represent, while showing
how measured variables can be related to the final calculated result.

{\it The point of the theory section is to connect what we knew before the lab
to what we have discovered through the lab, to introduce the reader to the important
ideas that can be applied to the physical relationships, to connect how the relationships
(possibly expressed via equations) might be verified by considering a particular graph,
and to compare what we should see if the theory is correct to what we might see
if the theory were actually different than we are proposing.}

\item[Procedure:] A brief outline of the experimental procedure.
{\bf This must be in the student's own words}, not copied or even transposed
from the instruction sheets or from other laboratory manuals.  Apply the
general ideas of the theory to justify or explain the specific steps.
Note that scientific reports are generally written in the past tense,
passive voice: ``measurements of the half-life were made using the technique\ldots''
rather than the active voice: ``we measure the half-life by\ldots''

{\it The points of this section are: to enable the reader to visualize which measurements
were made and how they relate to the theory, to allow you to repeat the experiment
at a later date, and to provide enough information that another student (who has not
done the lab) to reconstruct your experiment.}

\item[Data:] Provide a table of measured and calculated data including
uncertainties and units.  Show how your calculations were performed.  Whenever
possible, a graphical record of the data should be given.
The original data sheet must have been turned in at the end of lab.
Even if you did not measure the data in order smallest-to-largest,
you should report it this way so that patterns in the data are obvious by
glancing down the column.

\item[Analysis:] Analyze your data based on the predictions of the
Theory Section.  Describe important features of the data and how they
express various features of the theory,
such as: Is the graph linear or quadratic? What are the slope and intercept?
Is your result reasonable and consistent in the context of the theoretical expectations?
Cite uncertainty or \%-uncertainty as applicable.
Cite \%-error or \%-difference as applicable.
Give a {\bf quantitative} statement of the sources of uncertainty and their
effect on the results of the experiment.  Explain the steps taken to
minimize the uncertainties.

{\it The point of this section is to provide the connection between the data and
the theory in order to draw a conclusion in the next section.}

\item[Conclusions:] A brief discussion of what you can
conclude {\it about} the initial assumptions and objectives {\it based on} the analysis.
Reference the theory section as appropriate.
Cite the relevant results from your analysis which support your conclusion.
Why do these numbers support your conclusion?

\end{description}

\note{\subsubsection{The Presentation}

For some labs the lab groups will present their results orally
as well as in a group report.  Each student will have
7-10 minutes to present a section of their group's oral report.  After each group
presentation, questions may be asked of the group for up to 5 minutes.
Student grades will be a combination of individual and group performance.

Begin writing the report in the week of the experiment.
The grade for the written report will be the same for all members of the group.

\subsubsection{Peer Review}

In scientific fields, quality is controlled by the peer-review process.
An article is submitted to a journal.  The editor sends it to experts in the field
(to colleagues and competitors) who review it and make recommendations to the editor.
The editor then returns it to the author who re-writes it and re-submits it
until the editor accepts it.   Turn in your report with a code name.
Somebody will review it for content and clarity.
You will then re-write your report based on these comments and a grade will be given for
the report as well as for the review.
}
















\onecolumn \section[Measurements and Reminders]{Common Measurements and Reminders}

\subsection{Repetitive Motion}
\label{ss:oscillation}

When any object moves with a pattern which repeats at regular time intervals, it can be described as \latex{``}\html{"}oscillating" and is said to be in periodic motion.
This applies to pendulums swinging, clock-hands whirring, merry-go-rounds spinning, water-waves undulating, sound waves, light-waves, radio-waves, a heart-beat, etc.
Regardless of the object in motion or the type of motion, there are a few terms which are appropriately applied.

\bigskip
The period, $T$, is the amount of time elapsed during one oscillation.  (Periodic motion is motion which has a period.)  The units are seconds-per-oscillation; although you may also see them as minutes-, hours-, or days- (among other possibilities) per oscillation.
\begin{quote}
{\bf Aside:}\/
Despite these units being a time per oscillation, it is often treated as a straight time interval; the number of oscillations in that case is assumed to be 1.  Technically the period is not simply a time, but the time is found by multiplying the period by the quantity $1\unit{oscillation}$.  So, the conversion is easy to do in your head.
\end{quote}
Numerically, a large period indicates more time per oscillation, which describes slower motion.

\bigskip
The frequency, $f$, is the rate of oscillation: the number of oscillations during some time.  The frequency is the reciprocal of the period, $f=\boxfrac{1}{T}$, and has units of oscillations-per-second.  Numerically, a large frequency (a small period) indicates more oscillations during some time, which describes faster motion.

\bigskip
The angular frequency (or angular velocity), $\omega$ is another related term.  This is generally reserved for circular motion, but can be applied to periodic motion as well.  The difference between the frequency and the angular frequency is that the angular frequency is measured in {\sl radians}-per-second rather than in {\sl oscillations}-per-second.  Radians are considered a unitless dimension -- the natural measurement of angles.  There are $2\pi$ (radians) in $360^\circ$ (which is one full circle/oscillation).  So, we convert from frequency to angular frequency via
\[ \omega = \frac{2\pi}{1 \unit{osc}} f  = \frac{2\pi}{T (1 \unit{osc})}\]
Notice that in the second equality, we see the product of the period and 1 oscillation.  this is the time (measured in seconds) of one oscillation.  The units of $\omega$ are \latex{``}\html{"}reciprocal seconds" ($\unit{s^{-1}} = \unitfrac{1}{s}$).

\bigskip
Your pulse is generally measured in beats-per-minute.  This is a frequency.

\subsubsection{Measurements of These Quantities}

In order to measure any of these quantities, we need to time the oscillations.  We obviously don't need to measure both the frequency and the period since we can easily find one from the other.  These variables are measuring the same relationship, but expressing it differently.  We need to decide on the convenience of when to stop counting.  You can either measure the period by timing a specified number of oscillations or you can measure the frequency by counting the oscillations during a specified amount of time.

If you count the number oscillations during a specified amount of time, you must worry about the final fraction of an oscillation.  In the following example, you will notice that we ignore the initial and final fractional beats.  This adds to the uncertainty.  (I have assumed it is $\pm1\unit{beat}$, but it could easily be $\pm 2 \unit{beats}$ depending on the timing of our start and stop times.)

\begin{quote}
{\bf Example:}\/ When your pulse is taken, it is often taken for ten or fifteen seconds and extrapolated to a minute.
It is possible to take my pulse for one minute.
In doing so, you may count 15 beats in the first ten seconds ($\approx 90\unitfrac{beats}{min}$).  If you continue, you might find 29 beats after twenty seconds ($\approx 87 \unitfrac{beats}{min}$).  Perhaps then you see 43 beats after thirty seconds ($\approx 86 \unitfrac{beats}{min}$).

The problem here is that {\sl extrapolation} is intrinsically imprecise.  If we track the uncertainty of the first measurement,
%
\[ \frac{15\pm 1 \unit{beats}}{10\unit{sec}}
= 1.5 \pm 0.1 \unitfrac{beats}{sec} \convert{60\unit{sec}}{1\unit{min}}
= 90 \pm 6 \unitfrac{beats}{min} \]
%
On the other hand, if you count for $20\unit{sec}$, then you find
%
\[ \frac{29\pm 1 \unit{beats}}{20\unit{sec}}
= 1.45 \pm 0.05 \unitfrac{beats}{sec} \convert{60\unit{sec}}{1\unit{min}}
= 87 \pm 3 \unitfrac{beats}{min} \]
%
These numbers are consistent, but the second is more precise.  We cannot say at this point which is more accurate because we do not know the {\sl true}\/ solution.  One more time:
%
\[ \frac{43\pm 1 \unit{beats}}{30\unit{sec}}
= 1.4\overline{3} \pm 0.0\overline{3} \unitfrac{beats}{sec} \convert{60\unit{sec}}{1\unit{min}}
= 86 \pm 2 \unitfrac{beats}{min} \]
%
\end{quote}

\bigskip
The obvious trend in this example is that longer measurements correspond to more precise results.  Therefore, when measuring a period (or a frequency) you should consider long time intervals.  As mentioned above, we should also worry about fractional oscillations when measuring a frequency.

On the other hand, if we measure the period, then we can use the precision of the clock and measure whole oscillations.  In doing this, we ought to measure the time for {\sl many} (greater than 20) oscillations and divide that time by the number of oscillations.  The more oscillations which are averaged over in this manner, the better, because your timing-uncertainty is divided by the number of oscillations as well.  Furthermore, you have reaction-time uncertainty at the beginning and at the end.  To minimize this effect we should make these 2 oscillations a small percentage of our total oscillations:  2 is 20\% of 10 oscillations, is 10\% of 20 oscillations, and is 1\% of 200 oscillations.

Additional improvements can also be made by repeating the measurement two or three times.

\subsection{Waves}
\label{ss:waves}

Sec.~\ref{ss:oscillation} deals with repetitive motions and describes the period, frequency, and angular frequency.  In this section, we consider waves (like water waves, but also sound waves and light waves).  In addition to repeating themselves, they have some physical size to them.  We can discuss the property of {\sl wavelength}.


\bigskip
The wavelength, $\lambda$, (similar to the period) is the distance spanned during one oscillation.  The units are meters-per-oscillation.  (Also similar to the period, the units of wavelength are sometimes called meters, but this is not technically correct.)
Numerically, a large wavelength obviously indicates a larger wave.

\bigskip
Although there is no quantity which is analogous to the frequency, there is a quantity which is analogous to the angular frequency:  The wave number, $k$, is the spatial rate of oscillation. The units of the wave number are \latex{``}\html{"}reciprocal meters" ($\unit{m^{-1}} = \unitfrac{1}{m}$).  It is related to wavelength the same way angular frequency is related to the period:
\[ k = \frac{2\pi}{1 \unit{osc}} \frac{1}{\lambda}  = \frac{2\pi}{\lambda (1 \unit{osc})}\]
The $2\pi$ is present here for the same reason it was present in the definition of the angular frequency.  Numerically, a large wave number (a small wavelength) indicates a smaller wave.

Recalling that the velocity is measured in distance per time, we can use these quantities to find the speed of the wave.  Since $\lambda$ is meters-per-oscillation and period is seconds-per-oscillation, we can write the speed in terms of the wavelength and either the period or the frequency.  We can also write it in terms of the wavenumber.  The typical notation is the wavelength and the frequency.
\[\fbox{$v = \lambda f$} = \frac{\lambda}{T}  = \frac{\omega}{k} \]

\subsection{Excel Reminder}
\label{ss:excel}

Hopefully you have saved last semester's lab manual.  If not, a copy of the Excel lab can be found in the print shop (in Old Main) for the price of making a copy.

\bigskip\noindent

The Excel function [LINEST] will generate the data
relevant to a linear trendline.  (It only works for linear trendlines.)
%
\begin{enumerate}
\item In the formula \[ \mbox{=index(index(linest(L2:L12,A2:A12,true,true),1),1)}\]
the first \latex{``}\html{"}1" means \latex{``}\html{"}give the value, not the
uncertainty." The second \latex{``}\html{"}1" means \latex{``}\html{"}give the
slope, not the intercept."
%
\item In the formula \[ \mbox{=index(index(linest(L2:L12,A2:A12,true,true),1),2)}\]
The \latex{``}\html{"}1" means \latex{``}\html{"}give the value, not the
uncertainty." The \latex{``}\html{"}2" means \latex{``}\html{"}give the
intercept, not the slope."
%
\item In the formula \[ \mbox{=index(index(linest(L2:L12,A2:A12,true,true),2),1)}\]
The \latex{``}\html{"}2" means \latex{``}\html{"}give the uncertainty, not the
value." The \latex{``}\html{"}1" means \latex{``}\html{"}give the slope, not
the intercept."
%
\item In the formula \[ \mbox{=index(index(linest(L2:L12,A2:A12,true,true),2),2)}\]
The first \latex{``}\html{"}2" means \latex{``}\html{"}give the uncertainty,
not the value." The second \latex{``}\html{"}2" means \latex{``}\html{"}give
the intercept, not the slope."
\end{enumerate}

\addtocounter{section}{-2}
\renewcommand{\thesection}{\arabic{section}}
\renewcommand{\thefigure}{\arabic{section}.\arabic{figure}}
\renewcommand{\theequation}{\arabic{section}.\arabic{equation}}
\renewcommand{\theenumi}{\arabic{section}.\arabic{enumi}}

\onecolumn




%-------------------------------------------------------------------------------------

\onecolumn
\section{Speed of Sound in a Cardboard Tube}
\revised{(Jan, 2007)}
\label{s:sound-cardboard-tube}

\subsection{Introduction}

Longitudinal sound waves can be produced by any vibrating source.  The frequency of these waves is determined solely by this vibrating source.  There must also be a medium (solid, liquid, or gas) in order for the sound wave to be propagated.  These waves do not travel through the medium instantaneously.  There is a finite wave speed which is determined by the characteristics of the medium.   The wave speed is not dependent on the source.  If the medium is changed, then the speed of sound also changes.  For example, the pressure of a gas, the temperature of a gas, and the gas composition are all factors which affect the speed of sound in the gas.

Sound waves can experience interference just like waves on a string, especially when the waves are inside a tube, like an organ pipe.   Traveling waves can reach the end of the tube, then they can be reflected back in the direction in which they came.  There are now two sets of waves which can interfere, that is the two sets of amplitudes are added together.   At certain frequencies this interference gives rise to a special wave called a standing wave.  This is a resonance effect.  Standing waves can occur in tubes which have only one end open, or in tubes that have both ends open.   The derivation of the equation for these special resonant frequencies will be slightly different for these open or closed tubes.  See your text book for the needed pictures and equations.

\subsection{Experimental Objectives}

The purpose of this experiment is to determine the speed of sound in an air column, both from an open and a closed tube.  Data should be taken at a number of different resonant frequencies.

\subsection{Pre-Lab Work}

Show a set of pictures and equations for the first five harmonics, for both an open and for a closed tube.  Show and explain what nodes and antinodes are.   Show and explain what a standing wave is.
	
\subsection{Experimental Procedure}

Initial set-up:
\begin{lablist}
\item The source of the waves will be a stereo type speaker, powered by a frequency oscillator.   The waves will move through the normal room air inside of a cardboard tube.  The speaker will be placed at one end of the pipe.
\item A microphone (Pasco) will be placed to receive the waves at the same end of the pipe.
\item The Pasco Interface and the software Data Studio in the oscilloscope mode will be used to determine which frequencies give the maximum sound intensity for resonance.
\item Measure the length of the tube.
\item Calculate the resonant frequency for the n=3 harmonic, using an approximate value for the speed of sound.
\item Set the generator frequency near this calculated value.
\end{lablist}
%
Doing the experiment:
%
\begin{lablist}
\item Adjust the sound intensity amplitude on the generator (not too loud).
\item Adjust the frequency (on the generator) until the amplitude (on the scope) is actually at the maximum.  This is called the resonance condition. Record the small range of frequencies which keep the amplitude at a maximum.
\item Repeat this procedure to determine 4 additional resonance frequency points.
\item Repeat this for both an open tube and for the closed end tube. (Five points for each type of tube.)
\item Make a graph of the resonant frequency versus the harmonic number, for each tube type.
\item Determine the speed of sound from each graph.
\item Calculate a speed of sound value at $20^\circ$C.
\item Determine the precision and accuracy for this experiment.
\end{lablist}


\subsection{Questions}

\begin{enumerate}
\item Does the speed of the wave depend on the frequency of the oscillator?
\item How does the speed of sound in air vary with the air temperature?   By how much would the results of your experiment change if you conducted the experiment outside today?
\item What is meant by resonant condition?
\item In interference, at least 2 sets of waves are added together.  What 2 sets of waves are added together in this experiment?
\item Why does the closed tube only show resonance for the odd harmonics?
\item Demonstrate how sound waves can be reflected by the open end of a tube.
\item What random errors might be in this experiment?  And show any evidence of them.
\item What systematic errors might be in this experiment?  And show any evidence of them.
\end{enumerate}





%-------------------------------------------------------------------------------------


\section{Pressure Demonstrations - An Exercise}
\revised{Jan 15, 2009}

The lab room has been set up with six lab-stations; each has what is usually a class-room demonstration.  For today, you will be exploring the phenomenon of interest in order to discover the point of it.  You will then write about a half-page on each.  It should be thorough enough to explain the phenomenon to a colleague who missed today's lab.  It is possible that the class exam for this portion of the course will have a question on one of these phenomena.

You should form into five groups so that there is always an empty station.  Take notes in your lab notebook, but then write it up pretty, and in the order of the section numbers, to turn in.

\hrule
\subsection{Glassware and Capillaries}

\subsubsection{Equipment}
\begin{description}
\item[Capillaries] A rack of glass tubes, the base has water, the tubes rest through an upper support with their base in the water.
\item[Glassware] A medium beaker of colored water.  A single piece of glass that has four ``arms'' reaching up, each with a different shape, empty of water.  (See Fig. 9.7 on page 292 and the photograph above Fig. 9.10 on page 293 of your textbook by Hecht.)
\end{description}

\subsubsection{Glassware}

Pour some water into the glassware.  It will be easiest if you pour into the widest column.
%
\begin{question} \itemsep 0in
\item Do all four columns allow water to raise to the same level?  Why or why not?
\item Do all four columns allow water to raise to the same {\it volume}?  Why or why not?
\end{question}
%
Carefully unclamp the glass and, while watching the fluid, even more carefully pour the water back into the beaker.
%
\begin{question}  \itemsep 0in
\item Do you notice anything interesting about the way the water flows while you pour it back into the beaker?
\item Describe the orientation of the tilt of the glassware as you poured it back into the beaker.  (Maybe a picture would help.)  If you don't notice anything interesting, then pour it more slowly and note the water level and describe where the water is as it empties.
\end{question}
%
Carefully reclamp the glassware.  Gently plug one of the columns with your finger or thumb.  Pour the water back into the glassware while maintaining a seal on that column.
%
\begin{question}  \itemsep 0in
\item Do all four columns allow water to raise to the same level?  Why or why not?
\item Predict what would happen in all columns if you were to remove your finger from the plugged column.  Verify your prediction.
\end{question}
%
Carefully unclamp the glassware.  Gently plug one of the columns with your finger or thumb and, while watching the fluid, even more carefully pour the water back into the beaker while maintaining a seal on that column.
%
\begin{question}
\item Explain what happens to the fluid.  Why does this happen?
\end{question}
%

\subsubsection{Capillaries}

Apparently, fluid always flows to maintain the same level in every available column so long as all columns are open to the air.  Look closely at the water levels in the various capillaries.
%
\begin{question}
\item Using the terms ``meniscus'' and/or ``surface tension'' explain why the water levels are different.
\end{question}
%

\hrule
\newpage
\subsection{The Siphon and Diving Bell}

\subsubsection{Equipment}

\begin{description}
\item[Siphon] Two large beakers, one nearly filled with water; four lab-jacks: one as small as possible, three fully extended with two of the three stacked; one medium-length rubber tube.
\item[Diving Bell]  A small beaker and one of the two large beakers from the Siphon lab.
\end{description}

\subsubsection{The Diving Bell}

There are two questions in your text by Hecht on this topic:  Chap 9,  Question 11 and Problem 52.

\bigskip
\noindent
Using the nearly-full large beaker as the ocean and the small beaker as the diving bell (underwater air reservoir), turn the small beaker upside down and submerge it into the water.  (Be careful not to overflow the water, PLEASE!)
%
\begin{question}
\item Why can a diving bell / submarine have an open bottom and not fill with water?
\item Does the diving bell sink if you let go while it is submerged?  Does it float?  Does it tip sideways?
\item Let {\it some} water into the diving bell.  Does anything change?  What if you completely empty the air from the diving bell (making it useless as a diving bell!)?
\item It {\it should be} possible to allow just enough water in so that the diving bell neither floats nor sinks.
\item Is it possible to use the siphon tube to add or remove air from the diving bell?  Please explain why or why not.
\end{question}

\subsubsection{The Siphon}

There are two questions in your text by Hecht on this topic:  Chap 9,  Question 7 and Multiple Choice 11.

\bigskip
\noindent
Place the two large beakers, one full and one empty, side by side on the table.  Hold the tube above the full beaker and insert one end of the tube into the beaker.
%
\begin{question}
\item Does water enter the tube?  How much water?  Why?
\item Remove the tube from the water.  Plug one end of the tube with your thumb and submerge the other end.  Does water enter the tube?  How much water?  Why?
\item With the tube still inserted in the water and your thumb still plugging the other end, remove your thumb (from the tube not from your hand!).  Does water enter the tube?  How much water?  Why?
\item Repeat the previous step, but plug the tube with your thumb again before all of the water flows into the tube.  Try to catch the water level at various levels.  How is it possible that you can ``catch'' the water in the tube {\it above} the level of the water in the beaker?
\end{question}
%
Empty the water from the tube.  Do not plug the tube and submerge a portion of the tube into the water in the full beaker.  With the tube plugged (on top!) and containing some water, remove the tube from the water.  Be careful to keep the bottom of the tube straight down.  Notice that you can deposit this water into the other beaker.
%
\begin{question}
\item Repeat this, but when you lift the tube out of the water, still plugged on top, tilt the bottom of the tube sideways; but not aimed at anybody!  Keep the top plugged the whole time!  What happens?  Why?
\end{question}
%
Without plugging the tube, insert the entire tube into the water in the full beaker so that the tube is filled with water.  Very carefully, under water, plug one end of the tube.  Do NOT allow the other end of the tube out of the water in the full beaker.  Take the plugged end of the tube out of the water and place it into the empty beaker.
%
\begin{question}
\item Remove your thumb from the tube.  What happens?  Why?
\item As long as the end-that-was-in-the-originally-filled-beaker remains under water, can the middle of the tube be at any height?  Can the end-that-was-removed be at {\it any} height?  What are the restrictions?
\item How much water flows?  When does it stop?  Why?
\end{question}
%
Put all of the water back into a single beaker.  Place this beaker on the lowest lab-jack.
%
\begin{question}
\item Repeat the siphon process.  Does this change anything?
\end{question}
%
Put all of the water back into a single beaker.  Place this beaker on the extended-but-not-stacked lab-jack.
%
\begin{question}
\item Repeat the siphon process.  Does this change anything?
\end{question}
%
Put all of the water back into a single beaker.  Place this beaker on the stacked lab-jacks.
%
\begin{question}
\item Repeat the siphon process.  Does this change anything?
\item Why does height matter?  What is the equation in your text that expresses this idea (faster flow with larger surface-height difference)?
\end{question}
%

\hrule
\subsection{The Palm Glass}

\subsubsection{Equipment}
\begin{description}
\item[Palm Glass] A completely sealed glass with colored water inside.  The glass has two bulbs connected by a long narrow tube.  Fluid can flow from one bulb to the other by holding the connecting tube and tilting the glass so that one bulb is higher in the air.
\end{description}

Hold the tube between a single finger and thumb.  Tilt the glass so that all of the fluid is in the lower bulb.  Gently place the lower bulb, keeping it lower than the other bulb, into your hand and gently close your fingers around the bulb.  Watch the higher bulb while you wait.  Wait a little more.
%
\begin{question}
\item What happens to the fluid?  Why?
\end{question}
%
Without changing the orientation of the bulbs, gently grasp the tube with a finger and thumb near the center.
%
\begin{question}
\item Does the fluid return?  Why or why not?
\end{question}
%
Repeat the original step and wait for a short time.  Without changing the orientation, gently grasp the upper bulb with your free hand and then remove the original hand.
%
\begin{question}
\item What happens to the fluid?  Why?
\end{question}
%

%\vfill
\hrule
\subsection{The Black Hole and Suction Cups}

There's a science fiction story about a black hole that was headed towards the Earth.  The hero of the story flew dangerously close to it, threw in lots of positive electrical charge and then used negative charge to tow it into orbit around the Earth.  Once it was near a stable orbit, the hero threw in the negative charge in order to make it neutral (and un-towable, so it would continue in the stable orbit) and enclosed it in a perfectly spherical shell.  But the shell had a stop-cock in it.  Once enclosed, the hero sold ``perfect vacuum'' to whomever had a need of it.  The hero hooked a vessel up to the stop-cock and it sucked out any gases!

\subsubsection{Equipment}
\begin{description}
\item[Suction Cups] Two suction cups, apparently identical\ldots a shallow pan or bowl or beaker of water to wet the face of the suction cups.
\item[The Black Hole] Two hemispheres with a rubber gasket in the slot of one.  The portion of each hemisphere that will be connected should have vacuum grease smeared around it.  One hemisphere as a nozzle that should be closed.  The other hemisphere has had its handle broken off and only has a stub left.  It is very sad and many of us cried.  You will have to manage to continue without it.  You will also have a caliper or a meter stick and hand-controlled vacuum pump, with tube attached (but not attached to the hemisphere).  Be sure that you know {\it where} the release valve is on the vacuum pump and {\it how to use it} BEFORE you start.
\end{description}

\subsubsection{The Black Hole}

Press the two hemispheres together.  Pull them apart.
%
\begin{question}
\item Notice that it is slightly easier to separate them if you slide one away from you and the other towards you than if you merely pull the hemispheres directly away from each other.  Why is this?
\item Does it matter if the nozzle is open or closed?
\item Notice that one hemisphere has a lip around the edge so that if you press them together, crushing the rubber gasket, then the lip interferes with the sliding of the hemispheres across each other and they must be pulled apart.
\end{question}
%
Attach the tube from the vacuum pump to the nozzle on the hemisphere.  Open the nozzle.  Press the two hemispheres together.  While holding them together, have somebody work the pump.  The gauge on the pump should change.  Close the nozzle and remove the tube from the hemisphere.  Try to pull the hemispheres apart.  (This is how the one handle broke!)
%
\begin{question}
\item The vacuum pump removed the air from inside.  Why is this so much more difficult to separate?
\item If there is nothing inside, then what is holding the hemispheres together?
\item Although the surface area of a sphere is $A=4\pi R^2$, we need the cross-sectional area - the area of the circle between the hemispheres: $A=\pi R^2$.  Convert $P=1\unit{atm}$ to the appropriate units and compute the force holding the spheres together.
\item  The previous question assumed that there is a perfect vacuum inside the sphere.  Assuming that you brought the inner pressure to $P=0.25\unit{atm}$, the pressure difference is $0.75\unit{atm}$.  Use {\it this} value to find the force holding the spheres together.
\end{question}
%

\subsubsection{The Suction Cups}

Wet each suction cup and press them into the table.  Pull up.
%
\begin{question} \itemsep 0in
\item One works the way you expect.  The other doesn't.  Why?  What's the difference in the cups?
\item Notice that, for the one that works, it is slightly easier to remove it if you slide it to the edge of the table.  Why is this?
\item For the one that works, why do you have to press it into the table?  What does this do?  Why is that important?
\item How does a suction cup work?
\end{question}
%

\hrule

\subsection{The Cartesian Diver and Galilean Thermometer}

\subsubsection{Equipment}

\begin{description}
\item[Galilean Thermometer] A Galilean thermometer.  (This is a glass tube with a fluid that looks like water - but isn't - and several differently colored bulbs with a temperature sign dangling from each.  The bulbs with a sign that refer to a temperature cooler than room temperature are that the bottom and the bulb with signs that indicate temperatures above room temperature are at the top.  One bulb may be drifting in the middle somewhere.
\item[Cartesian Diver]  There are two forms to this.  (1) A glass tube mostly filled with water, with a rubber membrane strapped on top and a small glass ``diver'' inside.  (2) A plastic bottle mostly filled with water, with a small glass ``diver'' inside.  In either case, the glass diver may actually be an eye-dropper.
\end{description}

\subsubsection{The Cartesian Diver}

With format (1), press down on the rubber membrane \underline{gradually} pushing it further into the tube until something interesting happens.  (You should not be able to touch the water inside.)  With format (2), squeeze the plastic bottle with a \underline{gradually} firmer grip until something interesting happens.
%
\begin{question}  \itemsep 0in
\item Can you control the depth of the diver?
\item Can you make the diver hover halfway to the bottom?
\item Why does this action sink the diver?
    \begin{enumerate}  \itemsep 0in
    \item HINT:  Look closely at the diver from the side.  Does anything change when you make it submerge?
    \item HINT:  Why do things sink or float?  Why must the diver sink or float?  How does your action affect this?
    \end{enumerate}
\end{question}

\subsubsection{The Galilean Thermometer}

The bulbs float or sink according to the temperature in the room.  Different bulbs react when the room reaches a different temperature - they are labeled according to the temperature that will change their location.  In principle, if you warmed the thermometer with your hands, you would change which bulbs were submerged - PLEASE DO NOT DO THIS.
%
\begin{question} \itemsep 0in
\item Why does anything sink or float?
\item In the Cartesian Diver, the {\it diver} changed density.  These bulbs are sealed and therefore cannot change density.  What must be happening?
\item Rank the bulbs in order from most dense to least dense.
\end{question}

\hrule
\subsection{The Tornado and Coffee Mug}

\subsubsection{Equipment}
\begin{description}
\item[Coffee Mug] A thermal coffee mug with a sealable lid.  A McDonald's coffee cup with lid will do nicely.
\item[Tornado]  Option one:  A single, open, 2-liter plastic bottle, nearly filled with water; and a sink or other drain.  Option two:  Two 2-liter plastic bottle connected at the mouth by a specially-designed pipe-fitting, one of which is mostly filled with water.  The second bottle is merely to catch the water.
\end{description}

\subsubsection{The Tornado in a Bottle}

Turn the bottle with water upside down.
%
\begin{question}
\item It gurgles and sputters.  Why?  Why doesn't water simply flow smoothly out?
\end{question}
%
If you're using Option One, refill the bottle.  For either option, turn the bottle with water upside down, but this time squeeze the water-filled bottle as it empties.
%
\begin{question}
\item It spurts water out smoothly while you squeeze, but not between squeezings.  Why doesn't water gurgle while you are squeezing?  Why does it gurgle when you aren't squeezing?
\item[] (This looks similar to but is a different process than ``milking'' the bottle.)
\item Speculate about if this would work if you were holding a water-filled balloon that could change shape.
\end{question}
%
If you're using Option One, skip this question.  For Option Two, turn the bottle with water upside down, but this time squeeze the empty bottle as the water empties.
%
\begin{question}
\item It spurts air into the filled bottle while you squeeze.  What happens between squeezings?  Why?
\end{question}
%
If you're using Option One, refill the bottle.  For either option, turn the bottle with water upside down, but this time swirl the bottle(s) to create a tornado inside the bottle.
%
\begin{question}
\item Why does this not gurgle?
\item Speculate on the fastest technique for emptying a bottle of fluid.  Justify your speculation.
\end{question}
%

\subsubsection{The Coffee Mug}

Coffee mugs with lids will generally have a hole opposite the drinking hole.  (McDonald's puts theirs in the very center.)
%
\begin{question}
\item Why is there a second hole?
\item What would happen while you drank your coffee if there were only the one drinking hole?
\item Can you suck water out of a bottle that is completely sealed?  Consider a plastic water bottle, imagine that you open it and without pouring the water into your mouth, instead leave the bottle on the table and suck out all of the water.  Is it possible?  What would happen if you tried?
\end{question}
%



\vfill







\onecolumn%========================================

\section{Newton's Law of Cooling}
\revised{(??)}
\label{s:Cooling}

\vfill
\centerline{(To be Added)}
\vfill












%-------------------------------------------------------------------------------------

\onecolumn
\section{Speed of Sound with a Tuning Fork}
\revised{(Jan, 2006)}
\label{s:sound-tuning-fork}

This experiment is demonstrative rather than discovery-based.
The focus of the report will be to show how you understand the
various facets of the experiment, rather than on how your
result allows you to draw conclusions about a principle.
The features which you are demonstrating to yourself are
constructive and destructive interference, echoes,
that sound is a wave which can interfere with itself and what that means,
the size of a single wavelength of sound in air, and
how fast sound travels.

\bigskip\footnoterule\bigskip

This lab uses the familiar effect of an echo to clarify the concept of
superposition and interference.  As an additional bonus, we can use the
effect to calculate the speed of sound in air.

As may be deduced from envisioning the air as you clap or the vibration of a drumhead,
a sound wave consists of a series of condensations and rarefactions in a medium
(alternating regions which are squished and spread).  These disturbances travel
through the medium with a velocity which
depends only upon the properties of the medium.  At a boundary, the
disturbances are reflected; however, the boundary itself affects how the sound waves reflect.
If a sound wave strikes a surface which is more dense than the material in which it is traveling  (such as from air to water), a condensation is reflected as a condensation and a rarefaction
is reflected as a rarefaction.  On the other hand, if a sound wave strikes a surface which is less dense than the material in which it is traveling  (such as from water to air), a condensation is reflected as a rarefaction and a rarefaction
is reflected as a condensation.

It is possible to send sound into a tube and keep it vibrating back and forth in the tube.
The forward-going wave will interfere with the backward-going wave; this is called superposition.
If the waves ``fit'' inside the tube evenly, then the superimposed waves will either interfere {\sl constructively} to produce a louder sound (this is called resonance) or will interfere {\sl destructively} to produce a softer sound.

Car manufacturers can use this second principle and produce sounds which muffle the engine noise, making the inside of the car quieter.

So, let us take a tube with some water in it.  If we create a sound at the top, which is open, then the sound will enter the tube.  It will reflect off of the water and return.  Because the shape of the tube changes at the top as it leaves, some of the sound gets reflected back into the tube.  Whereas the water reflects the wave as described above, the wave reflected from the open end gets inverted (as a wave reflecting off of a less dense material).

As we adjust the length of the tube (by adjusting the depth of the water) the interference of the sound wave will affect the sound that comes out in an obvious way.  (If it is not obvious now, it will be in lab.)  In order for resonance to occur, the length of the tube (down and back up) must equal what is called a half-integral number of waves: Let $l$ be the depth of the air in the tube.  Resonance occurs when $2l=\frac{1}{2}\lambda$, when $2l=\frac{3}{2}\lambda$,
$2l=\frac{5}{2}\lambda$, $2l=\frac{7}{2}\lambda$, etc.

Sec.~\ref{ss:waves} discusses the relationship between wavelength, frequency, period, and velocity of a wave.  Use this information with the tuning forks provided (tuning forks each have a specific frequency) to deduce the velocity of sound.  Harmonics may make your answer one-half or one-fourth of the correct answer.  You should be able to deduce the correct result from your data.

Determine the velocity of sound in air using four tuning forks of different
frequencies.  Then, using the value for velocity determined above, determine
the frequency of an \latex{``}\html{"}unknown" tuning fork.

\vfill

{\bf Warning:} Ask your instructor about harmonics.










%========================================
\onecolumn

\section{Specific Heats of Metals}
\revised{Nov 26, 2007}
\setcounter{dave}{0}
\label{s:spheat}

\subsection{Introduction}

When an aluminum pie tin is taken out of a $200^\circ$C oven, it will also be at a temperature of $200^\circ$C if it was in thermal equilibrium with the oven.  If the pie is then taken out of the tin, the tin will cool off very quickly and could be picked up with a bare hand within 30 seconds.  On the other hand, the pie itself (the filling) will stay hot for about 15 minutes.  This same idea is also true for increasing the temperature of different objects.  An empty aluminum pie tin when placed in a $200^\circ$C oven will increase its temperature quickly; however the pie's temperature would increase at a slower rate.  This is because the heat absorbed by the pie tin is much less than the heat absorbed by the pie.  This is true for two reasons.  First, is due to their relative masses.  The larger mass will take longer to change its temperature.  Secondly, the substances are just different and their behavior is different.  The quantity of heat that is gained (released) by a substance depends on:  1) the mass of the sample, 2) the nature of the substance, and 3) the magnitude of its temperature change.

At a given pressure and within a certain temperature range the heat transferred to the substance per unit mass of the substance and per unit change in the temperature is a specific characteristic of that substance and is a constant.  This quantity is called the specific heat of the substance (c).  The specific heat has units of joules/(kg$\cdot^\circ$C) or calories/(g$\cdot^\circ$C), where one calorie of heat is defined as the amount of energy required to raise the temperature of 1 gram of water from 14.5$^\circ$C to 15.5$^\circ$C.  The specific heat of water is then defined to be 1 cal/(g$\cdot^\circ$C).

It is convenient to also define a quantity called the molar heat capacity, which is the heat required to raise one mole of the substance one $^\circ$C.  One mole contains Avogadro's number of particles.  In 1819, Dulong and Petit discovered that most metals had nearly the same molar heat capacity (~6 cal/(mole$\cdot^\circ$C).  This discovery was an important piece in the development of the kinetic theory of matter.

If a system is thermally insulated from its surroundings, and if no work is performed on the system, then the heat lost by one part of the system must be gained by another part of the system.  This is called the method of mixtures and is the most common method of thermal calorimetry.  This is an example of the law of conservation of energy, and this principle is the basis for the determination of the specific heat of metals.


\subsection{Experimental Objectives}

The purpose of this experiment is to study the principles of calorimetry in the determination of the specific heat of a particular metal.  The class will then combine their results to test the Dulong and Petit Law.


\subsection{Procedure}

\begin{lablist}
\item A calorimeter cup is a two piece aluminum cup.  The inner Al cup is insulated from the environment by an outer cup, the two cups are connected by an insulating ring.
\item Put a small amount (just enough to completely cover your metal object) of approximately 15$^\circ$C water into the inner can.
\item Tie a short piece of thread to your metal sample, so that it can be removed from the water.  Heat the object in boiling water until the metal sample has reached an equilibrium temperature (about 8 minutes).
\item Remove the metal from the boiling water.  Quickly, but carefully lower the metal into the cold water in the calorimeter cup.  Use a stirring rod to achieve a quicker equilibrium temperature.
\item Record all of the mass values including the inner cup and the stirring rod and initial and final temperature values.
\item Repeat this entire process for a second and third set of measurements.
\end{lablist}

\subsection{Analysis}

\begin{lablist}
\item Use the conservation of energy principle to calculate the specific heat of your metal.  Explain what loses heat and what gains heat in this experiment.   Explain the difference between heat and temperature.   Explain that a substance does not contain heat.
\item Calculate an experimental uncertainty for the specific heat of your metal.  Compare your experimental value of the specific heat with the known value.   Calculate the \% accuracy.
\item Discuss the assumptions made in this experiment, and any random or systematic errors associated with these assumptions. Quantify the uncertainties.  Explain how closely the experiment achieves the ideal conditions assumed when deriving the working equations of the calorimeter.
\item Calculate the molar heat capacity of your metal, and compare to the Dulong and Petit Law.
\item Discuss the results from the rest of the lab groups.
\end{lablist}

\subsection{Questions}

\begin{question}
\item If a much larger amount of cold water was used, what would happen to the results of the experiment?  How would this affect the experimental uncertainty?
\item Does it make any difference whether the water is weighed before or after the transfer?
\item If the metal carried 1 ml of hot water with it, how would the experimental value of the specific heat be affected?
\item What type of uncertainty will the thermometer itself introduce into the experiment?  How important is this?
\item Could this experiment be modified to measure the specific heat of a liquid?  How?
\item If the cold water also had ice in the water, how would this change the calculations?
\item In solar heating applications, the sun's energy is stored in some sort of a medium until nightfall.  Should the medium have a high or a low specific heat value?
\end{question}








%========================================

\twocolumn[\section{Latent Heat}
\revised{Jan, 2006}
\label{s:laheat}

Recall from \hyperref{the {\bf Specific Heats of Solids} Lab}{Lab~}{ ({\bf Specific Heats of Solids})}{s:spheat} that
when two substances at different temperatures are placed in thermal contact,
the hotter substance loses heat while the colder gains heat until thermal
equilibrium is reached.  It is assumed that the heat lost by one is equal
to the heat gained by the other:  Recall Eq.~(\ref{eq:spheat}),
$-Q_{\rm lost} = Q_{\rm gained}$,
where $Q = mc\Delta T$ is the heat necessary \underline{to change}
\underline{the temperature} of
an object, $m$ is the mass of the substance, $\Delta T$ is the change in the
temperature, and $c$ is the specific heat.\\]

\bigskip
In this lab, we are interested in what happens when one of the objects
{\bf changes phase}.  If the heat goes into changing the phase (e.g.,
solid to liquid)
rather than into warming the object $(\Delta T)$, then we write that the
heat is
%
\begin{equation}\label{eq:laheat}
Q = mL
\end{equation}
%
where $L$ is called the latent heat.  There is a latent heat associated with each phase change.
\begin{itemize}
\item {\bf Latent heat of fusion, $L_f$}, refers to the heat associated with the liquid/solid phase change.
\item {\bf Latent heat of vaporization, $L_v$}, refers to the heat associated with the vapor/liquid phase change.
\end{itemize}


\bigskip
For example, given one ice cube ($5\unit{g}$) at $-10^\circ\,{\rm C}$, which is to
be warmed to $30^\circ\,{\rm C}$, we must {\bf warm}\/ the ice to the melting point,
{\bf melt}\/ it, and then {\bf warm}\/ the {\it melted ice} some more.  (There are three separate terms.)  The heat required to do so is
\begin{eqnarray*}
Q & \!\! = \!\! & m c_{\rm ice} \, \Delta T + m L_f + m c_{\rm water} \, \Delta T \\
Q & \!\! = \!\! & (5\,{\rm g})(c_{\rm ice})[(0^\circ\,{\rm C}) - (-10^\circ\,{\rm C})]
      + (5\,{\rm g})L_f
      \\ &&
      + (5\,{\rm g})(c_{\rm water})[(30^\circ\,{\rm C}) - (0^\circ\,{\rm C})]
\end{eqnarray*}
%
This expression gives the amount of heat required regardless of how the ice is
warmed.  The ice may be warmed by placing it into soda.  This equation ignores
the (significant) heat lost by the soda.

\bigskip
Using the same shot material that you used \htmlref{last week}{s:spheat}, figure out how to measure the latent heat of fusion for water.  Consider the questions from the previous lab as well as the following:
%
\begin{enumerate}
\item Water has a large latent heat.  You can only heat a finite amount of shot.  Does this affect how much ice you should use?
\item When considering $Q_{\rm lost}$ or $Q_{\rm gained}$, you will have to decide which terms ($m\,c\,\Delta T$ and/or $mL$) to include for the materials present.  Base your decision on the heat flow for your experiment.
\begin{enumerate}
\item If an object changes temperature, it will need an $mc\Delta T$ term.
\item If an object changes phase, it will need an $mL$ term.
\item Although the mass doesn't change, any $mc\Delta T$ term used can only be applied to one phase at a time: ice and water have \underline{different} specific heat values.  (Notice that the equation above has one $mc\Delta T$ for warming the ice, and another for warming the {\sl melted ice}.
\end{enumerate}
\item Recall \hyperref{this question}{Question~}{}{l:phase} (in \hyperref{the first lab)}{Lab~}{).}{s:thermal}
\begin{enumerate}
\item If your ice is not melting, do you know what temperature it is?  Will the shot need to warm it before melting it?
\item If your ice *is* melting, then you must be aware that when you find the mass for the $mL$ term, this $m$ refers to the mass of \latex{``}\html{"}ice which melts due to the heat of the shot," not due to melting by the air.
\item It is possible to set up the experiment so that you are not adding water with your ice and you are still able to measure the mass of your ice.
\item[HINT 1:]  Measure the mass of the ice indirectly.
\item[HINT 2:]  No mass is lost during the experiment.
\end{enumerate}
\item You will need to be able to stir the ice and the shot together.  This will be difficult unless there is some liquid present initially.
\end{enumerate}

%\addcontentsline{toc}{section}{Presentations for Labs \protect{\ref{s:spheat}} and \protect{\ref{s:laheat}}}






\onecolumn%========================================

\section[Ohm's Law]{Ohm's Law}
\revised{Feb 21, 1997}
\label{s:Ohm}

\subsection{Introduction}

Most materials are either electrical conductors or insulators.  An insulator blocks or resists the motion of electrons, that is, the current I (amperes) through the material is very low (essentially zero) and its resistance R (ohms $\ohm$) is very high.  On the other hand, good electrical conductors have a very low resistance.  Georg Ohm (1787-1854) discovered that the potential difference or voltage drop V applied across a conductor is proportional to the current I through the conductor, with the ratio of V/I equal to the resistance R of the conductor.  This empirical law is named Ohm's Law.  Resistors are devices which follow Ohm's Law and have a nearly constant resistance (an ``ohmic'' device).   This resistance value may range from very low to very high for different resistors.   Some electrical devices (lamps, diodes, transistors) however do not have a constant resistance, their resistance changes dramatically as the voltage and/or the current changes.   These devices are called ``nonohmic.''   The resistance of these devices is also calculated using Ohm's Law but from the slope of the tangent line ($\Delta V / \Delta I$), at a particular V and I.

There must be a potential difference or voltage across a resistor in order for current to flow through the resistor.  The voltage can be supplied by an energy source like a power supply or a battery.  There must also be a completed pathway from the potential source through the circuit and back to the source in order for the current to flow through the resistor.

A multimeter can be used to measure the values of the voltage, current and resistance.  It can be used as a voltmeter by connecting the two lead wires across (in parallel with) a particular device.  This gives a measurement of the voltage drop across that particular device.  The voltmeter has a very high internal resistance so that the meter uses or draws very little current.  An ideal voltmeter has an infinite internal resistance.  The multimeter can be used as an ammeter to measure the current by placing the meter in series with that part of the circuit for which the current is to be measured.  The ammeter has a very low resistance so that there is only a very small voltage drop across the meter.  An ideal ammeter has a zero internal resistance.  The multimeter can also be used as an ohmmeter to measure the resistance of a resistor.  For this measurement the resistor should be removed from the circuit and the two lead wires connected across the resistor.  The meter itself supplies a standard voltage in this measurement so there should never be any other currents passing through the conductor.  One can check the zero of the ohmmeter by connecting the two lead wires together, this is called a short.  One should be careful to select the correct meter setting, the correct range and polarity to avoid damaging the meter.  One should start out on the highest range setting first.   Changing the range setting just changes the resistance within the meter, and changing the resistance changes the current passing through the meter.  The meter can be damaged if too much current passes through it.


\subsection{Experimental Objectives}

In this experiment, you will investigate Ohm's Law with ``ohmic'' resistors, using a standard resistors.  You should also become familiar with the use and operation of the different functions of a multimeter.

\subsection{Pre-Lab Work}

\begin{lablist}
\item Define the terms:  current, voltage, and resistance.
\item Show a picture of three resistors connected in series.
\item Define a short circuit, show a picture of a resistor shorted with a wire.
\item Sketch the expected graph of voltage versus current for an ohmic resistor.  What should the y-intercept be?   What does the slope represent?
\end{lablist}

\subsection{Experimental Procedure}

\begin{lablist}
\item Measure the resistance of three different resistors, with an ohmmeter.
\item Connect a DC voltage supply, a milli-ammeter, and the three resistors in series.
	Does it matter where in the circuit the ammeter is placed?
\item Have the instructor check your circuit, before you turn on the power supply.
\item Set the voltage supply at 1 volt.  Measure the current through the resistors and voltage across each of the resistors.  Also measure the voltage across each one of the resistors and the ammeter. Repeat these measurements for about 8 different  points varying the power supply voltage between 0 - 15 volts.
\end{lablist}

\subsection{Analysis}

\begin{lablist}
\item Find the functional relationship between the voltage across each resistor and the current through the resistors.    Plot the voltage (volts) versus the current (amps),  place all three resistors on the same graph  (different series).
\item Find the functional relationship between the three voltages across each of the  series resistors  and the voltage supplied by the source.
\item Carry out a linear regression (on Excel) for all three of the resistors used.  Quote the provided slope and intercept values and their uncertainties.  Also quote the $R^2$ value, and the p-values for the slope and intercept.
\item Discuss the accuracy and precision of the experiment.
\item Discuss any random and systematic errors.
\item Discuss whether connecting the voltmeter in the two different ways  (the second way includes the voltage across the ammeter)  makes any difference.
\end{lablist}

\subsection{Questions}

\begin{enumerate}
\item Why is it desirable for an ammeter to have a low resistance, while a good voltmeter should have a very high resistance?
\item Why can the resistances of the wires in the circuit be ignored?
\end{enumerate}













\onecolumn%========================================

\section{Ohm's Law for Ohmic and Non-ohmic Materials}
\revised{Oct, 2003}
\label{s:nonOhmic}

\subsection{Introduction}

Most materials are either electrical conductors or insulators.  An insulator blocks or resists the motion of electrons, that is, the current I (amperes) through the material is very low (essentially zero) and its resistance R (ohms $\ohm$) is very high.  On the other hand, good electrical conductors have a very low resistance.  Georg Ohm (1787-1854) discovered that the potential difference or voltage drop V applied across a conductor is proportional to the current I through the conductor, with the ratio of V/I equal to the resistance R of the conductor.  This empirical law is named Ohm's Law.  Resistors are devices which follow Ohm's Law and have a nearly constant resistance (an ``ohmic'' device).   This resistance value may range from very low to very high for different resistors.   Some electrical devices (lamps, diodes, transistors) however do not have a constant resistance, their resistance changes dramatically as the voltage and/or the current changes.   These  devices are called ``nonohmic.''   The resistance of these devices is also calculated using Ohm's Law  but from the slope of the tangent line ($\Delta V / \Delta I$), at a particular V and I.

There must be a potential difference or voltage across a resistor in order for current to flow through the resistor.  The voltage can be supplied by an energy source like a power supply or a battery.  There must also be a completed pathway from the potential source through the circuit and back to the source in order for the current to flow through the resistor.

A multimeter can be used to measure the values of the voltage, current and resistance.  It can be used as a voltmeter by connecting the two lead wires across (in parallel with) a particular device.  This gives a measurement of the voltage drop across that particular device.  The voltmeter has a very high internal resistance so that the meter uses or draws very little current.  An ideal voltmeter has an infinite internal resistance.  The multimeter can be used as an ammeter to measure the current by placing the meter in series with that part of the circuit for which the current is to be measured.  The ammeter has a very low resistance so that there is only a very small voltage drop across the meter.  An ideal ammeter has a zero internal resistance.  The multimeter can also be used as an ohmmeter to measure the resistance of a resistor.  For this measurement the resistor should be removed from the circuit and the two lead wires connected across the resistor.  The meter itself supplies a standard voltage in this measurement so there should never be any other currents passing through the conductor.  One can check the zero of the ohmmeter by connecting the two lead wires together, this is called a short.  One should be careful to select the correct meter setting, the correct range and polarity to avoid damaging the meter.  One should start out on the highest range setting first.   Changing the range setting just changes the resistance within the meter, and changing the resistance changes the current passing through the meter.  The meter can be damaged if too much current passes through it.


\subsection{Experimental Objectives}

In this experiment, you will investigate Ohm's Law with ``ohmic'' and ``nonohmic'' resistors, using a standard resistor and a small incandescent light bulb and a diode.  You should also become familiar with the use and operation of the different functions of a multimeter.

\subsection{Pre-Lab Work}

\begin{lablist}
\item Define the terms:  current, voltage, and resistance.
\item Sketch the expected graph of voltage versus current for an ohmic resistor. What should the y-intercept be?   What does the slope represent?
\end{lablist}

\subsection{Experimental Procedure}

\begin{lablist}
\item Measure the resistance of two different resistors, a mini-lamp, and a diode, with an ohmmeter. For the diode measure the resistance with both polarities.
\item Connect a DC voltage supply, a milliammeter, and one of the resistors in series.
\item Have the instructor check your circuit before you turn on the power supply.
\item Measure the current through the resistor and voltage across the resistor.  Repeat these measurements for about 7 different points varying the power supply voltage between 0 - 3 volts.
\item Replace the resistor with the mini-lamp in the circuit and repeat the above steps.
\item Replace the resistor with the diode in the circuit and repeat the above steps.
\end{lablist}


\subsection{Analysis}

\begin{lablist}
\item Find the functional relationship between the voltage across each resistor and the current through the resistors.
\item Make graphs of the voltage versus the current.
\item Make graphs of the resistance versus the current.
\item Discuss how Ohm's Law applies to the two different types of resistors used.
\item Discuss the accuracy and precision of the experiment (for only the first constant R, resistor).
\end{lablist}


\subsection{Questions}

\begin{enumerate}
\item Why is it desirable for an ammeter to have a low resistance, while a good voltmeter should have a very high resistance?
\item Why can the resistances of the wires in the circuit be ignored?
\item Why is the resistance in these resistors (lamp and diode) not constant?
\end{enumerate}



\onecolumn%========================================

\section{Non-Ohmic Behavior of Diodes and Lamps}
\revised{Aug, 2008}
\label{s:nonOhmicDiode}

\subsection{Purpose}

The object of this laboratory exercise is to investigate the special properties of a normal silicon diode The IV characteristic curve of a diode will be determined and graphed.   Ohm's Law will also be investigated for a small mini-lamp bulb.

\subsection{Introduction}

A diode is a combination between a P-type (silicon doped with boron, with extra holes or positive centers) and a N-type (silicon doped with arsenic, with 5 valence electrons) semiconductor.  A joining of an N-type and a P-type is called a P-N junction, or just a diode. This device is nonohmic because its resistance is not linear with increasing current. This relation is determined from a plot of the current (I) versus the voltage (V) and is characteristic for each device.   The physical processes that occur within a diode are rather complex, but the diode action can be simplified with three approximations.  The first is to consider the diode as a switch, when the diode is forward biased the switch is open, and when it is reversed biased the switch is closed.  The second approximation is that there is a forward junction potential barrier that must be overcome in order to operate the diode. This is about 0.6 volts.  The third approximation adds a small forward biasing resistance in series with the device or an infinite resistance if reversed biased.  Look for diodes in your text book.

If the voltage across the diode is zero, still some of the electrons of the N-type are attracted to the P-type side of the junction. This sets up an equilibrium region (zero net charge).  As the voltage across the diode increases, the electric field through the silicon also increases. More of the electrons in the N-type are attracted to the P-type, setting up a current through the silicon.  This also changes the size of the equilibrium region,  it decreases with increasing Voltage.  The size of this region is related to the resistance of the semiconductor.  Therefore the resistance of the diode decreases with increasing voltage across the diode. When the diode is reverse biased then the size of the equilibrium region increases and the resistance increases.  Therefore the diode conducts with one polarity (the ``ON'' state)  and does not conduct with the reverse polarity (the ``OFF'' state).

The mathematical model that is used for this device can be given as:
\[ I = I_s \left( e^{V/V_T}  -1\right) \]
where I is the current through the diode, V is the voltage across the diode, $V_T$ is a relative comparison voltage, usually given as 26 mV at room temperature, and $I_s$ is called the saturation current and it is a very small negative current ($\approx -2 \times 10^{-9}$ amps).


\subsection{Procedure}

\begin{enumerate}
\item Use a multimeter (ohmmeter) to test the polarity of the diode before it is placed in the circuit, do this for both forward and reversed biasing. One polarity gives a high resistance and the other gives a lower resistance.
\item With the given diode, map out the entire IV characteristic curve.  Use about a 2.5 volts max ramp function for the forward biasing, from a function generator, and set the frequency at about 2 Hz.  Use a $1000\ohm$ resistor in series with the diode to limit the current from the power supply.  Take the data using the Pasco interface and the program ``Data Studio.''   Measure the voltage across the diode in channel B and the voltage across the resistor in channel A.  So the current through the diode can be determined from the data in channel A.  A 2 hertz frequency is 0.5 seconds for the period, and so you should collect about 500 measurements per second to give 250 data points for one period.
\item Repeat this process for a mini lamp bulb.
\end{enumerate}

\subsection{Analysis}

\begin{enumerate}
\item Determine the diode's resistance as a function of the voltage across the diode.  The resistance is the slope of the tangent line at each point of the curve,  or it can be stated that the resistance is the derivative of the curved data.  ``Data Studio'' will plot the derivative of the data.
\item Test and compare your data with the mathematical model equation given above.  Choose a couple of different data length regions to compare: i.e., diode voltage 0 to .4 V,  .45 to .5 V,  .5 to .6 V, and .6 to .7 V.  Do this work in Excel.
\item There were 2 constants given in the model equation.  Would the fit be better with different constants?
\item Show a ln(I) versus V plot.
\item Show a  ln(I) versus ln(V) plot.
\item Discuss the results of these graphs.
\item Experimentally determine the diode's junction potential (the turn ``ON'' potential), from these graphs, also including the resistance vs  V graph.
\item Repeat parts of this analysis for the mini-lamp.
\end{enumerate}

\subsection{Questions}

\begin{enumerate}
\item Briefly explain the theory and operation of a P-N junction.
\item Compare your IV curve with the one in the textbook.
\item Comment on how much the diode's resistance changes.
\item Comment on how a diode can be used as a switch.
\item Comment on the accuracy of the three approximations stated in the introduction.
\item Comment on the differences between the mini-lamp, and the diode.
\end{enumerate}











\onecolumn%========================================

\section{RC Circuits}
\revised{Feb, 2007}
\label{s:RCeasy}

\subsection{Objectives}
\begin{lablist}
\item To investigate the nature of a capacitor in a D.C. circuit.
\item To determine the time constant of a D.C. circuit containing resistance and capacitance.
\end{lablist}

\subsection{Procedure}
\begin{lablist}
\item Measure the capacitance and resistance of these two elements that you will use in your circuit, with a multimeter.  Be sure to do this BEFORE constructing the circuit.  Compute the value of $R \times C$.  Record this value.
\item Using a function generator, take a data set  (through the Pasco 750 interface) of the square wave input voltage (by itself) versus time.   First, set the max. voltage at about 5 volts.   Adjust the dc offset to give the lower voltage at zero volts and set the period of the wave to about 10 time constants
\item Set up the resistor and the capacitor in series with the square wave generator.  Some capacitors must be connected in a circuit with care being taken to notice the polarity of the capacitor.  Be sure to check to see if there are $+$ and $-$ signs on the capacitor.
\item Measure both the voltage across the capacitor and the voltage across the resistor, through the Pasco 750 interface.  Set the number of data points taken per second to achieve  75 to 100 data points in the discharging part of the cycle.  If  C=14mF and R=3kW,  then a period of about 1 second would be good and 500 data points per second would also be good.   Let it run for 2 full periods, so that you can select the best starting point.  Make a plot of this data from ``Data Studio.''
\item The function generator also has an internal resistance.  This resistance is also in series with the circuit.  So, what is the total resistance of the circuit?
\item Transfer both data sets to Excel.
\end{lablist}

\subsection{Analysis}

\begin{lablist}
\item In ``Data Studio'' estimate the time constant from the graph both for the charging and from the discharging cycles.       Does charging and discharging give the same time constant?     Plot $V_R$  vs time and $V_C$  vs time.
\item Kirchhoff's loop Rule for a series circuit states that the voltages are additive, and in this case yields $V_{\rm input} = V_R + V_C$.  In Excel, make a column for $(V_R + V_C)$, to check this rule.
\item In Excel, for each of the data points in the discharging part of the cycle (for $V_C$), determine the current in the circuit.  The R and C are connected in series, so the current in the resistor and in the capacitor are exactly equal.  Plot the current in the circuit vs. time.  V= IR,  so $I = V_R / R$.
\item This graph should show an exponentially decreasing function.  The curve should be expressible as:
    \[ I(t) =  I_0 \, e^{-t/\tau} \]
    where $\tau$ is the length of time required for the current to fall to 1/e (about 37\%) of its initial value.  This length of time is called the TIME CONSTANT of the circuit, and depends on the effective resistance and capacitance in the circuit.
    \begin{itemize}
    \item To determine the time constant of the circuit, the equation must first be linearized.  Taking the natural log of both sides of the equation yields:
					\[ \ln(I)  =  \ln(I_0)  -  \frac{t}{\tau} \]
    \item Thus, plotting the ln(I) vs. time should yield a straight line.  Regression analysis will determine the time constant.  Interpret the slope and intercept values.  Quote the uncertainties and the p-values.  Determine the time constant of the circuit and compare it with the value $R\times C$ (ohmmeter results).  Discuss the significance of this value.
    \end{itemize}
\item Using the Time Constant determined from your regression analysis and the value of the resistance you determined with the ohmmeter and the internal resistance of the function generator, calculate the true capacitance of your capacitor.  Use this value for the capacitance in all subsequent calculations and analysis.
\item Discuss the accuracy and the precision.  Discuss any systematic errors.
\end{lablist}









\onecolumn%========================================

\section{Discharging a Capacitor through a Lamp}
\revised{Oct, 2006}
\label{s:RClamp}

Normally the time constant of an RC circuit is much less than one second.  If we are to light a (6 V) lamp bulb the time constant must be much longer.   This means that the capacitance value must be very large, and partly because the resistance of the lamp is rather small.  The charge stored on a large capacitor can be very dangerous.  So be careful.   One coulomb is enough to stop a person's heart.

As current passes through the bulb,  the bulb heats up.  This changes the resistance of the bulb,  which changes the time constant of the circuit.  This means that the charging or discharging results are not pure exponential functions, like were observed before.   This laboratory will investigate the qualitative and the quantitative nature of  discharging a capacitor through a mini-lamp.

\subsection{Procedure}

\begin{enumerate}
\item Set up a RC series circuit,  with a large capacitor (0.1 Farads) and about a 30 ohm resistor.   Through Data Studio, collect a Vc data set.  Have at least 50 to 100 data points in the exponential part of the curve.  Determine the effective C value from the RC time constant value. Use a semilog scale graph to get the time constant.  Measure the R with a multimeter.   Use a switch to change from charging to discharging.  Use this value of C in the next parts of the experiment.   Set the voltage of the power supply at about 6 volts.
\item Take the 30 ohm resistor out of the circuit and put in the mini-lamp.   Also place in the series circuit a 1 ohm resistor, so that the current in the circuit can be determined.
\item Place two lamps in series,  how much longer does the lamp stay lit?   Explain.
\item Through Data Studio, correct a data set for: $V_C$, $V_R$, and $V_{\rm lamp}$.  Transfer these data sets to Excel.   Show these graphs, with the first of the data points shown at t=0.
\item The resistance of the lamp bulb can be calculated as a function of the time using Ohm's Law at each point.   First though, determine the current through the bulb, using Ohm's Law and the data for the voltage across the  1 ohm resistor.
\item Plot the resistance of the lamp ($R_{\rm lamp}$)  versus the current  (I).
\item What is the best model for this data set?   Linear, exponential?    Try a few different types of fit functions.   Try a  log - log plot.
\item How does the resistance of the bulb change with current?   Discuss this point.  Discuss the results of the fit model equations, slopes, errors, p-values, $R^2$, etc.
\item Discuss each of the data sets and compare to a true exponential RC decay.  Discuss the differences.   Which models work the best?
\item Calculate the total Q stored in the set of capacitors.
\item Calculate the total energy stored in the set of capacitors.
\end{enumerate}




\onecolumn%========================================

\section[Wheatstone Bridge]{The Wheatstone Bridge: A Measurement of Small Resistances and the Resistivity}
\revised{Sept, 2004}
\label{s:MagField}

These instructions also refer to the details of the \underline{Slide-Wire Wheatstone Bridge}, as written in the \underline{Selective Experiments in Physics}, which are added as an appendix to these pages.

\subsection{Procedure}

\begin{enumerate}\itemsep .02in
\item Set up the apparatus like in Figure 4.  Use a 1 volt power source.  Consider $R_2$ the unknown resistor.
\item For the resistive coils supplied (6), calculate the resistances for each with Eq. (6).  Use the given resistivities in Table 1.  Look up the diameters for these wires given the gauge (Brown and Sharpe  (B \& S)).   Look these up in the Handbook of Physics and Chemistry (15-33)  as well as the ohms/cm  values for each.
\item Measure their resistances with the multimeter.
\item Set the resistor $R_1$ near the value of $R_2$,  so that the slide wire should be close to 50 cm.
\item Between measurements use the power supply switch (off) so that current through the wires does not heat them.   The resistivity changes with temperature.
\item Use a resistor box (5k$\ohm$) in series with the galvanometer if large deflections are observed.
\item Set the slide wire near 50 cm.  Press the left button on the galvanometer.   Which direction does the needle move?  Move the slide wire until the galvanometer is balanced.  Use the right button on the galvanometer for the final tests.  When the galvanometer reads zero the system is balanced and the same amount of current passes through both sides of the Wheatstone Bridge.
\item Record the slide wire distances (a \& b).
\item Calculate the resistance of the coil.
\item Change the value of $R_1$ by 1 ohm and repeat.
\item Repeat this process for each of then given coils
\item What is the precision of your measurements?  How much variation is there is the slide-wire measurement?  (1 mm or is it closer to 5 mm?)   Measure this variation.
\end{enumerate}


\subsection{Analysis and Discussion}

\begin{enumerate} \itemsep .02in
\item For the coils (same metal and same diameter) make a graph of the resistance versus the length of the wire. Interpret  the  regression analysis (slopes and intercepts, and errors and p-values).   Compare your results to the theory (eq. 6).
\item When the diameter was changed, do the results still follow eq. 6?
\item When the material was changed, do the results still follow eq. 6?
\item Calculate \% differences between your results and eq 6, for each coil.
\item Do a propagation of error calculation (for R) and compare this to the accuracy just calculated.
\item Look for and discuss sources of systematic errors.
\item Look for and discuss sources of random errors.
\item Also answer questions: 1,6,7 from the other sheets.
\end{enumerate}


%========================================

\section{Magnetic Field of a Magnet}
\revised{Oct, 2008}
\label{s:Magnet}

\subsection{Introduction}

The source of all magnetic fields is moving charges.  A coil of wire with a current in it will produce a magnetic field near the wire.  A permanent ferro-magnet produces a magnetic field through the alignment of atomic dipole moments in regions called domains.  All matter can be considered to consist of microscopic magnets.   Under normal temperature conditions the thermal motions tend to randomize these microscopic magnets.  But when matter is placed into an external magnetic field these microscopic magnets tend to align with the external B field.  The elements iron, nickel, cobalt have a strong tendency to align into domains and to have these domains align with an external B field.  Alnico (Aluminum, nickel, cobalt) is one of many composite materials, which is used as a core in making electromagnets.

	The degree of magnetization $M$ of a sample material depends on the applied B field, on the temperature, and on the magnetic susceptibility ($\chi_m$).   The magnetic susceptibility is not a true constant for a sample,  but this parameter also depends on the temperature, and on the saturation magnetization $M_s$.  For paramagnetic materials $\chi_m$ is about 10-5.  For ferromagnetic materials $\chi_m$ can be from about 2 to 150, in most room temperature situations.

A bar magnet is the simplest model of a magnetic dipole.  Magnetic monopoles have not been discovered.  A magnetic dipole has 2 poles called the North and the South poles.  Let's say that each pole has a strength of $m$ and that the poles are separated by a distance\footnote{Note that $2l$ is not the thickness of the magnet but the distance between the force centers; this is  approximately 0.9 times the geometric length.} of $2l$.   The dipole moment is then defined by $p = (m)(2l)$.  The magnetic field at a distance $x$ (see the figure provided) was first shown by Charles Coulomb and can be expressed by
\[ B = \frac{2 p x }{ (x^2 - l^2)^2}. \]
An expansion of this expression gives,
\[ B = \frac{2p}{x^3} \left( 1 + \frac{2l^2}{x^2} + \frac{3l^4}{x^4} + \ldots \right).\]

\subsection{PART A}

\subsubsection{Purpose}
To study the magnetic field of a dipole magnet as a function of the distance from the magnet.

\subsubsection{Equipment}
\begin{lablist}
\item One neodymium magnet
\item A Hall Effect Probe (B field measurement in Gauss)  1 gauss = $10^{-4}$  tesla
\item The Pasco sensor has an uncertainty of about 10 gauss on the 1X range.
\item A horizontal positioning apparatus is precise to about 0.1 mm
\end{lablist}

\subsubsection{Procedure}
\begin{lablist}
\item Measure the thickness of the magnet. Use calipers.
\item Set up the apparatus.   Connect the probe to the Pasco interface and Data Studio.
\item Zero  (tare) the  B probe.  (without the magnet)
\item Set-up the magnet  as close as possible to the B probe (attached to the position apparatus, set at zero).   There will be some zero point error with this.
\item Measure B and a (from the picture).  Repeat this process, increasing the distance  for each trial by about 2 mm, until the total B field drops below 50gauss.
\end{lablist}

\subsubsection{Analysis}
\begin{lablist}
\item Using the picture determine the distance x for each data point.
\item Plot B versus x.
\item Plot log(B) versus log(x).
\item A significant zero point error in the placement of the sensor can cause the initial few data points to not fit a straight line.  This zero point error might be up to 1.5 mm.  In steps of 0.3 mm (up to 1.5 mm), add this amount to the values of x.  Replot the data.  Which of these data sets gives the best straight line?  Also look at the residuals graph.
\item From this best log-log plot, list and explain the meaning of the slope and intercept values.  Also quote the relative uncertainties of these 2 parameters.
\item Plot $B$ versus $x^3$.  Use the best corrected  x values.  Does a straight line fit this data?  Look at the residual plot, and the $R^2$ value.   Explain the meaning of the slope and intercept values.    If there is a y-intercept, then the x values can be adjusted; as you just did previously for a zero point error;  to correct this systematic error.
\item Re-plot the linear graph, if this zero point error is observed.
\item Discuss the precision and accuracy of this data compared to this linear model.
\item Use the more general equation  $\displaystyle B= \frac{2px}{(x^2 - l^2)^2}$,  as a fit function to your data.          Plot this equation (as a second series) on the graph of B versus x.  Since the form of the equation is different, the value of $p$ may change.  Adjust the value of $p$ until the fit looks reasonable.  Again use the residuals graph to help choose the best fit.  The value of $l$ can also be adjusted to obtain a better fit.  Show that the units in this equation for $B$ match the units of Tesla.
\item There are three different models used here in this exercise.  Compare the process,  the results, the uncertainties, for these three different models. Compare the fit in each case to the data.   Compare $R^2$ values, p-values, uncertainties residual plots, etc.
\item This problem is somewhat difficult because the precision of $l$ and $p$, and because of the zero point error.  It is usually not correct to adjust the data to obtain a better fit.  One needs to be upfront with the reporting of these kinds of changes.  If we consider these parameters as unknowns then it is not necessarily bad science to make slight changes.   For example, if we were to carry out a complex nonlinear fitting routine with three unknown parameters, we could just let the software  choose the best values for the best fit.   In this case, if we take the approach that the adjustments we take are small and make the changes carefully then this is similar to the nonlinear fitting process.
\item What is your best guess for dipole moment for this neodymium magnet?
\item What is your best guess for its uncertainty?
\end{lablist}


\subsection{PART B}

\subsubsection{Purpose}
To measure the magnetic field of a solenoid, at its edge and to measure the magnetic field at the edge of the solenoid with a ferro-core in place.  Then  to determine the magnetic susceptibility of the ferro-core.

\subsubsection{Equipment}
\begin{lablist}
\item A coil (with known N)
\item A ferro-core.
\item The Hall Probe (B field)
\item (10 amp) Power Supply
\end{lablist}

\subsubsection{Procedure}
\begin{lablist}
\item Set-up the Hall Probe so that it is at the very edge of the solenoid. (No core) What is the value of N?   Measure the solenoid length.
\item Measure the B field for 0,1,2,3,4,5 amp.
\item Place the core in the solenoid.
\item Measure the B field for 0,1,2,3,4,5 amp.
\end{lablist}

\subsubsection{Analysis}
\begin{lablist}
\item Make graphs of  B versus I (both sets of data on one graph).
\item Interpret the slope and intercept values of both data sets.
\item Discuss the experimental precision of this experiment.
\item Use Ampere's Law to derive the equation for the B-field inside a solenoid.
\item At the coil edge the B-field is about $\txtfrac{1}{2}$ of the inside value.   This $\txtfrac{1}{2}$ is a geometric factor.  For your data set, determine this geometric factor, and its precision.
\item Determine the magnetic susceptibility for this particular ferro-core, and its precision.
\item Does this value seem reasonable?  Explain.
\end{lablist}

\onecolumn%========================================

\section{The Earth's Three-Dimensional Magnetic Field}
\revised{Nov, 2006}
\label{s:3DMagField}

\subsection{Introduction}

The Earth's magnetic field is usually modeled like a magnetic dipole, near the center of the Earth.   At the surface of the Earth, the magnetic field is in general a three-dimensional vector.  Set-up a component system with three perpendicular axes;  first is a radial component towards the center of the Earth, second is a latitudinal component, directed towards the North magnetic pole, and third is a longitudinal component, directed perpendicular to the other two, this will be tangential to a circle around the Earth's magnetic axis.  If the Earth's field was exactly like a dipole then this longitudinal component would be zero.   In reality the Earth's magnetic field is not a true dipole and also because there are local concentrations of magnetic materials there is a nonzero longitudinal component.  This component can be measured,  but we will assume that it is a small component and it will not be measured today in this experiment.	The angle between the magnetic field and the surface of the Earth is called the dip inclination angle.  This angle can be measured with a dip needle magnet.  This dip needle must be aligned in a N-S direction,  so that the longitudinal component is not a part of the orientation.

\subsection{Purpose}

To measure the components of the Earth's Magnetic Field.

\subsection{Apparatus}

Dip needle magnet, a compass,  the flip-coil apparatus, Pasco 750 interface with voltage probe.

The flip-coil consists of a coil (about N=1000) which is spring loaded so that when the coil is released the coil rotates through an angle of 180 degrees in a few tenths of a second.  The Earth's magnetic field is a vector, which can point through the flip-coil.  When the magnetic flux through the coil is changed, a voltage is induced in the coil.  The coil has a certain resistance, so there is a certain current induced into the coil as well.  The induced voltage can be measured as a function of the time directly through the Pasco 750 interface.  This voltage will not be a constant over the time interval.


\subsection{Procedure}

\begin{lablist}
\item Align the dip needle magnet N-S, and record the dip inclination angle.
\item Align the flip-coil in the N-S direction,  so that the horizontal component of the Earth's field does not pass through the coil,  only the vertical component is captured.   The axis of rotation will be parallel to the B-field component that is not to be measured.  Tape a piece of paper to the table and draw the N-S line on the whole length of this paper.  Tape the large protractor down also.
\item Does the induced current change direction at the midpoint of the rotation, or does the current direction remain the same during the entire rotation?  Test the right-hand-rule with the apparatus and as the coil rotates.
\item Take a data set, and graph the Induced Voltage versus the time.
\item Record the mean value of the voltages (just within the peak), and determine the total time of the event.  Data Studio also will calculate the area under the curve.
\item Repeat this 4 times.
\item Derive an equation for the total change in flux for this process.
\item Calculate the mean area of the inductor coil from the inside and outside diameters given on the apparatus.
\item Calculate the vertical component of the Earth's magnetic field.
\item Repeat this process 8-10 more times but with the apparatus rotated (horizontally) by $\pm$90 degrees each rotation direction. You can use step sizes of 5 degrees (near 0) and step sizes of  10 or 20 degrees in other places.
\item How sensitive is the value of this component to being exactly aligned?  In the future,  will great care need to be  used in this alignment?
\item Make a plot of this B-field component versus the angle.
\item Is there a max or min near the zero degree mark?   And why?
\item From these trials, estimate the experimental uncertainty, of the induced voltage and of the component of the magnetic field strength.
\item[]
\item Rotate the apparatus (on its side) so that when flipped only the horizontal component of the Earth's field passes through the coil.
\item Repeat the same method as before, but now the horizontal component of the Earth's field can be determined, and its experimental uncertainty.
\item Since this horizontal component is the smaller component, the alignment might be more important, so more care with this setup.  And take 6-8 trials where the angle is adjusted.
\item[]
\item From these two perpendicular components determine the total magnetic field strength,  the dip inclination angle, and their experimental uncertainties.
\item[]
\item Compare the two methods for the dip angle measurement.
\item How does the dip angle compare to our Earth latitude?
\item What would be the result if the coil were turned through 360 degrees, instead of 180 degrees?
\item Discuss the major sources of experimental error in this experiment.
\item How important is the flip-coil orientation?
\end{lablist}

\twocolumn[%========================================

\section{Magnetic Field of the Earth}
\revised{Spring, 2006}
\label{s:MagField}

In the following lab, we will be investigating two techniques for measuring the magnetic field of the Earth.  Please compare (\%-diff) the values and the uncertainties.  Based on the measurements, what is your best guess as to the value of the magnetic field of the Earth in Abilene, magnitude and direction?  Neither experiment will work well if your magnetic field swamps out the Earth's magnetic field.  \\]

\subsection{Tangent Galvanometer}
\label{s:Bfields}

Consisting of a vertical coil of wire with a compass at its center,
the tangent galvanometer uses the vector nature of magnetic fields to
calculate the local magnitude of the Earth's magnetic field.
%(A simple compass will tell us the direction.)
As current begins to flow through
the coil, a magnetic field is generated whose direction can be found from
the right hand rule and whose magnitude can be calculated from
%
\begin{equation}
\label{eq:coilB}
B_c = \frac{\mu_0 I N}{2 r}
\end{equation}
%
where $B_c$ is the coil's magnetic field strength, $I$ is the
current (in amps), $\mu_0 = 4\pi\ten{-7}\unitfrac{N}{A^2}$, $N$ is the
number of turns of the wire in the coil, and $r$ is the radius of the coil.

Using components, resultants, magnitudes
and directions, devise a technique to combine the controllable
magnetic field due to the coil with the uncontrollable, but fixed, magnetic
field of the earth in such a way as to calculate the magnitude of the
Earth's magnetic field.  Several hints are given below.

Carry out this experiment with five turns of wire once for each direction
of the coil's field and repeat for ten turns of wire.
Determine your average value for the strength of
the Earth's magnetic field in Abilene.
%
\begin{enumerate}
\item If there are three magnetic fields present, along which does a compass point?
\item Can you control the direction of any magnetic field present?
\item Can you control (and know the value of) the magnitude of any magnetic field present?
\item Can you relate components to resultants and vice versa?
\item Components {\bf must} be perpendicular to each other.
\end{enumerate}



%========================================

\subsection{Magnetic Field of the Earth}
\revised{Spring, 2006}
\label{s:BofE}

A small bar magnet suspended in a magnetic field will align itself
with that field.  A small displacement will set it
oscillating in {\bf simple harmonic motion}\/ about the equilibrium position.
The motion is similar to a pendulum's oscillation (in fact, this is a torsion pendulum)
and we find that the total mechanical energy is conserved.
The initial PE is $-\mu B \cos\theta_{\rm max}$, where $\mu$ is the magnetic moment and $\theta_{\rm max}$
is the initial displacement angle.  The initial KE is zero.
The final PE is $-\mu B$ because $\theta=0$ when the magnet is aligned with the field.
The final rotational KE, written in terms of the period, $\frac{1}{2} I \theta_{\rm max}^2 (2\pi/T)^2$.
Conservation of energy relates the field, $B$, to the period:
\begin{equation}
\label{eq:bareB}
B = \left[ \frac{I \, \theta_{\rm max}^2}{2\mu(1-\cos\theta_{\rm max})} \right] \left(\frac{2\pi}{T}\right)^2
\end{equation}

Since measuring the initial displacement angle $\theta_{\rm max}$ is very difficult, we will consider
the square bracket $\left[ \frac{I\, \theta_{\rm max}^2}{2\mu(1-\cos\theta_{\rm max})} \right]$ to be a single, unknown quantity
that does not change.
If we then increase the field by a known amount, $B_c$, so that the new field is $B_E+B_c$,
we can measure a new period, $T_+$.  This gives a new version of Eq.~(\ref{eq:bareB})
%
\begin{equation}
\label{eq:dressedB}
B + B_c = \left[\frac{I\, \theta_{\rm max}^2}{2\mu(1-\cos\theta_{\rm max})}\right] \left( \frac{2\pi}{T_+}\right)^2
\end{equation}
%
If we then decrease the field by the same known amount, $B_c$, so that the new field is $B_E-B_c$,
we can measure a new period, $T_-$.  This gives another version of Eq.~(\ref{eq:bareB}).
Since the magnetic field due to a current coil, $B_c$, can be found with Eq.~(\ref{eq:coilB})
and since $T_+$ and $T_-$ can be measured, it is possible to solve these two equations for
the magnetic field of the Earth $B_E$.

Determine the strength of the earth's magnetic field by measuring as indicated above.
Calculate uncertainties on your results and compare them
with the value from the first technique.

%\addcontentsline{toc}{section}{Peer-Review of Lab \protect{\ref{s:MagField}}}







\onecolumn%========================================

\section{Reflection and Refraction at a Plane Surface}
\revised{Mar, 2005}
\label{s:refraction}

\subsection{Introduction}

	Sir Isaac Newton developed a particle theory of light to explain two commonly observed optical properties of light, reflection and refraction, by using the principles of geometry.  When light is incident onto any surface, some of the light may be reflected off from the surface and some of the light may be transmitted through the surface.  The reflected light follows a simple law (law of reflection), which states that the angle of the incident ray is equal to the angle of the reflected ray.  These angles should be measured with respect to a normal line to the reflecting surface at the point of incidence.  The transmitted light follows a simple law called the law of refraction or Snell's Law, named after Willebrord Snell (1591-1626).  The light rays which are transmitted through the interface and into a new medium may be bent (refracted) at the interface.  This law is stated mathematically by the ratio of the sine of the incident angle, $\theta_i$, to the sine of the refracted angle, $\theta_r$, and where this ratio is a constant for any two given media and for light of a given wavelength.  Again, all angles are measured with respect to the normal to the surface.
%
\begin{center}
\begin{tabular}{lc} \\
LAW OF REFLECTION: & $\displaystyle \theta_{\rm incident} = \theta_{\rm reflected}$ \\ \\
LAW OF REFRACTION: & $\displaystyle \frac{\sin\theta_{\rm incident}}{\sin\theta_{\rm refracted}} = \frac{V_1}{V_2} =  \frac{n_2}{n_1}$ \\ \\
\end{tabular}
\end{center}
%
where $\theta_{\rm incident}$ is the angle of incidence in medium 1, $\theta_{\rm reflected}$ is the angle of reflection in medium 1, $\theta_{\rm refracted}$ is the angle of refraction in medium 2, $V_1$ is the speed of light in medium 1, $V_2$ is the speed of light in medium 2, $n_1$ is the index of refraction of medium 1, and $n_2$ is the index of refraction of medium 2.  The index of refraction is then the ratio of the velocity of light in a vacuum to the velocity of light in the medium: $n = c/V$.  The index of refraction for a vacuum is defined to be 1, and the index of refraction for air is also very nearly 1.

\subsection{Pre-Lab Work}

\begin{lablist}
\item Draw a ray diagram locating the virtual image of an object, where the object is say 3-5 cm in front of a plane mirror.  Use a small arrow as the object.  The image is located by drawing at least two rays from both the front and rear ends of the arrow, and then using the law of reflection on these four rays.
\item Using Snell's Law, derive an equation for the incident critical angle (the refracted angle is 90), in terms of the two indices of refraction for the two mediums.
\item Show that when light strikes a parallel glass plate of length L at an incident angle i1, the emerging ray is found to have the same direction as the incident ray but is found to be displaced laterally by an amount D, where,
    \[ D= L \frac{\sin \left(\theta_i - \theta_r\right)}{cos \theta_r} \]
\end{lablist}

\subsection{Experimental Objectives}

\begin{lablist}
\item Study and experimentally demonstrate the law of reflection from a plane surface.  Be able to describe the virtual image formation using the theorems of geometry and by tracing the paths of the light rays.
\item Determine a relationship between the angle at which two plane mirrors are set and the number of virtual images that are seen.
\item Study and experimentally demonstrate the law of refraction from a plane surface by tracing the paths of the light rays.
\item Experimentally determine the index of refraction and the critical angle for a glass plate.
\end{lablist}


\subsection{Procedure}

\subsubsection{Reflection at a single plane surface}

Place a sheet of paper on top of a piece of cardboard.  Place the red Plexiglas reflection plate in the center of the paper.   One side is slightly beveled, place this side toward you..  Trace the outside edge of the glass plate onto the paper.  Place a pin in the paper 5 cm from the plate, on the side toward you.  This side is now called the object side.  By looking through the glass plate (from the object side) locate the virtual image and place another pin at that location.  Complete the ray diagram, (choose at least two different random incident angles) to locate the image and compare the results with the law of reflection (i.e. measure the angles of incidence and reflection).  Repeat this procedure with another object like a pen cap or write a word or name on the paper and trace the reflection.  Again, locate the image and test the law of reflection at 2 or 3 different points on the image/object.  Use a ruler and a protractor to make your diagram.

\subsubsection{Reflection using a pair of plane mirrors}

Place a ruler equidistant in front of the two hinged mirrors.  Adjust the angle of the mirrors to investigate relationship between the angle between the mirrors) and the number of virtual images, plus the object itself.  That is, at what angle can you see two images, three images, etc?  Draw numerous diagrams for your results.  Use the law of reflection to locate the image positions.  Note that one virtual image can be like an object to form other images.  That is, a mirror can be seen as a virtual mirror. What geometrical shapes are seen for each of the special angles found?  Draw simple sketches for each of the cases observed.  For the pentagon case, draw an accurate scaled diagram, and test the law of reflection at each of the reflecting surfaces.  Determine a relationship (give a formula) between the angle at which two plane mirrors are set and the number of virtual images that are seen.

\subsubsection{Refraction at a plane surface}

Use a laser and a thin square piece of plate glass to map out the refracted light rays at both surfaces of the square plate (like in figure 1) and to test the law of refraction at both surfaces. The incident ray should be at least 45 degrees and the light ray should also exit from the opposite side of the glass plate. Show that your data is consistent with Snell's Law. Use this data to determine the index of refraction of the glass plate. Compare this value to the textbook value.  Experimentally show that the equation for D given before is valid.

\subsubsection{The Critical Angle}

Use the thin triangular prism in this part.  Again map out the refracted rays.   Have the incident ray hit one side about $\txtfrac{1}{4}$ of the way from the apex and with a rather large incident angle.  A refracted ray should be seen exiting from the second surface of the prism.  Now slowly rotate the paper (and the prism) until the refracted ray has an angle of refraction of 90 degrees.   The critical angle is now the incident angle at the second surface of the prism.  Experimentally determine the critical angle for this piece of glass.  Experimentally show total internal reflection by locating the reflected ray from the second surface.  Test the law of reflection for this ray.


\subsection{Questions}

\begin{enumerate}
\item Why does a plane mirror reverse left and right?     Draw a ray diagram.
\item What happens to the number of image(s) if the angle between the two plane mirrors is $90^\circ$?, $30^\circ$? $14^\circ$?
\item What happens to the speed of light, the wavelength and the frequency of the light in a medium with a greater index of refraction?
\item Under what conditions is the angle of refraction greater than the angle of incidence?
\item What can be concluded about the refracted ray and also the reflected ray when the angle of incidence is greater than the critical angle?
\end{enumerate}








%========================================

\twocolumn[\section{Index of Refraction}
\revised{Spring, 2006}
\label{s:index}
%
When observing objects through glass or water, optical illusions distort the image of
any object.  A common example is the apparent bend in an actually straight straw
when a portion of it is underwater.  The effect is also commonly noticed with fish
in a tank.  The effect occurs when light passes from one medium to another and is
called the refraction of light.\\]

The amount of refraction depends on both the material the light passes from
and the material the light passes into.  The refraction is caused by a change in the
speed of light through the material.  The ratio of the speed of light in the vacuum, $c$,
to the speed of light in the material, $v$, controls the amount of refraction.
This ratio is called the index of refraction, $n=c/v$.  The angle used to measure the
refraction effect is $\theta$, the angle from the path of light to a line normal
to the surface.  Notice the labels in the figures.

The equation that relates the amount of refraction $(\theta)$ and the type of material
$(n)$ on either side of the boundary is Snell's Law:\newline
%\hspace*{.9in} $n_{\rm in} \sin\theta_1 = n_2 \sin\theta_2$\hfill\
\[ n_{\rm in} \sin\theta_1 = n_2 \sin\theta_2 \]

To observe and measure the refraction, place a lens on a piece of
cardboard that has been covered with a piece of paper.  Draw the outline of
the lens on the paper and do not move the lens.  Place a pin somewhere near the lens
(See pin 1 in Fig.~\ref{f:square-lens}).  Place a second pin (pin 2) further away in a line
that points towards the lens.

When you look through the lens, pins 1 and 2 will appear to be shifted over.
(See pins $1'$ and $2'$ in Fig.~\ref{f:square-lens}.)
% Place a third pin (pin 3) on the
% side of the lens nearest your eye so that it is in
%
\begin{figure}[bh]
\begin{center}
\begin{picture}(10,10)(0,1)
% Glass
\put(3,3){\line(1,0){5}}
\put(3,3){\line(0,1){5}}
\put(8,8){\line(-1,0){5}}
\put(8,8){\line(0,-1){5}}
% light rays
\put(5,8){\line(-2,3){2}}
\put(5,8){\line(1,-5){1}}
\put(6,3){\line(2,-3){2}}
% normal lines
\multiput(5,6)(0,.6){8}{\line(0,1){.3}}
\multiput(6,1)(0,.6){8}{\line(0,1){.3}}
% Pins
\put(4.5,8.75){\circle*{.2}}
\put(3.5,10.25){\circle*{.2}}
\put(6.5,2.25){\circle*{.2}}
\put(7.5,0.75){\circle*{.2}}
% Apparent pins
\multiput(5.5,3.75)(-1.125,1.6875){5}{\line(-2,3){.5}}
\put(2.5,8.25){\circle{.2}}
\put(1.5,9.75){\circle{.2}}
% label angles
\put(6.1,2){\tiny$\theta_a$}
\put(5.6,4.8){\tiny$\theta_g$}
\put(4.9,5.8){\tiny$\theta_g$}
\put(4.2,9.3){\tiny$\theta_a$}
% label indexes
\put(7,6){\small$n_g$}
\put(7,8.5){\small$n_a$}
\put(3.5,2.2){\small$n_a$}
% label pins
\put(4.4,8.2){\tiny $1$}
\put(3.3,9.7){\tiny $2$}
\put(6.7,2.2){\tiny $3$}
\put(7.7,0.7){\tiny $4$}
\put(2.3,7.7){\tiny $1'$}
\put(1.2,9.2){\tiny $2'$}
\end{picture}
\caption{Square Lens Ray Diagram}
\label{f:square-lens}
\end{center}
\end{figure}
%
Place a third pin (pin 3) on the
side of the lens nearest your eye so that it is in
line with the image of the first two pins.
Place a fourth pin in line with the previous three.
Before you remove the lens, invite the instructor to verify your placement with
a laser-pointer.

Draw a straight line through each pair of pins to the edge of the lens.
Draw another line inside the lens from where the pin-lines meet the surface.
Look up the index of refraction of air and of glass and notice that they have a
range of values.  Select a reasonable value for $n_a$ and use Snell's law to
determine the index of refraction for the glass.

Notice that the angle decreases ($\theta_a>\theta_g$) when the light
passes from small $n$ (air) to large $n$ (glass).
Similarly, when the light passes from large $n$ (glass) to small $n$ (air), the
angle increases ($\theta_g>\theta_a$).

Now consider the prism.  Place the first to pins as indicated in
Fig.~\ref{f:triangle-lens}.  I have started the drawing.
%
\begin{figure}[bh]
\begin{center}
\begin{picture}(10,8)(0,0)
% Glass
\put(3,0){\line(0,1){5}}
\put(8,2.5){\line(-2, 1){5}}
\put(8,2.5){\line(-2,-1){5}}
% light rays
\put(6,3.5){\line(-2,3){3}}
\put(6,3.5){\line(1,-5){.25}}
% normal lines
\multiput(5,1.5)(.75,1.5){5}{\line(1,2){.4}}
% Pins
\put(5.5,4.25){\circle*{.2}}
\put(4.5,5.75){\circle*{.2}}
% label angles
\put(5.9,4.65){\tiny$\theta_a$}
\put(5.6,2.3){\tiny$\theta_g$}
% label indexes
\put(3.4,4){\small$n_g$}
\put(7,3.5){\small$n_a$}
\put(7,1){\small$n_a$}
% label pins
\put(5,4.2){\tiny $1$}
\put(4,5.7){\tiny $2$}
\end{picture}
\caption{Triangular Lens Ray Diagram}
\label{f:triangle-lens}

\end{center}
\end{figure}
%
Predict where the
second pair of pins will be.  {\bf Before} you try it, inform the instructor
of your prediction.  Be prepared to justify your prediction.

Place the second pair of pins in their appropriate location and draw the lines as before.

Calculate the speed of light in each material.
Use the uncertainty to determine if $v$ is significantly different than $c$.




\onecolumn%========================================

\section{Thin Lenses}
\revised{Mar, 2004}
\label{s:ThinLens}

\subsection{Introduction}

Mirrors and/or lenses are the fundamental components of all optical systems, including video projection systems, eyeglasses, cameras, telescopes and microscopes.  All of these systems either reflect (mirrors) or refract (lenses) the light to form images of objects.  Images are formed at points in space where light rays actually intersect or at the points from which they appear to intersect.  Some images can be projected onto a screen, these are called real images.  Some other images can only be visually observed and can not be projected onto a screen, these are called virtual images.  A camera forms a real image on a piece of film where light rays from the object converge.  When using a telescope or a microscope, one actually is observing a virtual image.  The light rays from the object never actually converge at the image location; the image is located by the backward extensions of the light rays.  Another example of a virtual image is the image formed by a plane mirror.

There are also two main types of thin lenses, concave (diverging) and convex (converging), with either real or virtual images formed.  To set up diagrams showing object and image positions, the first step is to label the principle optical axis, which is a line through the center of the lens and perpendicular to the lens.  The object and the image are then located along this line and relative to the position of the lens. The distance along the principle axis from the lens to the object is called the object distance ($d_o$).  The distance along the principle axis from the lens to the image is called the image distance ($d_i$).  A special optical case occurs when the object distance is infinite, then the incident light rays will be parallel to the principle axis and will be refracted by the lens such that the light rays or their backward extensions will converge to a special point.  This point is called the focal point, not to be confused the focus point of the light. The focal length distance ($f$) is then the distance along the principle axis from the lens to the focal point.  For all thin lenses there is a physical relationship between the focal length, the image distance, and the object distance.
\[ \frac{1}{f} = \frac{1}{d_o}+ \frac{1}{d_i}. \]
This equation can be derived using the mathematics of geometry, trigonometry and algebra. In order to make this equation work for all cases, a sign convention for all of these terms needs to be established.   Look for the Sign Conventions Table for thin lenses in your textbook.

One way to locate the image position in an optical system is to draw what are called ray diagrams.  These geometrical diagrams start with drawing the principle axis and the placement of the lens on this axis.  There are then usually three rays (lines) that are drawn.  There are an infinite number of rays that are possible to draw, but there are 3  that are easy to draw, so we usually only draw these 3 rays.  These lines begin at the top of the object.  Draw the first ray parallel to the axis, at the lens it will be refracted and will pass through the focal point.  Draw the second ray from the object through the near side focal point; at the lens it will be refracted and will then travel parallel to the optical axis.  A third ray from the object that travels to the center of the lens will travel straight through the lens.  If these three rays converge, then the convergence point locates the image position.  If these exit rays are diverging, then draw the backward extensions of the diverging rays, which will then converge locating the position of a virtual image.


\subsection{Experimental Objectives}

The experimental objectives of this experiment are:  to determine the focal length of a converging lens,   to investigate the nature of image formation,  to systematically locate the real images for a converging (positive) lens as a function of the object distance,  to investigate the relationships for determining the magnification, and to locate the virtual images both with positive and negative lenses.

\subsection{Pre-Lab Work}
\begin{lablist}
\item Explain how to draw the three rays necessary to make a thin lens ray diagram.  Be very specific.
\item Draw a set of ray diagrams locating the real image formed with a converging lens,  one where the object is placed at a distances: $3f$, $2f$, $1.5f$ and $0.5f$  from the lens.
\item Write out and briefly explain the sign conventions for thin lens, show what the signs mean in a simple diagram.
\item Write out and briefly explain the equation for magnification (eq.23.10).
\end{lablist}


\subsection{The Experimental Setup}

The experimental equipment consists of an optical bench with holders for the lamps, lenses, objects and screens.

\subsection{Procedure}

\subsubsection{Measure the Focal Length of a Lens}

\begin{lablist}
\item If it is sunny outside take your convex lens and a ruler outside.  The sun is an infinitely far away source, so its rays are all parallel.  Focus the sun's light onto a piece of paper and then measure the distance from the center of the lens to the paper.  This distance is the focal length.
\item On the optical bench, use a pin, a wire screen or a transparent slide as an object, and illuminate the object with a lamp.  Place the object at the far end of the optical bench.  Place your converging lens at the following positions so that the object distance is:
    \begin{enumerate}
    \item a large distance from the lens,
    \item at two other distances slightly greater than twice the focal length,
    \item at twice the focal length,
    \item at three other distances between ($f$) and ($2f$).
    \end{enumerate}
    Move the screen back and forth until a sharp image of the object is formed on the screen.
\item[]
\item Record the object and image distances.
\item Record the object and image sizes.  Measure it directly with a ruler.
\item Determine whether the image is real or virtual, inverted or erect, and magnified or not.
\item Use the thin lens formula to calculate the focal length of your lens, for each of the above cases.  Calculate the average of these values.
\item Determine the magnification for each cases, use both formulas and compare the two.
\item Draw an accurate and scaled ray diagram for at least two of your experimental cases.  Use at least three rays for each diagram.
\item Make comparisons between the experimental results and the calculations and the ray diagram; for both the image positions and for the magnifications.
\end{lablist}


\subsubsection{Virtual Images}

For a couple of trials, both for a convex and for a concave lens locate the image positions. Virtual images may be located by the method of parallax.  Set up a pin or pencil vertically along the line of sight to the image.  Try to find a position of the pencil such that there is no parallax between it and the image when you move your head from side to side.  Draw ray diagram pictures for these cases.

\subsubsection{How an Image is Formed}
\begin{lablist}
\item Set up a single lens system to achieve a magnification of one.
\item Is it possible to view a real image directly with your eye, or do you need a screen? Explain what you observe.
\item If one-half of the lens was covered up, what happens to the image?  Try covering a few different halves.  Explain.
\item Describe how the image compares with the object, i.e. erect or inverted, and/or a left-right reversal.
\item Describe what you see if the screen is moved a few inches in either direction.
\item Describe in your own words how an image is formed.
\end{lablist}

\subsection{Analysis}
\begin{lablist}
\item Construct a graph of object distance versus the image distance.
\item Interpret the graph.  What is the corresponding equation?  Does it have any important intercepts, slopes or asymptotes?  What point denotes an equal object-image distance?  Does the graph indicate the focal length?  Does the curve indicate whether the image is real or virtual?
\item This graph can be linearized by plotting different variables.  How?  What is the significance of the corresponding slopes and intercepts?
\end{lablist}

\subsection{Questions}

\begin{enumerate}
\item Is there an upper limit to the distance of the image from the lens?  If yes, then what is that value?
\item Under what conditions can a lens be used as a magnifying lens?
\item What happens if the object is placed at the focal length of a convex lens?
\item Is the following statement true for a convex lens with a real image?
    \begin{quote}
    ``When the image is nearest to the object, the image is the same size as the object.''
    \end{quote}
\end{enumerate}







\onecolumn%========================================

\section{Thin Lenses and Mirrors - Part I}
\revised{Apr, 2007}
\label{s:LensMirror}

\subsection{Introduction}

Mirrors and/or lenses are the fundamental components of all optical systems, including video projection systems, eyeglasses, cameras, telescopes and microscopes.  All of these systems either reflect (mirrors) or refract (lenses) the light to form images of objects.  Images are formed at points in space where light rays actually intersect or at the points from which they appear to intersect.  Some images can be projected onto a screen, these are called real images.  Some other images can only be visually observed and can not be projected onto a screen, these are called virtual images.  A camera forms a real image on a piece of film where light rays from the object converge.  When using a telescope or a microscope, one actually is observing a virtual image.  The light rays from the object never actually converge at the image location; the image is located by the backward extensions of the light rays.  Another example of a virtual image is the image formed by a plane mirror.

There are also two main types of thin lenses, concave (diverging) and convex (converging), with either real or virtual images formed.  To set up diagrams showing object and image positions, the first step is to label the principle optical axis, which is a line through the center of the lens and perpendicular to the lens.  The object and the image are then located along this line and relative to the position of the lens. The distance along the principle axis from the lens to the object is called the object distance ($d_o$).  The distance along the principle axis from the lens to the image is called the image distance ($d_i$).  A special optical case occurs when the object distance is infinite, then the incident light rays will be parallel to the principle axis and will be refracted by the lens such that the light rays or their backward extensions will converge to a special point.  This point is called the focal point, not to be confused the focus point of the light.  The focal length distance ($f$) is then the distance along the principle axis from the lens to the focal point.  For all thin lenses there is a physical relationship between the focal length, the image distance, and the object distance.
\[ \frac{1}{f} = \frac{1}{d_o}+ \frac{1}{d_i}. \]
This equation can be derived using the mathematics of geometry, trigonometry and algebra.  In order to make this equation work for all cases, a sign convention for all of these terms needs to be established (see the Table in your text).

One way to locate the image position in an optical system is to draw what are called ray diagrams.  These geometrical diagrams start with drawing the principle axis and the placement of the lens on this axis.  There are then usually three rays (lines) that are drawn.  There are an infinite number of rays that are possible to draw, but there are 3  that are easy to draw, so we usually only draw these 3 rays.  These rays (lines) begin at the top of the object.  Draw the first ray parallel to the axis, then at the lens it will be refracted and will pass through the focal point.  Draw the second ray from the object through the near side focal point, then at the lens it will be refracted and will then travel parallel to the optical axis.  A third ray from the object that travels to the center of the lens and then it will travel straight through the lens.  If these three rays converge, then their convergence point locates the image position.  If these exit rays are diverging, then draw the backward extensions of the diverging rays, which will then converge locating the position of a virtual image.



\subsection{Experimental Objectives}

The experimental objectives of this experiment are:
\begin{enumerate} \itemsep 0in
\item to test the laws of reflection and refraction with mirrors and lenses,
\item to determine the focal length of a concave and a convex mirror,
\item to determine the focal length of a converging  and a diverging lens,
\item to test the Lensmaker's Equation,   and
\item to systematically locate real images for a converging (positive) lens as a function of the object distance (using the thin lens formula).
\end{enumerate}

\subsection{Pre-Lab Work}

\begin{lablist}\itemsep 0in
\item Define the focal length of a lens, using a ray diagram with parallel rays incident onto a positive converging lens.
\item Draw the three rays necessary to make a thin lens ray diagram.  Be very specific.
\item Write out and briefly explain the Lensmaker's Equation.
\item Write out and briefly explain the Thin Lens Equation
\item Write out and briefly explain the equation for magnification, both using image and object heights and using image and object distances.
\end{lablist}

\subsection{Procedure}

The experimental equipment consists of an optical bench with holders for the lamps, lenses, mirrors, objects and screens.

\subsubsection{Convex and Concave Mirrors}

\begin{enumerate}
\item Place the ray light box , label side up, on a white sheet of paper on the table.  Adjust the light box so it produces three primary color rays.  Shine the rays onto the plane mirror side (of the triangular mirror piece).  Indicate the colors and directions of the incoming and outgoing rays.

    Show the law of reflection for one of the three rays.

    Question:    Are the three colored rays reversed left-to-right by the plane mirror?

\item Use a new sheet of white paper and the concave side of the triangular (cylindrical) mirror piece.  Use the five parallel white rays from the light box and shine them onto the concave mirror.  Trace the surface of the mirror onto the paper and draw the set of 5 incoming and outgoing light rays and indicate their directions.  Circle the focal point. Measure the focal length, use a ruler.   Use a compass to draw a circle that matches the radius of curvature, then measure the radius of curvature with a ruler.

\item Repeat these steps with the convex mirror.

    Question:	What is the relationship between the focal length of the cylindrical mirror and its radius of curvature?   Do your results confirm this relationship?

\item Mount the light source at one end of the Optics Bench. The crossed arrow target should be aimed toward the mirror.  Place the concave mirror (and holder that slides onto the Optics Bench) at the other end  (about 30 cm from the light).    Place the half-screen a few centimeters from the mirror and between the light and the mirror.  Adjust the screen and/or the mirror until the reflected image is brought into focus on the screen.  Measure both the image distance and the object distance.

    \begin{enumerate}
    \item Calculate the focal distance, using the formula $\displaystyle \frac{1}{f} = \frac{1}{d_o}+ \frac{1}{d_i}$.
    \item Calculate the accuracy, using the given value of $f$ marked on the mirror holder.
    \item Calculate the magnification.
    \item Show the orientation of the image of the crossed arrow target compared to the target itself.
    \end{enumerate}
\end{enumerate}

\subsubsection{Thin Lenses (Concave and Convex)}

\begin{enumerate}
\item Place the ray light box, label side up, on a white sheet of paper on the table.  Adjust the light box so it produces four parallel rays of light.  Place the double concave lens on the paper and center it and perpendicular to the light rays.  Trace the outside of the lens and trace in the light rays,  and the direction of the rays.  From the front surface of the lens there is also a set of reflected rays; they are faint, but you can rotate the lens back and forth to see them better.  Circle this reflected focal point.   Remember from the previous exercise with mirrors that the radius of curvature ($R$) is twice this distance.  Calculate this distance $R$.  For this particular lens the radius of curvature for both sides are the same, so $R_1 = R_2$, and for a double concave lens both of these distances are defined to be negative.
	
    For a concave lens the light rays diverge when refracted at the lens.  To determine the  focal length for these rays, draw their backward extensions.  This intersection point is the focal point.  The distance from this point back to the center of the lens is called the focal point.  Measure this focal distance.

    The Lensmaker's Formula is
    \[\frac{1}{f} = (n - 1) \left(\frac{1}{R_1} + \frac{1}{R_2}\right).\]
    The index of refraction for this piece of plastic is 1.5.  Using your measured $R$ distances, calculate $f$ with this formula.  Remember that this lens is also called a negative lens because the $f$ is negative.  Compare your measured $f$ and your calculated $f$ with a \% difference.  The true $f$ is 12cm.  Compare your measured $f$ with this given value with a \% difference.

\item Mount the light source at one end of the Optics Bench. The crossed arrow target should be aimed toward the lens.  Place the convex lens (20 cm F.L. and holder that slides onto the Optics Bench) at the other end (about 30 cm from the light).    Place the tall screen a few centimeters past the lens.   Adjust the screen and/or the lens distance until the refracted image is brought into focus on the screen.  Measure both the image distance and the object distance. Measure both the image and object height (use a ruler).     Without moving the screen, find another position of the lens where a clear image can be seen.  Repeat the same set of measurements.

    For these 2 cases, calculate the magnification.  Use both sets of formulas and calculate a \% difference for each trial. Use $M = d_i/d_o$  and use $M = H_i / H_o$, where $H$ indicates the height of the image or object.
    For these 2 cases, use the thin-lens formula and compare to your experimental results.   Give a \% difference.

    Draw an accurate ray diagram for these 2 different cases.  Use at least three rays for each diagram.  Discuss the sign conventions used for the ($d_o$), ($d_i$), ($f$), and $M$.
\end{enumerate}

\subsection{Questions}

\begin{enumerate} \itemsep 0in
\item Is it possible to view a real image directly with your eye, or do you need a screen?  Try it.
\item If one-half of the lens was covered up, what would happen to the image?  Try it.
\item Explain the process of how a real image is converged/ projected onto a screen.
\end{enumerate}


\onecolumn%========================================

\section{Thin Lenses and Mirrors - Part II}
\revised{Apr, 2007}
\label{s:LensMirror2}

Recall your work with the mirrors and lenses in Lab~\ref{s:LensMirror}.

\subsection{Procedure}

\subsubsection{Single Thin-Lens System}

Collect data changing the distance from the light to the screen by at least 10 cm, between data points.   Again, like last week there will be two positions of the lens to give a clear image on the screen. Find and record the same data as before.  Repeat this for a total of 4 light to screen distances, and this yields 8 image and object distances.   For all points make sure that the object distance is larger than the focal length.

Make a graph on Excel of object distance vs image distance.  This graph will have two asymptotes.  Explain their meaning.

Make a graph on Excel of  $1/d_o$ vs $1/d_i$ with your 8 data points.   Determine both the x and y intercepts.  Explain the meaning of both of these intercepts.   Explain the meaning of the slope of the line.   Calculate the accuracy of the focal length for this experiment.


\subsubsection{Double Thin-Lens System}

\begin{lablist}
\item Set up a 2-lens system using 2 lenses (f= -150 mm and f=100 mm), an object, and a screen, on the optical bench.
\item Choose an initial object distance, and a distance D between the two lenses so that the final image is a real image.
\item Experimentally determine the image location, and  the magnification.
\item Calculate the image positions (both of them) and compare this to the experimental result.
\item Compare the two methods for getting the magnification.
\end{lablist}

	
\subsubsection{Refracting Telescope System}

\begin{lablist}
\item Using just the objective lens (f=250 mm), and an illuminated scale as the object at the far wall, experimentally locate the real image formed by this lens.  Compare this to the focal length of the lens.
\item Aim your telescope at the object on the far wall.  The field of view is very small.
\item Use the eyepiece lens (f=100 mm),  and place it as in the figure in the text, (figure 25.8).
\item View the magnified virtual image with your eye very near the eyepiece lens.  With the use of both eyes (one viewing through the telescope and one looking directly at the object) experimentally determine the magnification.  The magnified image of the scale and the unmagnified scale will be superimposed.  Count the number of divisions of the scale as viewed directly that cover exactly one division of the magnified scale.
\item Calculate the image magnification using the appropriate telescope equations, $\displaystyle M = \frac{f_o}{f_e}$.
\item Compare the experimental results with these two calculations  (\% differences).
\end{lablist}


\subsection{Questions}
\begin{enumerate}
\item If a telescope is accurately focused on a distant object, in what direction must the eyepiece be moved to focus on a nearby object?
\item Is the near point of the eye (25 cm) necessary and used for both the telescope and the microscope calculations of the magnifications?
\item Is the following statement true?   "When a real image is produced by a lens, the image is nearest to the object when it is the same size as the object."
\item Is there an upper limit for the distance of the image from the lens?  If yes, then what is that value?  Why?
\end{enumerate}




\twocolumn[ %========================================

\section{Wavelength of Light}
\revised{Spring, 2006}
\label{s:light}
\setcounter{dave}{0}

Like some previous experiments, this experiment is demonstrative rather than discovery-based.
The focus of the report will be to show how you understand the various facets of the experiment, rather than on how your result allows you to draw conclusions about a principle.  The features which you are demonstrating to yourself are indirect measurements using similar triangles, the properties of a diffraction grating, that light can interfere with itself and what that means, the size of a single wavelength of light, and what spectral lines are.]

\bigskip\footnoterule\bigskip
Indirect measurements can be used to measure quantities which are not directly
accessible.  The wavelength of light is such a quantity.  The wavelength of
visible light is such that some tens of thousands of waves are required to
reach one centimeter.
%
\begin{question}
\item Try to envision/draw tens of thousands of water ripples within $1\unit{cm}$.
\end{question}
%

%Recall from the previous lab ({\bf Speed of Sound}) that waves can superimpose constructively or destructively.  The loud and soft of sound
Recall from the previous lab ({\bf Lab~\ref{s:standingwaves} Standing Waves on a String}) that waves can superimpose constructively or destructively.  The large amplitude wave and the randomly wobbling waves
are respectively analogous to the bright and dim of light.
In order to see the bright and dim, the waves must interact with each other.  In this lab we will use a diffraction grating to make the waves diffract.
In {\bf Lab~\ref{s:standingwaves} Standing Waves on a String}, the waves were reflecting.
In {\bf Lab~\ref{s:refraction} Reflection and Refraction}, we will also consider refracting waves.
%
\begin{question}
\item What is the difference between reflection and refraction?
\item What is the difference between refraction and diffraction?
\item What does monochromatic mean?  (Hint: the prefix \latex{``}\html{"}chromo-" means color.)
\end{question}
%
The diffraction grating is a piece of glass or other transparent
material upon which a large number of closely spaced lines have been engraved.
When a beam of monochromatic light is shown upon a diffraction grating,
the light is scattered into a series of bright and dim spots as portrayed in Fig.~\ref{f:diffract}.
%
\begin{figure}[t]
\begin{center}
\begin{picture}(13,10)
\put(7,5){\line(-1,0){7}}
%
\put(6.75,3){\line(1,0){.5}}
\put(6.75,3){\line(0,1){4}}
\put(7.25,7){\line(-1,0){.5}}
\put(7.25,7){\line(0,-1){4}}
%
\put(7,5){\line(4,5){3}}
\put(7,5){\line(1,1){4}}
\put(7,5){\line(4,3){5}}
\put(7,5){\line(2,1){5}}
\put(7,5){\line(4,1){5}}
\put(7,5){\line(1,0){5}}
\put(7,5){\line(4,-1){5}}
\put(7,5){\line(2,-1){5}}
\put(7,5){\line(4,-3){5}}
\put(7,5){\line(1,-1){4}}
\put(7,5){\line(4,-5){3}}
%
\put(3.6,2.8){\vector(1,0){3.4}}
\put(3.6,2.8){\vector(-1,0){3.4}}
\put(3.4,2){$L$}
\put(.1,4.1){\vector(0,1){.9}}

\put(.1,4.1){\vector(0,-1){.9}}
\put(-.6,4){$y$}
%
\put(9,5){\oval(1,1)[tr]}
\put(9.7,5.1){$\theta$}
%
\put(5,5){\oval(1,1)[bl]}
\put(4,4.4){$\theta$}
%
\multiput(7,5)(-1.2,-.3){6}{\line(-4,-1){.75}}
\multiput(0.2,5)(0,-.6){3}{\line(0,-1){.35}}
%
\end{picture}
\caption[The scattering of monochromatic light through a diffraction grating.]{The scattering of monochromatic light through a diffraction grating.  The light that you see coming out at angle $\theta$ will appear to have come along the dashed line.  You can measure $y$ and $L$, which determine $\theta$.}
\label{f:diffract}
\end{center}
\end{figure}
%
In this figure, the the light is actually coming from the left towards the right.
When it hits the diffraction grating, the beam splits into many new beams, each of which travel at specific angles.
If you are standing to the right of the diffraction grating, the beam that happens to diffract at the angle labeled $\theta$ will appear as if it came from the direction indicated by the dashed line on the left.
The angles at which the beam diffracts depend specifically on the wavelength of the incident light.
%
\begin{question}
\item\label{q:theta} Determine, from trigonometric relations, how measuring $y$ and $L$ in Fig.~\ref{f:diffract} will allow you to calculate $\theta$.
\end{question}
%

Fig.~\ref{f:grating} shows a very zoomed in view of the light and the grating.
%
\begin{figure}[bthp]
\begin{center}
\begin{picture}(13,10)
\multiput(7,2)(0,3){3}{\line(-1,0){7}}
\multiput(7,2)(0,3){3}{\line(2,1){5}}
\multiput(6.75,-.5)(0,3){4}{\line(0,1){2}\line(1,0){.5}}
\multiput(7.25,1.5)(0,3){4}{\line(0,-1){2}\line(-1,0){.5}}
\multiput(6.875,8.25)(.75,-1.5){6}{\line(-1,2){.5}}
%\put(6,10){\line(1,-2){4}}
\put(6.25,6.5){\vector(0,1){1.5}}
\put(6.25,6.5){\vector(0,-1){1.5}}
\put(5.75,6.45){$d$}
\put(7.75,4.75){\vector(2,1){.75}}
\put(8,4.875){\vector(-2,-1){.75}}
\put(8.125,4.25){$l$}
%
\put(5.25,5){\vector(0,1){3}}
\put(5.25,5){\vector(0,-1){3}}
\put(4.25,5.2){$2d$}
\put(8.3,2.25){\vector(2,1){1.25}}
\put(8.3,2.25){\vector(-2,-1){1.25}}
\put(8.125,1.25){$2l$}
%
\put(7,8){\line(1,0){3}}
\put(8,8){\oval(1,1)[tr]}
\put(8.5,8.2){$\theta$}
\end{picture}
\caption[A \ close-up \ of \ the \ scattering \ of \ monochromatic light through a diffraction grating.]{A close-up of the scattering of monochromatic light through a diffraction grating.  Without considering the details of the interaction, assume that the light scatters in all directions from each of the slots.  Since only the light which heads out in the same direction can superimpose and strike the same location on the screen, we will only consider the light which exits in parallel beams at angle $\theta$.}
\label{f:grating}
\end{center}
\end{figure}
%
Since only the light which leaves in parallel will stay together as it makes its way to the screen, this is the light that will be superimposed on itself.  It these rays are crest-to-crest and trough-to-trough, then they interfere constructively and make a bright spot.  If they are crest-to-trough, then they interfere completely destructively and no light reaches that spot on the screen.  The determining factor is the length $l$.  If that is one wavelength exactly, then we get constructive interference.  If it is a half-wavelength or a quarter, or anything other than an integer number of wavelengths, then we do not get constructive interference and we do not get a bright spot.  The length $l$ is determined by the angle $\theta$ which we are looking from.  (Recall Fig.~\ref{f:diffract} shows that we only get bright spots at certain angles. These are the angles which correspond to $l=\lambda$, $l=2\lambda$, etc.)
%
\begin{question}
\item\label{q:l} Determine, from trigonometric relations, how knowing $d$ and $\theta$ in Fig.~\ref{f:grating} will allow you to calculate $l$.
\item Write the expression from Questions~\ref{q:theta} and~\ref{q:l} in terms of the same trig function (both with a sine or both with a cosine).
Solve these equations for $l$ in terms of $d$, $y$, and $L$.  (You might need to use the Pythagorean Theorem.)

Aside: The second order line comes from a larger angle found when $l=2\lambda$.  Similarly, the third order dot comes from the $l=3\lambda$ angle. etc.
\end{question}
%

\subsection*{The Experiment}
Hold the diffraction grating up to various light sources, notice that it spreads the light out into a spectrum (like a rainbow).  If the light only provides specific colors, then you get lines.  If it is an incandescent light, then you will get a broad spectrum.
Notice also that when you look through the diffraction grating, the colors of the light repeat.  These are the first, second, and third orders of the spectrum.  There are more orders, but they get fainter the further you go.

Set a florescent discharge tube at the far end of the table and the diffraction grating at the near end.  You will need to know the distance $L$.
Do not measure $y$ from the central point to the dot, as labeled in Fig.~\ref{f:diffract}.
Rather, measure the distance from the left dot to the right dot (of the same color and the same order) and divide this by two to get $y$.
It should also be possible to measure the $y$ value for the second order as well as for the third order; please do so.  The results for each order are related in a very specific way; figure out the pattern and, for each color, average the two corrected results of all orders.  Finally, compare (\%-error) your measurements with the accepted values from the table.




\subsection*{Questions}
%
\begin{question}
\item It is possible that the incident light has red and blue light.  Recall that red light has a longer wavelength than blue light.  Consider that when $l=\lambda_{red}$ we get a bright red line and when $l=\lambda_{blue}$ we get a bright blue line.  Which color will be seen closer to the central source of light?  Why?
\item When heated, every chemical element emits a unique set of colors.  When viewed through a diffraction grating, each color is separated in the way described.  These bright spots are called the ``spectral lines'' of the element.  How do astronomers identify the chemical elements that make up the stars if they can't go visit the stars to get a sample?
\item How do ``neon'' signs work?  Why are some orange and others yellow or blue or red?  Are these all ``neon lights''?
\end{question}
%



\onecolumn%========================================

\section{The Plane Diffraction Grating Experiment}
\revised{Apr, 2003}
\label{s:diffraction}



\subsection{Procedure}
\begin{enumerate}
\item Measure the diffraction angle of the four lines in the Balmer series of Hydrogen, both for m=1 and m=2.   Take the average of the right and left values.  Measure these four lines with two different gratings  (like 100 and 600 lines per mm).
\item Calculate these four wavelength values using the Balmer series equation  (eq. 43.25, on page 1061.)
\item Use these values to calculate the distance between the slits, d. (once for each line).   Compare this to the stated value on the grating.
\item Using this average value of d, measure the  diffraction angles and calculate the wavelengths of a few additional lines; from He, Hg, and Na.  Also look for pairs of lines that are some  what close together, like the yellow lines of Na and Hg.   Make a graph of  the  diffraction angles versus the wavelengths, one graph for each grating.  That is, for each grating place all of the data taken from the multiple lamp sources.  Determine the dispersion  D as a function of  the wavelength,  $D =  \frac{d\theta}{d\lambda}  =  m/ ( d cos \theta )$.   That is, compare your graph to this equation.   For the data pairs that are close together,  also check this dispersion equation.
\item Using a Na data and the Hg data,  compare the resolving power of two gratings (100 lines per mm and 600 lines per mm) with your measurements.
\item Of course, carry out precision and accuracy calculations for each part.
\end{enumerate}

%\twocolumn

%\addtocounter{page}{1}
%\addcontentsline{toc}{section}{Index}
%\addtocounter{page}{-1}

%\label{s:index}
%\input{G-UP-Labs2.ind}





\twocolumn[ %========================================

\section*{Simple and Compound Pendulums}
\revised{Spring, 2006}
\label{s:pendulum}

A pendulum is an object suspended from a point in such a manner that it can
swing in an arc of a circle.  A {\it simple}\/ pendulum is one in which all of
the mass is concentrated in a point at the end of a massless cord.  Obviously,
simple pendulums do not exist.  However, we may make a real pendulum that
approximates the ideal simple pendulum by using a very light string with a
(relatively) large mass at the end. \\]

\noindent
\underline{\sc A Simple Pendulum}
%
\begin{enumerate}
\item \label{l:rates}  Do all pendulums swing at the same rate?  If so, can you determine the guiding principle which determines this rate?  If not, what variable(s) determine the rate?
\item What is the frequency of a grandfather clock pendulum?  The period?  Are you sure?
\end{enumerate}
\setcounter{dave}{\arabic{enumi}}

Investigate the behavior of a simple pendulum and the factors which govern the period of its oscillations (see Sec.~\ref{ss:oscillation}).
Three factors that we might expect to have an
effect on the period are the {\bf amplitude}\/ of the swing, the {\bf mass}\/ of the pendulum
bob, and the {\bf length}\/ of the pendulum.

Using the guidelines below, investigate the amplitude, the mass, and the length independently --- vary one while holding the others constant.
Before carrying out the experiments, write down your predictions for the outcomes.
%
\begin{enumerate}\setcounter{enumi}{\thedave}
\item How will you measure the amplitude of the swing?  (Work with small and medium amplitudes; do only one very large amplitude.)
%\item The length is from the pivot point at the top to the center of mass of the bob.
\item Be sure to record the value of all three variables while varying any one of them.
\item Measure the period a few times for each data point according to Sec.~\ref{ss:oscillation}.
\item Vary each parameter over several (more than four) values.  If you suspect that the period does not depend on a variable, use a few widely spaced values.  If you then find that the period does depend on that value, then take more data points in between these values.  Be organized and present your data in order of values not in the order taken.
{\bf \item On all graphs, place $T$ on the horizontal in units of seconds per radian rather than seconds per oscillation.}
\item Except for the large amplitude trial, do you observe any relationship between the {\bf amplitude}\/ and the period of the oscillations?
Between the {\bf mass}\/ and the period?
Between the {\bf length}\/ and the period?
\item Plot a graph for each parameter that affects the period of oscillation.
\item What shape is each graph drawn?
Write the equation (with coefficients from the trendline) which best fits your graph.
Figure out the units of each coefficient.
\item What would a vertical line indicate?
\item One \ of \ the \ coefficients \ in \ your \ trendline \latex{\newline} equation should be a familiar quantity.  What quantity is it?  What value do you find?
\item[Hint:] What forces a pendulum to swing?
\end{enumerate}\setcounter{dave}{\arabic{enumi}}
%
Solve the equation that contains the familiar quantity for the period in units of seconds per oscillation (not seconds per radian),  {\bf thereby deriving the equation for a simple pendulum.}

%\vfill%\subsubsection*{Experiment Design}

Design a pendulum for each period of $0.25\unit{sec}$, $1\unit{sec}$, and $1 \unit{min}$.
Design a pendulum which has the same frequency as your pulse.
Test your designs.

\vspace{.3cm}\hrule\vspace{.3cm}

\noindent
\underline{\sc A Compound Pendulum}

Ask your instructor for the equation of a compound pendulum.  Compare and contrast it to the equation you derived for the simple pendulum.  When does the compound pendulum equation simplify to the simple pendulum?

Figure out how to measure the moment of inertia for a ruler swung about a pivot which is $30\unit{cm}$ below the highest point.

UP students:  Compare this to the theoretical value found using the parallel-axis theorem.





\onecolumn %========================================

\section*{The Behavior of Springs}
\revised{Spring, 2006}
\label{s:hooke}

All springs stretch when pulled.
Hooke's Law describes the relationship between the extension of a
spring, $x$, and the force required to cause the extension, $F$.
We can find the force which causes the stretching by hanging known
weights from the spring.  Do so with 10 different weights, ranging
from $950\unit{N}$ to $10000\unit{N}$.
For each, measure the extension.  Plot the force versus the extension.
%
\begin{enumerate}
\item What does the graph look like?
\item Use Excel to find the equation of the best trendline; \ this \ relationship \ (equation) \ is known as Hooke's Law.
What are the coefficients in Hooke's Law?  If you think it should be linear, then find the uncertainty on the slope and intercept (Recall Sec.~\ref{ss:excel}).  Otherwise, simply note the equation and the shape of the trendline.
\end{enumerate}
\setcounter{dave}{\arabic{enumi}}
%
{\bf You have just experimentally derived Hooke's law} for your spring.
Each spring has characteristic coefficients.
In the equation of the trendline, the coefficient of the linear term
is denoted $k$ and is known as \latex{``}\html{"}the spring constant."
%
\begin{enumerate}\setcounter{enumi}{\thedave}
\item What are the units of the spring constant?
\end{enumerate}
\setcounter{dave}{\arabic{enumi}}
%
{\bf Measure the spring constant from the graph for this and another spring.}
You may plot both on the same graph paper.

\bigskip
In order to check our initial measurement, we would like to also make an
alternative measurement of the spring constant.
Carry out a detailed error analysis of both measurement methods
in order to determine whether or not the
values calculated from the different methods are consistent.

\bigskip
As might be expected, the equation for the oscillation period, $T$, of a
mass that undergoes periodic motion due to a spring involves the spring constant.
As you did last week, measure the period associated with several small amounts
of mass (on the order of 200 grams or so).  {\bf Note}: the oscillating mass
should be displaced only slightly from the equilibrium position.
A large initial displacement can change the attention of the lab from periodic
motion to projectile motion\ldots which would be bad.

Plot $m$ versus $T$, where $T$ is measured in units of seconds per radian,
not seconds per oscillation.
%
\begin{enumerate}\setcounter{enumi}{\thedave}
\item What shape is your graph?
Write the equation (with coefficients from the trendline) which best fits your graph.
Figure out the units of each coefficient.
\item One of the coefficients in your trendline equation should be a familiar quantity.  What quantity is it?  What value do you find?
\item[Hint:] What are the units of the coefficient?
\end{enumerate}\setcounter{dave}{\arabic{enumi}}
%
Solve the equation that contains the familiar quantity for the period in units of seconds per oscillation (not seconds per radian).  {\bf You have just derived the equation for the period of a spring.}





\onecolumn%========================================

\section{Double Lens Systems: A Telescope and a Microscope}
\revised{Mar, 2008}
\label{s:DoubleLens}

\subsection{Introduction}

Mirrors and/or lenses are the fundamental components of all optical systems, including video projection systems, eyeglasses, cameras, telescopes and microscopes.  All of these systems either reflect (mirrors) or refract (lenses) the light to form images of objects.  Images are formed at points in space where light rays actually intersect or at the points from which they appear to intersect.  A camera forms a real image on a piece of film where light rays from the object converge.  When using a telescope or a microscope, one actually is observing a virtual image, where the image is at a position near infinity.

Combinations of lenses are used extensively in most optical instruments.  It is instructive to see how a two-lens system achieves its magnification.  In this experiment, three different 2-lens systems will be investigated both qualitatively and quantitatively.  Qualitatively, systems can be studied with ray diagrams and with classifications of the object and image; i.e. are they erect or inverted, real or virtual, magnified or smaller.  Quantitatively, the systems can be compared with calculations of the image distances, and both lateral and angular magnifications.  The thin lens formula can be used for each lens in the system, and there are equations specific for certain telescopes and/or microscopes that can be used.  Read your text.

\subsection{Experimental Objectives}

The experimental objectives of this experiment are\ldots
\begin{lablist} \itemsep -6pt
\item \ldots to determine the image locations in a 2-lens system,
\item \ldots to investigate the relationships for the total magnification in a 2-lens system,
\item \ldots to determine the angular magnification for a simple magnifier lens,
\item \ldots to set up an optical bench telescope and to determine its magnification,
\item \ldots and to set up an optical bench microscope and to determine its magnification.
\end{lablist}

\subsection{Pre-Lab Work}
We will use the notation that an image size is denoted $h_i$, an object size is $h_o$, an image location is $s_i$, and an object distance is $s_o$.  When we use the double-lens system, let $l$ be the distance between the lenses and measure $s_i$ and $s_o$ from its respective lens.
All pages below refer to Hecht, third edition.
\begin{lablist}\itemsep -2pt
\item Show a picture how the image from the first lens ($s_{i1}$) becomes the object for the second lens ($s_{2o}$).  Write down the relationship between $s_{i1}$, $l$, and $s_{o2}$.
\item Briefly explain the equations for angular magnification: (pg 862) $\displaystyle M = \frac{d_n}{f}$
\item Draw the complete ray diagram for a simple compound microscope (Fig.~24.29 on pg 869)
\item Briefly explain the equation for angular magnification for a simple microscope: (See pg 869)
$\displaystyle M_{\rm A} = M_{\rm TO} \cdot M_{\rm AE}$
\item Draw the complete ray diagram for a simple telescope
\begin{itemize}
\item Fig.~24.30 (pg 870) describes a telescope used to observe relatively close objects.
\item Fig.~24.31 (pg 871) describes a telescope used to observe astronomically-far objects.
\end{itemize}
\item Briefly explain the equation for angular magnification for a simple telescope (See pg 871)
$\displaystyle M_{\rm A} = -\frac{f_O}{f_E} $
\end{lablist}


\subsection{The Experimental Setup}

The experimental equipment consists of an optical bench with holders for the lamps, lenses, objects and screens.

\subsection{Procedure}

\subsubsection{A Two-Lens System}

\begin{lablist}
\item On the optical bench, use the illuminated arrow as the object.  Place the f = -15cm diverging lens 12 cm from the object. Place the f = +10 cm converging lens 6 cm farther from the first lens. With the screen, find the final image position for this 2 lens system.  That is, place the screen at a position where the final image is in the best focus.  Measure the image and object distances.
\item Measure the image size, and calculate the overall magnification. M= image size /object size.
\item Draw a ray diagram for the total system but show it in two parts, one for the first lens and one for the second lens.
\item Show the calculation, calculate the image position for the first lens, then calculate the second image position.  Calculate the magnification for each lens and then for the total system.    $M_{\rm total} = M_1 \times M_2$; where $M_1$ = image distance / object distance.
\item Calculate \% differences and explain any discrepancies. %differences between the observations and the calculations.
\end{lablist}

\subsubsection{Angular Magnification}

\begin{lablist}
\item Place the object (lamp) at the 20 cm mark on the track.  Place the $f=10$cm lens at the 11 cm mark on the track. Place or hold another ruler at the 25 cm  position (resting on top of the lamp box). Place your eye at the 0 cm position.
\item Determine how many cm on the actual scale fit into 1 cm  on the magnified scale.   Look at the magnified image through the lens and simultaneously look at the ruler (above the lens).   Compare to the text book value of  $\displaystyle M_A = \frac{25}{f}$.   Try again for 30 cm.
\item Measure {\it your} near point distance (Fig.~24.23 pg 862), the closest distance an object can be to {\it your} eye and still have it be clear.    Each person in the lab group should do this part themselves.  Which is a better near point number for you,  $\displaystyle M_A = \frac{25}{f}$ or $\displaystyle M_A = \frac{30}{f}$  or some other ratio?
\end{lablist}


\subsubsection{A Compound Microscope}
\begin{lablist}
\item Set up the optics exactly like Fig.~24.29 on pg 869.  Use the $f=20$cm lens as the objective lens and the $f=10$cm lens as the eyepiece lens.
\item Estimate the total magnification (just visually).  Look at the image through the lens system with one eye and look at the object directly, outside the optics system.  Do it like as given above for the single lens system.
\item Calculate the magnification. (Eq.~24.17 on pg 869)
\item Change the distance between the lenses by 1 or 2 cm.  How does the final image change?  Does this change give a better comparison to the magnification calculation?  Change the distance between the lenses by a factor of 2.  Again compare to the magnification formula.
\item Each person in the lab group should do this part themselves.
\end{lablist}

\subsubsection{A Simple Refracting Telescope}

\begin{lablist}
\item Use a $f=25$cm lens as the objective lens and the $f=10$cm lens as the eyepiece lens. Set them accordingly to the diagram (Fig.~24.31 on pg 871).
\item Aim your telescope at the object on the far wall.  The field of view is very small.
\item View the magnified virtual image with your eye very near the eyepiece lens.  With the use of both eyes (one viewing through the telescope and one looking directly at the object) experimentally determine the magnification.  The magnified image of the scale and the unmagnified scale will be superimposed.  Count the number of divisions of the scale as viewed directly that cover exactly one division of the magnified scale.
\item Make comparisons between the experimental results and the calculations and the ray diagram and compare the magnification with the text book formula.
\item Try moving the lenses (one or both) by a cm or less.  How does the final image change?   Does this change give a better comparison to the Magnification calculation?  Try this kind of change a couple of times.	
\item Each person in the lab group should do this part themselves.
\end{lablist}

\subsection{Questions}

\begin{enumerate}
\item If a telescope is accurately focused on a distant object, in what direction must the eyepiece be moved to focus on a more nearby object?  Explain.
\item Explain why the equations might not give very accurate comparisons to the observed results.
\end{enumerate}




\onecolumn%========================================

\section{Diffraction of Light}
\revised{Spring, 2011}
\label{s:Diffraction}
\setcounter{dave}{0}


\subsection{Introduction}

Under most everyday circumstances, light behaves like a particle.  If you place a solid object in the way of a stream of photons, the photons will be blocked by the object and the object will cast a shadow.  Under this assumption, we can trace rays of light using geometry, as we have done for reflection, refraction, and thin lenses.  However, light is known to behave as both a particle and a wave.  The wave nature of light becomes readily apparent when the object placed in the path of a beam of light is at or near a size equal to the wavelength of the light.  When visible light is passed through a very narrow slit, the light spreads out from the slit and forms a diffraction pattern.  When light passes through two narrow slits set at some small distance apart, the light from one slit with interfere with the light from the other slit creating an interference pattern.  This exercise explores the behavior of light through both a single slit and a double slit.

\subsection{Background - Single Slit}
If light of wavelength $\lambda$ passes through a slit of width $D$, where $D >> \lambda$, then the light will behave like a particle and pass directly through the slit as seen on the left side of the Figure~\ref{f:diff.diagram} below.  For a small slit, with $D$ on the order of $\lambda$, light will be emitted from the slit as seen on the right side of Figure~\ref{f:diff.diagram}.  If $D$ is small and on the order of $\lambda$, then light will behave like a wave and spread out from the slit as if it was being emitted from a point source.  This idea is called Huygens' Principle.
%
\begin{figure}[p]
\begin{center}
\includegraphics[width=6in]{single_slit_diagram.png}
%JOE%\vspace{1in}
\end{center}
\caption{Examples of $D>>\lambda$ and $D<\lambda$.}
\label{f:diff.diagram}
\end{figure}

According to Huygens Principle, every point at the slit can be thought of as an individual light source emitting light waves in radial directions.  The light from different points along the slit can interfere with one another as they emerge from the slit.
Consider first tracing the path of light emerging from two points along the slit - one very near the edge of the slit and the other very near the center of the slit.  This situation is seen in Figure~\ref{f:diff.aperature}, with a slit of width $D$.  Light is emitted at an angle $\theta$ with respect to a line running perpendicularly through the center of the slit.
%
\begin{figure}[p]
\begin{center}
\includegraphics[width=4in]{single_slit_aperature.png}
%JOE%\vspace{1in}
\end{center}
\caption{A ``zoomed in'' view of the aperture.  We can imagine the light that is actually passing through the aperture is, instead, coming from a collection of point sources that fill the aperture.}
\label{f:diff.aperature}
\end{figure}

It should be noted that the path length for the light near the edge of the slit is shorter than the path length for the light at the center of the slit.  Using simple geometry, it can be shown that the difference in path length is equal to $\frac{D}{2} \sin\theta$.  The light along each path will eventually meet up again, say at a distant screen.  If the difference in path length is equal to an integer multiple of the wavelength ($\lambda$, $2\lambda$, $3\lambda$, \ldots), then the two wavefronts will combine constructively, producing a bright spot on the screen.  If the difference in path length is equal to a half integer multiple of the wavelength ($\txtfrac{1}{2}\lambda$, $\txtfrac{3}{2}\lambda$, $\txtfrac{5}{2}\lambda$, \ldots), then the two wavefronts will combine destructively, producing a dark spot on the screen.  These conditions apply to any two individual points in the slit that are spaced $\frac{D}{2}$ apart.  Therefore, on the screen, a series of bright and dark spots will appear, as seen in the intensity profile in Figure~\ref{f:diff.graph}.
%
\begin{figure}[p]
\begin{center}
\includegraphics[width=6in]{single_slit_graph.png}
%JOE%\vspace{1in}
\end{center}
\caption{The intensity profile of light passing through a slit with $D\approx \lambda$.  The peaks indicate bright spots, brighter where the peak are higher, and the troughs (minima) indicate dark spots.}
\label{f:diff.graph}
\end{figure}

Notice that there are several dark spots at specific intervals.  The brightest spot is straight ahead, $\theta=0$.  Each subsequent dark spot is a wider angle, with one side being positive and the other side being negative angles.  The condition for minima (or dark spots) is that the path length difference, seen in Fig.~\ref{f:diff.aperature}, is equal to a half-integer multiple of the wavelength.  In terms of measurable parameters, this can be written as:
%
\begin{equation}\label{eq:diff.minima}
\frac{D}{2} \, \sin\theta_m = \frac{m \lambda}{2} \mbox{\hspace{.5cm} where $m=\pm 1, \pm 2, \pm 3, \ldots$}
\end{equation}
%
$\theta$ is now given a subscript$m$ to indicate that it is the quantity that changes for each value of $m$.
Solving Eq.~(\ref{eq:diff.minima}) for $\sin\theta_m$, results in:
%
\begin{equation}\label{eq:diff.minima2}
\sin\theta_m = \frac{m \lambda}{D} \mbox{\hspace{.5cm} where $m=\pm 1, \pm 2, \pm 3, \ldots$}
\end{equation}
%
While measuring the angle, $\theta_m$, is theoretical possible, it is much easier to approximate the angle using the geometry of the setup.  Assuming a setup as seen in Figure~\ref{f:diff.setup} (viewed from above), laser light passed through a slit will produce a diffraction pattern on a screen placed a distance $L$ away from the slit.

In Figure~\ref{f:diff.setup}, the distance $x$ represents the distance between the central maximum ($\theta=0$) and a minimum on the screen.  In this situation the angle $\theta$, the same angle as in Eq.~(\ref{eq:diff.minima2}), is given by Eq.~(\ref{eq:diff.minima3}).  It should be noted that in most experimental setups $L \gg x$, which implies that the hypotenuse $\sqrt{L^2+x^2} \approx L$ -- That is, the difference is smaller than the experimental uncertainty. In this case, the tangent of the angle is approximately equal to the sine of the angle.
\begin{equation}\label{eq:diff.minima3}
\sin\theta \approx \tan\theta = \frac{x}{L}
\end{equation}
Finally, combining Eqs.~(\ref{eq:diff.minima2}) and~(\ref{eq:diff.minima3}), we eliminate the angle and express {\bf the location of the minima for single-slit diffraction} in terms of easily measured quantities.
%
\begin{equation}\label{eq:diff.minima4}
\frac{x}{L} = \frac{m \lambda}{D} \mbox{\hspace{.5cm} where $m=\pm 1, \pm 2, \pm 3, \ldots$}
\end{equation}
%
%
\begin{figure}[bht]
\begin{center}
%JOE%\vspace{1in}
\includegraphics[width=6in]{diffraction_setup.png}
\end{center}
\caption{A ``zoomed out'' view of the apparatus.  The layout for the experiment shows the laser that produces the light that passes through the aperture, previously shown in Figs.~\ref{f:diff.diagram} and~\ref{f:diff.aperature}, and travels on to the screen at the far right.  The angle $\theta$ in this diagram is the same as that indicated in Fig.~\ref{f:diff.aperature} and the line drawn indicates the $x$ that corresponds to $m=3$, the third minimum.}
\label{f:diff.setup}
\end{figure}


\subsection{Objective}
Verify that Huygens' Principle accurately predicts the location of the minima in the intensity spectrum for single-slit diffraction.  Extend the analysis to understanding the pattern found with double-slit diffraction.

\subsection{Setup}

In order to verify the equations for diffraction above, you will be using a red laser and slits of varying widths.  Place the laser, slit aperture, and screen on the optics bench in front of you, as seen in Fig.~\ref{f:diff.setup}.  The laser should be about $3\unit{cm}$ from the slit aperture.  The screen can be placed on the opposite end of the optics bench, about $50 \unit{cm}$ away from the slit aperture.  For each diffraction pattern, before you start, tape a piece of paper to the screen in order to make a trace of the pattern of minima (dark spots).

\subsection{The Experiment}

\subsubsection{Single Slit Diffraction}

\begin{question}
\item Using the setup described above, produce a diffraction pattern using the $0.04 \unit{mm}$ slit.
    \begin{enumerate}
    \item Rotate the disk until the proper slit 'snaps' into place.  You should have a linear pattern of alternating maxima (brighter spots) and minima (dark spots) on the screen.  Without moving or tilting the screen, mark and label the location of the first set of minima on each side of the brightest, central maximum.  These locations mark the position of the first order minima ($m=1$).  Continue marking and labeling additional pairs of minima, at least to the 3rd order ($m=3$).
    \item You should now either know or be able to measure values of $x$, $L$, $m$, and $\lambda$ (with uncertainties) as described in the background section.
    \item[] Note that $x$ is the distance from the central maximum to the minima at each order $m$.  The best way to measure this value would be to measure from the 'left' $m^{\rm th}$-order minimum to the 'right' $m^{\rm th}$-order minimum and divide this distance by two.
    \item[] The wavelength of the laser light, $\lambda$, is posted on the laser.
    \item Using your values for each order, calculate the value of the slit width, $D$, and the uncertainty in this value.  Take an average of the values for each order and compare it to the known value of $D$.
    \end{enumerate}
\item Rotate the slit aperture to the $0.08 \unit{mm}$ slit and repeat the above procedure.  Record and label the position of as many minima as possible on a new sheet of paper.  Calculate the average value of the slit width and compared it to the new known value.
\item Based on your data and the equation describing diffraction, how should the spacing of the minima change as you increase the slit width?  Make a prediction in your notebook.  To test your prediction, rotate the slit aperture to the 'Variable Slit Width' position and slowly increase the size of the slit.  What happens to the diffraction pattern?
\item Based on the equation describing diffraction, how should the spacing of the minima change as you decrease the distance to the screen?  Make a prediction in your notebook.  To test your prediction, rotate the slit to some fixed slit width, then gradually bring the screen closer to the slit aperture.  What happens to the diffraction pattern?
\item Based on the equation describing diffraction, how would the spacing of the minima change if we were to use a smaller wavelength of light (say blue light)?  Based on your answer, describe how a diffraction grating (a transparent material with many small grooves [slits] etched into its surface) can break up white light into its component 'rainbow' of colors.
\end{question}

\subsubsection{Double Slit}

\begin{question}
\item Setup the experiment on the optics bench as initially described, but this time, use the slit-aperture wheel labeled 'Multiple Slit Set'.  Turn the wheel so that the laser is shining through the double slit pattern with slit-width $a=0.04 \unit{mm}$ and slit-separation $d=0.25 \unit{mm}$.  Observe the pattern produced on the screen.
\item[] Light is now shining through two slits, each with a width of $a=0.04 \unit{mm}$ and the two slits are separated by a distance of $d=0.25 \unit{mm}$.
\item How is the pattern produced by the double slit similar to the pattern produced by a single slit?
\item Upon closer inspection (literally, look closer), how is the pattern different from the single slit?
\item Rotate the slit aperture to the position labeled 'Comparisons', with a single slit of width $a=0.04 \unit{mm}$ directly next to a double slit with widths of $a=0.04 \unit{mm}$ and separation of $d=0.25 \unit{mm}$.  The diffraction pattern for each should be on the screen, side by side.
\item With the two diffraction patterns side by side, describe the appearance of the double slit pattern in terms of the single slit pattern.
\item Using what you've learned about the wave nature of light, describe what is occurring as light passes through a double slit.
\end{question}
\setcounter{dave}{0}

\subsection{Questions}
\begin{question}
\item When light passes through large openings, (like a doorway) a shadow is formed.  However, if you pay close attention to the shadow, you will notice that the edge of the shadow is not crisp and distinct, but instead fuzzy and blurred.  This blurring effect is also visible for the shadow of an isolated object, such as a person or a tree.  Explain this observation based on the principle of diffraction.
\item While this exercise used slits to form diffraction patterns, it is possible to form a pattern with any shape.  Based on your observation of the pattern formed by a slit, what would you guess for the appearance of a pattern formed by a square?  A circle?
\item Suppose you pass white light through a slit of width $0.04\unit{mm}$.  Based on Eq.~(\ref{eq:diff.minima2}), how far apart (in angle) would the red light ($\lambda = 650\unit{nm}$) appear compared to the blue light ($\lambda=475\unit{nm}$)?
\item Astronomers can use the concepts from this experiment to produce diffraction and interference patterns using the light from distant stars.  Using this information, they can determine the size of a single star and/or the distance between two very close stars.  These measurements correspond to which two parameters from today's experiment?
\end{question}





\onecolumn%========================================

\section{Radioactive Half-Life}
\revised{Spring, 2011}
\label{s:half-life}
\setcounter{dave}{0}


\subsection{Objectives}

\begin{lablist}
\item You will learn the proper technique for using a Geiger-Muller Counter, including finding the operating voltage and the background radiation.
\item You will use the G-M counter to determine the half-life of a radioactive isotope.
\item You will discover how the activity measured from a sample depends on the distance to a radioactive source.
\item You will discover how the activity measured from a sample depends on the amount of shielding around the source.
\end{lablist}


\subsection{Introduction}

Our understanding of the physical world was revolutionized by the discovery of nuclear radiations by the French physicist Henri Becquerel in 1896.  He found that uranium gave out a strange radiation that could penetrate paper and expose photographic plates.  Madame Marie Curie and her husband Pierre discovered two new elements polonium and radium, which were even stronger sources of these strange radiations.  The precise nature of these radiations were investigated by Ernest Rutherford, and for this he was awarded the Nobel prize in 1908.

\subsubsection{Definition of Terms}
\label{ss:def.terms}

The nucleus of every nuclide is characterized by its {\bf number of protons} ($Z$) and by its {\bf number of neutrons} ($N$).  The number of protons identifies the every natural or man-made element; helium always has one proton.  Helium always has two protons, etc.  Every atom that has more than one proton needs some neutrons to hold the nucleus together.  The number of neutrons is not always the same as the number of protons; hydrogen might have no neutrons, or it might have one neutron.  These are called {\bf isotopes} of the element.  For example, two common isotopes of carbon are C$^{12}$ and C$^{14}$.  Since carbon has $Z=6$, C$^{12}$ has $N=6$ ($12-6=6$) and C$^{14}$ has $N=8$ ($14-6=8$).  A plot of $Z$ versus $N$ is called The Chart of the Nuclides, which also displays other properties of the nuclides.  Some nuclides are stable and some are unstable.  The unstable nuclides are said to be radioactive.

{\bf Radioactivity} can be defined as the spontaneous process of unstable nuclides becoming more stable by releasing energy either as a particle with kinetic energy or as a photon of electromagnetic energy.  These released particles are called the nuclear radiations, of which there are three types: {\bf alpha} ($\alpha$) particles, {\bf beta} ($\beta$) particles, and {\bf gamma} ($\gamma$) rays.  When a nuclide emits a $\gamma$-ray, it goes from an ``excited state'' to either a ``lower excited state'' or to ``the ground state'' of the nucleus.  When a nuclide emits an $\alpha$ or a $\beta$ particle, then the nuclide changes its $N$ and $Z$ numbers; this involves the transmutation of one element into another.  An original radioactive species is called the ``{\bf parent} nuclide'' and, when it undergoes a nuclear transformation by $\alpha$- or $\beta$-emission, the resulting nuclide is called the ``{\bf daughter} nuclide.''  This process is called ``radioactive decay'' or ``radioactive disintegration.'' The daughter may also be unstable, in which case the daughter may also undergo radioactive disintegration, being the parent to a new daughter nuclide.

Consider the example of uranium.  Since uranium necessarily has $Z=92$, U$^{238}$ must have $N=238-92=146$ neutrons.  This isotope of uranium disintegrates by $\alpha$-emission.  Since an $\alpha$-particle always has $Z=2$ and $N=2$, uranium $\alpha$-decays to the isotope of the element with $Z=90$ (thorium) and $N=144$, referred to as Th$^{234}$.
\[ {\rm U}^{238}_{92} \rightarrow {\rm Th}^{234}_{90} + \alpha^4_2 \]
On the other hand, a $\beta$-decay changes one neutron in the parent into a proton in the daughter.  One example of this is the isotope of phosphorus-32, P$^{32}$ with $Z=15$ and $N=17$, which $\beta$-decays to sulfer-32, S$^{32}$ with $Z=16$ and $N=16$.
\[ {\rm P}^{32}_{15} \rightarrow {\rm S}^{32}_{16} + \beta^0_{-1} \]

\subsubsection{The Decay Process}
\label{ss:decay}

The rate of radioactive disintegration for some sample of an isotope is often used to indicate the level of radioactivity and is called the {\bf activity}.  There are two common units used when measuring the nuclear activity: A newer unit is the becquerel which is defined as a disintegration-per-second (dps).  The older, but still used, unit is the ``Curie'' defined as
\[ 1 \unit{Ci} = 3.7\ten{10}\unitfrac{disintegrates}{s} = 3.7\ten{10} \unit{Bq} \]
It is important to be aware that these units do not distinguish the {\it type} of radioactive decay that is occurring, only the number of disintegrates in a sample.

Radioactive disintegration is a random and statistical process.  This means that a very large number of atoms are needed for the disintegrates to be modeled by the laws of probability.  Furthermore, there is no physical process that can affect the disintegration process.  Given a large number of radioactive atoms, the overall number of atoms which will undergo radioactive disintegration is predictable, but the timing of the disintegration for any single atom is completely unpredictable.  The number of disintegrates is proportional to the number of atoms present.  As the radioactive atoms transmute, the number of remaining parent-atoms decreases.  This decrease in the activity is characterized mathematically with the equation
%
\begin{equation}\label{eq:half-life}
N(t) = N_0 \, e^{-\lambda t}
\end{equation}
%
where $N(t)$ is the number present at any given time, $N_0$ is the number present at the time you started your clock, and $\lambda$ is the ``{\bf decay constant}.''

The decay constant has units of inverse-seconds, so that the exponent of $e$ is unitless.  The decay constant is related to the decay rate of the parent nucleus.  The decay rate can also be expressed by the {\bf half-life} of the parent isotope, $t_{1/2}$, which is the time it takes for the number of parent nuclei present in the sample to decease to half of the original number.  (Since the activity is proportional to the number of nuclei present, the half-life also measures when the activity has dropped to half the original activity.)  The half-life and the decay constant are related to each other according to
\[ t_{1/2} = \frac{\ln 2}{\lambda} = \frac{0.693}{\lambda} \]

When counting the decay rate for a sample, one must be aware that nuclear radiations are constantly emitted from a wide variety of sources: Cosmic rays, radioactive isotopes in our own bodies (used in carbon dating), and naturally occurring radioactive isotopes in the soil, in the air, and in the building material of man-made structures.  This low-level ever-present radiation is referred to as the {\bf background radiation}.  (This is in part what allows evolution to continue.)

\subsubsection{The Penetrating Power}

The radiations emitted from a radioactive isotope can vary widely in their penetrating power.  This is due to three factors: (1) different radiation energies, (2) the different absorption properties of the radiation, and (3) the different absorption properties of the absorbing material.  The first is specific to the particular isotope that is decaying and this can be found on the Chart of Nuclides.  The second can be described in general terms for each of the three types of radiation: Alpha particles can be stopped completely by a sheet of paper.  Beta particles are about 100 times more penetrating than alpha particles.  A sheet of aluminum a few millimeters thick can stop most beta radiation. Gamma radiation can be very penetrating; it can pass right through the human body.  They can be detected after passing through a block of iron $30\unit{cm}$ thick.  This third depends on the type of shielding used.  Unfortunately, because of the first two factors, it is often difficult to create an absorption standard, especially something equivalent to human tissue other than actual human tissue.  However, one can find rough equivalents.

In terms of safety, one should be aware of the type of danger each type of radiation presents.  The danger of $\alpha$-emitters is that, although they can be stopped easily, if an $\alpha$-emitting parent isotope is inhaled or ingested, they do a lot of damage to what they are absorbed by.  Beta particles are a little more dangerous outside the body, but are also primarily an internal concern.  Gamma particles, because of their penetrating power, are a danger both inside and outside of the body.  Whenever you are in the vicinity of radioactive isotopes, you should be aware of the type of radiation and maintain suitable safety precautions.

It is possible to model the penetrating power of radiation.  Since $\alpha$ and $\beta$ particles require such thin shielding, they are more difficult to measure.  The intensity of $\gamma$ radiation after passing through a certain thickness of a given absorber is given by
\[ I(d) = I_0 e^{-\mu d} \]
where one might notice the similarity in form to Eq.~(\ref{eq:half-life}).  $I(d)$ is the intensity at any given depth, $I_0$ is the intensity before penetration, $d$ is the depth under consideration, and $\mu$ is the absorption coefficient (like $\lambda$ was the decay constant).  The absorption coefficient, $\mu$ depends on both the absorbing material and on the wavelength of the $\gamma$ radiation.As the thickness of the absorber is increased, the fraction of radiation passing through will decrease.  When exactly half of the radiation passes through the absorber, the thickness of the absorber is called the half-value-layer (HVL) or the ``half-thickness,'' $X_{1/2}$, in analogy to the half-life above.


\subsubsection{The Geiger-Muller Counter}
\label{ss:GM.counter}

Nuclear radiations cannot be detected directly by our senses, therefore a method employing the interaction of the radiation with matter is needed.  For example, in a Geiger-Muller counter (a G-M counter) the entering radiation ionizes gas molecules in the G-M tube.  When a voltage is applied to the tube, these charged ions give rise to a measurable electrical pulse.  Each pulse lasts for about a microsecond ($1\ten{-6}\unit{s} = 1 \unit{\mu s}$). During this short, but non-zero, time interval, the tube is discharging and no other electrical pulses can be counted.  The recovery time of the tube is referred to as the ``{\bf dead time}''.  Although the dead time is too fast for you to register, it does mean that the counter cannot count 100\% of the disintegrations.  Furthermore, radioactivity has variable energy values and the different types of radiation ($\alpha$, $\beta$, and $\gamma$) have different penetrating strengths.  For example, many of the $\alpha$ and $\beta$ particles can be stopped by the mica film on the face of the G-M tube. For $\beta$-particles, only about 5\% of those emitted actually are counted by the G-M tube.

In order for the G-M counter to count the ions, a voltage must be applied, pulling the charged ions to one side.  Every G-M tube has its own best operating voltage, which must be determined at the time of use.  There is a minimum voltage at which the G-M tube will be able to count -- called the starting voltage.  As one increases the voltage, the count-rate will also increase.  Each G-M counter has a small range of voltage values for which the count-rate depends weakly on the the voltage value.  The range is called the {\bf operating voltage}.  {\bf CAUTION:} Voltages above this range are in the ``discharge region'' and setting the voltage higher than the operating voltage and in the discharge region, can easily damage the G-M counter.  In general, you should keep the voltage at the low end of the operating voltage.


\subsection{Procedure:}

The preparation of the half-life sample takes about 15 minutes for each sample and must be measured immediately upon creating the sample.  So, different groups will be doing different parts of the experiment at any given time.  When the first sample is ready, GROUP 1 will measure the half-life while the other groups measure the background.  When the second sample is ready, GROUP 2 will measure the half-life while group 1 measure the background and groups 3 and 4 look at either the distance ratio or the shielding, etc.  Each part will take about 30 minutes.

When you make a measurement, you will convert it into the units counts-per-time-interval.  If you count 400 decays in 5 minutes, this is
$(400 \unit{counts}) / (5\unit{minutes}) = 80 \unitfrac{counts}{min} = 80\unit{cpm}$.
If you count 30 decays in 15 seconds, then this is
$(30\unit{counts})/(15\unit{seconds}) = 2 \unitfrac{counts}{sec} = 120 \unit{cpm}$.
This is like driving your car, you don't have to actually drive for an hour to travel $60 \unit{mph}$.

\subsubsection{Operating Voltage of a G-M tube}
\label{ss:operating.voltage}

With the G-M counter turned off, set up the tube within the tube stand and place a source disk on the second shelf.  Make sure that the voltage is set to its lowest setting.  Turn on the master power switch and allow the tube to warm up for at least one full minute.  (Once the tube has been turned on, do not turn it off until you are completely done with the experiment.)  Depress the [Reset] switch.  Start the counter and watch the rate at which it counts as you increase the voltage.  When you reach a voltage range where the count-rate does not increase while you vary the voltage, make a note of the voltage values.  Outside of that range, the count-rate will increase again -- If that happens, turn the voltage back down to the operating voltage range.
%
\begin{lablist}
\item Do not increase the voltage beyond $450\unit{volts}$.
\item Take your measurements at the low end of the operating voltage range.
\end{lablist}

\subsubsection{Background Radiation}
\label{ss:background}

Background radiation is everywhere, and because this radiation is adding to the count rate made by the sample source, it causes a systematic error in the count rate of the sample, especially when the sample activity is very low.  This systematic error can be corrected for by determining the background count rate and subtracting it from the sample activity.  As with all careful measurements, you will measure this several times and take an average.  Although the background rate is not actually constant, it varies randomly and we will assume it is constant within some standard deviation.
%
\begin{lablist}
\item Set-up the G-M counter at the proper operating voltage.
\item No source should be used in this part of the experiment, and all sources should be kept far from your detector.
\item Run the count for ten minutes and then calculate the count-per-minute.
\item Build a complete shield, using lead bricks, around the tube.  (Think about why you are doing this.)
\item Run the count for ten minutes and then calculate the count-per-minute again. (Is there a difference?)
\end{lablist}


\subsubsection{Half-Life}

Cesium-137 (Cs$^{137}$) has a fairly long half-life, long enough that you will not see a significant change in its activity during one 3-hour lab period.  However, Cs$^{137}$ $\beta$-decays to Ba$^{137}$ (barium), which has a half-life that is short enough to watch vary during one 3-hour lab.  However, we will need to separate the barium from the cesium in order to get a clean signal from the barium.  This process is an eluting system in which the short-lived daughter is washed from the long-lived parent isotope.  For this minigenerator, barium is washed from the cesium by passing a hydrochloric acid solution through the generator.

Eluting removes the barium so that you can measure its half-life, and then we must wait about 15 minutes for the cesium to regenerate a new equilibrium level of barium.  If you are not the first group measuring the half-life, then make a different measurement while you wait.
%
\begin{lablist}
\item When you are ready to make the measurement and the instructor ready to prepare your sample, be sure that you have reviewed the procedure carefully and your station is ready to make the measurement as soon as the instructor has prepared the sample.
\item The instructor will prepare the Ba$^{137}$ radioactive source and place it on shelf \#2.
\item As soon as the sample is in place, start counting.  Simultaneously start the counter and a timer.  Let the counter run in manual mode.
\item Record the number of counts and the time about every 15 seconds until the count rate is very close to the background count-rate.
\item Tabulate the counts-per-time-interval for each time, add a column that corrects the counts for the background effect.
\item Graphically display the results.  Fit this to the appropriately shaped curve.
\item Figure out how to do a linear regression in order to find the decay constant and the half-life.
\end{lablist}

\subsubsection{Activity versus Distance}

Since the G-M counter uses a tube to make its measurements, the radiation must be directed along the tube.  If the source were a point source, then the radiation would be emitted equally in all directions and only a small fraction would be counted by the tube.  This fraction-counted decreases in a very specific way as a function of the distance from the source to the window of the tube.  (HINT: This relationship is the same as the relationship used for sound or light.)  You will now determine that relationship.  \underline{First caution}: Since the source is not actually a point source, the relationship does not hold exactly; but if you do not start too close to the G-M tube, then the approximation is a little better.  \underline{Second caution}: If a beta source is used, then the results do not fit the model as well because $\beta$-particles tend to be easily absorbed by the air.
%
\begin{lablist}
\item Start with the sample on the shelf second-from-the-top.  Count for however long it takes to get at least 1000 counts.  Note the amount of time used.
\item Move the sample to the third shelf and, using this same time interval, measure the counts again.
\item Repeat this for the rest of the shelf positions.
\item Calculate a background-corrected activity for each shelf position.
\item Graph the activity versus the shelf number and see if you can figure out the appropriate fit-curve.
\end{lablist}


\subsubsection{Shielding}

You will be using the highest shelf available such that there is room for the shielding between the sample and the G-M tube.  Ideally, the shielding will be on shelf 1 and the sample will be on shelf 2; but if your thickest shield does not fit on shelf one, then all of the shielding will have to be on shelf 2 and the sample will be on shelf 3.

Notice that there are multiple types of shielding, aluminum and lead.  They have equivalent thicknesses listed on the box.
%
\begin{lablist}
\item Start with the sample on the appropriate shelf.  Count for two minutes.  Record the counts-per-minute.
\item Place the thinnest shield in place and measure the counts again.
\item Repeat this for five different shield-thicknesses.
\item Calculate the background-corrected activity for each count.
\item Graph the activity versus the shield thickness and see if you can figure out the appropriate fit-curve.
\end{lablist}


\subsection{Questions}
\begin{question}
\item If you did not subtract the background from the source activity, how big of an error would be introduced by neglecting to do this?
    \begin{enumerate}
    \item Is the error the same size for all measurements? or does it affect some data more than others?
    \item How would this affect the trendlines?  Are they shifted up? shifted over? curved more? curved less?  Why?
    \end{enumerate}
\item For each of the sources that you used, name the parent, the daughter, and the emitted radiations.  Write a nuclear equation for each process.
\item For each of the sources that you used list the number of isotopes for that element.  (These are listed on the Chart of Nuclides in the back of the room.)  How many of the isotopes are stable?
\item For each of the sources that you used, name the final stable element in its decay scheme.
\item Show that the relationship given between the half-life and the decay constant is valid.
\item What is the accuracy of your measurements?
\end{question}



\onecolumn%========================================

\section{The Michelson Interferometer}
\revised{Spring, 2011}
\label{s:interferometer}
\setcounter{dave}{0}

Sections in {\sf ``sans serif'' font} are direct quotes from the PASCO Scientific ``Interferometer'' manual, publication 012-02675B\footnote{The equipment site is \\
\centerline{http://store.pasco.com/pascostore/showdetl.cfm?\&DID=9\&Product\underline{\ }ID=1560\&groupID=638\&page=Manuals}
and the manual can be downloaded directly from \\
\centerline{ftp://ftp.pasco.com/Support/Documents/English/OS/OS-8501/012-02675b.pdf}}.

\sffamily
\subsection{Objective}

In many scientific and industrial uses of interferometers, a light source of a known wavelength
is used to measure incredibly small displacements - about $10^{-6} \unit{m}=1\unit{\mu m}$ {\rm (one micron)}. However, if
you know the distance of mirror movement, you can use the interferometer to measure the
wavelength of a light source. In this experiment you will use the interferometer to measure
the wavelength of your laser light source.

\subsection{Background}

In 1881, 78 years after Young introduced his two-slit
experiment
{\rm (See Lab~\ref{s:Diffraction}, Diffraction of Light)},
A.A.~Michelson designed and built an
interferometer using a similar principle.  Originally
Michelson designed his interferometer as a model to test
for the existence of the ether;
{\rm [his null result is in part responsible for the ether-hypothesis no longer being]}
considered a viable hypothesis.
Michelson�s interferometer has now become a widely used
instrument for measuring the wavelength of light, and for
using the wavelength of a known light source to measure
extremely small distances.

Figure~\ref{f:michelson.interferometer} shows a diagram of a Michelson interferometer.
A beam of light from the laser source strikes the
beam-splitter. The beam-splitter is designed to reflect
50\% of the incident light and transmit the other 50\%.
The incident beam therefore splits into two beams; one
beam is reflected toward mirror $M_1$, the other is
transmitted toward mirror $M_2$. $M_1$ and $M_2$ reflect the
beams back toward the beam-splitter. Half the light
from $M_1$ is transmitted through the beam-splitter to the
viewing screen and half the light from $M_1$ is reflected
by the beam-splitter to the viewing screen.
%
\begin{figure}[h]
\begin{minipage}[b]{3in}
\begin{center}
%JOE%\vspace{1in}
\includegraphics[width=3.25in]{Michelson_Interferometer.png}
\caption{The essential function of the Michelson interferometer.}
\label{f:michelson.interferometer}
\end{center}
\end{minipage}
\hfill
\begin{minipage}[b]{3in}
\begin{center}
%JOE%\vspace{1in}
\includegraphics[width=2in]{Michelson_interference_pattern.png}
\caption{An approximate view of the expected interference pattern.}
\end{center}
\fbox{\begin{minipage}{3in}
NOTE: Do not be concerned if your pattern
shows irregularities or has fewer fringes. As
long as fringes are clearly visible, measurements
will be accurate.
\end{minipage}}
\label{f:michelson.interference.pattern}
\end{minipage}
\end{figure}
%

In this way the original beam of light splits, and
portions of the resulting beams are brought back
together. The beams are from the same source and
their phases highly correlate. When a lens is placed
between the laser source and the beam-splitter, the
light ray spreads out. An interference pattern of dark
and bright rings, or fringes, is seen on the viewing
screen, as shown in Figure~\ref{f:michelson.interference.pattern}.
%
Since the two interfering beams of light were split
from the same initial beam, they were initially in
phase. Their relative phase when they meet at any
point on the viewing screen, therefore, depends on the
difference in the length of their optical paths in reaching
that point.
{\rm The bright spots and dim spots in Fig.~\ref{f:michelson.interference.pattern}
is similar to the bright and dim spots of Lab~\ref{s:Diffraction}; in particular,
Fig.~\ref{f:diff.graph} (on page~\pageref{f:diff.graph}) shows the intensity profile of bright and dim spots, and if Fig.~\ref{f:diff.graph} were rotated in a circle, it would produce Fig.~\ref{f:michelson.interference.pattern}.}


By moving mirror $M_2$, the path length of one of the
beams can be varied. Since the beam traverses the
path between $M_2$ and the beam-splitter twice, moving
$M_2$ by $\!\!\txtfrac{1}{4}$ wavelength nearer the beam-splitter will reduce
the optical path of that beam by $\!\!\txtfrac{1}{2}$ wavelength. The
interference pattern will change; the radii of the
maxima will be reduced so they now occupy the
position of the former minima. If $M_2$ is moved an
additional $\!\!\txtfrac{1}{4}$ wavelength closer to the beam-splitter,
the radii of the maxima will again be reduced so
maxima and minima trade positions. However, this
new arrangement will be indistinguishable from the
original pattern.

%{\rm As mentioned above, the Michelson interferometer can be used either to calculate the wavelength, assuming one can measure a distance, or to calculate a distance, assuming one knows the wavelength.}
By slowly moving $M_2$ a measured distance $d_m$, and
counting $m$, the number of times the fringe pattern is
restored to its original state, the wavelength of the light
($\lambda$) can be calculated as:
$\lambda = \boxfrac{2d}{m}$.
%\[ \lambda = \frac{2d}{m} \]
If the wavelength of the light is known, the same
procedure can be used to measure $d_m$.

\subsection{Operation}
\label{ss:operating.interferometer}

%\subsubsection{The Interferometer}

The Michelson Interferometer is shown in
Figure~\ref{f:interferometer.equipment}.
%The alignment of the beamsplitter
%and the movable mirror, $M_2$, is
%easily adjusted by loosening the thumbscrews
%that attach them to the interferometer.
%The fixed mirror, $M_1$, is mounted on
%an alignment bracket. The bracket has two
%alignment screws to adjust the angle of the
%mirror.
{\rm Your instructor will align the beam before lab begins. This is a sensitive adjustment; please take care to not unalign the laser and the mirrors.

You will only be moving the $M_2$ mirror.}
%
The movement of $M_2$ toward and away
from the beam-splitter is controlled and
measured using the micrometer knob.
Each division of the knob corresponds to $1\unit{\mu m} ( =1\ten{-6}\unit{m})$
of mirror movement.
%
\begin{figure}[hbtp]
\begin{minipage}[c]{3in}
\begin{center}
%JOE%\vspace{1in}
\includegraphics[width=3.25in]{Interferometer_equipment.png}
\end{center}
\vspace{-12pt}
\caption{Interferometer.  The orientation of this image is the same as Fig.~\protect{\ref{f:michelson.interferometer}}; a laser will shine from the left and the screen will be placed at the top.}
\label{f:interferometer.equipment}
\end{minipage}
\hfill
\begin{minipage}[c]{3in}
\begin{center}
%JOE%\vspace{1in}
\includegraphics[width=3in]{Mirror_movement_mechanism.png}
\end{center}
\caption{Mirror Movement Mechanism}
\label{f:mirror.movement}
\end{minipage}
\vspace{-6pt}
\end{figure}
%


\subsubsection{The Movable Mirror}

%To measure the wavelength of light, the movement of
%$M_2$ must be measurable for distances about $1 \unit{\mu m}$.
%Also,
[\ldots] As the mirror moves, its reflective
surface must remain perpendicular to the axis of the
incident light beam.
%
A taut-band carriage is used to maintain the alignment
of the reflective surface of $M_2$ as it moves. The mirror
is mounted in a cradle that is fixed to two semi-rigid
aluminum bands. With this set-up the mirror is free to
move, but its movement is constrained to a line
parallel with the beam axis.

The micrometer mechanism controls and measures the
movement of $M_2$. The cradle of $M_2$ is attached to a
mylar strip that is attached to a lever arm. The displacement
of the lever is controlled with the micrometer
knob.
%
Suppose the micrometer knob is turned so it pushes the
lever in by a distance $d$ (see Figure~\ref{f:mirror.movement}).
The angle of
the lever arm changes by an amount $\theta$ such that $d = R \tan \theta$, as shown. Since the angle change is always
small,
$R \tan \theta = R \theta$, to a close approximation. This change
in the lever arm angle causes the mylar strip to be
pulled further around the lever post by an amount $r \theta$,
where $r$ is the radius of the lever post. The mirror is
therefore pulled away from the beam-splitter by the
amount, $r \theta$.
%
In this way, a relatively large displacement of the lever
($d = R \theta$) results in a much smaller displacement of the
mirror ($d_m = r \theta$). By selecting appropriate values for
$r$ and $R$, the motion of $M_2$
is controlled so that each
division on the micrometer dial corresponds to $1 \unit{\mu m}$ of mirror movement.
[\ldots]

\note{\subsubsection{Aligning the Interferometer}

\fbox{\begin{minipage}{6.5in}
NOTE: This alignment procedure is for
those using a PASCO scientific Optics Bench. If
you are not using an Optics Bench, tape a
straightedge to a flat level surface. The straightedge
will provide a substitute for the alignment
rail of the optics bench.
\end{minipage}}
\begin{enumerate}
\item Place the laser and the interferometer on the Optics
Bench, approximately $10$-$20 \unit{cm}$ apart (Figure~\ref{f:Adjusting.m1}).
Be sure that the edges of both units are flush
against the alignment rail of the bench. Place a
viewing screen as shown. (A blank sheet of white
paper taped to the cover of a book provides a convenient
screen.) Turn on the laser.
\item Loosen the thumbscrew that holds the beam-splitter
and rotate the beam-splitter so it is out of the beam
path of the laser as shown in Figure~\ref{f:Adjusting.m1}. Then loosen
the thumbscrew that holds $M_2$, the movable mirror.
Adjust the rotation of $M_2$ so the laser beam is reflected
directly back toward the aperture of the laser.
(The reflected beam need not be at the same
height as the incident beam, but it should strike the
front panel of the laser along a vertical line through
the aperture.) Hold $M_2$ in position and tighten the
thumbscrew.
%
\begin{figure}[hbt]
\begin{center}
%JOE%\vspace{1in}
\includegraphics[width=6in]{Adjusting_m1.png}
\end{center}
\caption{Adjusting $M_1$}
\label{f:Adjusting.m1}
\end{figure}
%
\item Rotate the beam-splitter so its surface is at an angle
approximately $45^\circ$ with the incident beam from the
laser (see Figure~\ref{f:Aligning.laser.spots}). You will see two sets of laser
spots on the viewing screen, corresponding to the
two paths that the beam takes in reaching the
screen. (Each path results in more than one laser
spot because of multiple reflections within the
beam-splitter.) Adjust the beam-splitter so the two
sets of laser spots are as close as possible, then
tighten the thumbscrew to secure the beam-splitter.
Using the alignment screws, adjust the angle of $M_1$
until the two sets of laser spots are superimposed
on the viewing screen (the two brightest spots must
be superimposed).
%
\begin{figure}[h]
\begin{center}
%JOE%\vspace{1in}
\includegraphics[width=6in]{Aligning_laser_spots.png}
\end{center}
\caption{Aligning the laser spots}
\label{f:Aligning.laser.spots}
\end{figure}
%
\item Place the lens holder on the optical bench as shown
in Figure~\ref{f:Positioning.lens}. Be sure its edge is flush against the
alignment rail. Then place the $18 \unit{mm}$ focal length
lens on the lens holder (it attaches magnetically).
Adjust the position of the lens on the holder so the
light from the laser, now spread out by the lens,
strikes the center of the beam-splitter. If you have
performed the alignment correctly, you will see an
interference pattern of concentric rings on the viewing
screen. If the alignment is not just right, the
center of the fringe pattern may not be visible on
the screen. Adjust the alignment screws on $M_1$
very slowly as needed to center the pattern.
\end{enumerate}
%
\begin{figure}[h]
\begin{center}
%JOE%\vspace{1in}
\includegraphics[width=6in]{Positioning_lens.png}
\end{center}
\caption{Positioning the lens}
\label{f:Positioning.lens}
%\end{minipage}
\end{figure}
%

}


\subsection{Procedure}
\begin{enumerate}
\item {\rm Verify that your instructor has aligned} the laser and interferometer \note{as described in Sec.~\ref{ss:operating.interferometer}, }so an interference
    pattern of circular fringes is clearly visible on your viewing screen.
\item Adjust the micrometer knob so the lever arm is approximately parallel with the edge of the
    interferometer base. In this position the relationship between knob rotation and mirror
    movement is most nearly linear.
\item Turn the micrometer knob one full turn counterclockwise. Continue turning counterclockwise
    until the zero on the knob is aligned with the index mark.
\item[] \hspace{-.45in} \fbox{\begin{minipage}{6.55in}
    NOTE: Whenever you reverse the direction in which you turn the micrometer knob,
    there is a small amount of give before the mirror begins to move. This is called
    mechanical backlash, and is present in any mechanical system involving reversals in
    direction of movement. By beginning with a full counterclockwise turn, and then turning
    only counterclockwise when counting fringes, you can eliminate backlash in your
    measurement.
    \end{minipage}}
\item If you are using a blank piece of paper as your viewing screen, make a reference mark on
    the paper between two of the fringes. You will find it easier to count the fringes if the
    reference mark is one or two fringes out from the center of the pattern.
\item Rotate the micrometer knob slowly counterclockwise. Count the fringes as they pass your
    reference mark. Continue until a predetermined number of fringes has passed your reference
    mark (count at least 20 fringes). As you finish your count, the fringes should be in the
    same position with respect to your reference mark as they were when you started to count.
\item Record $d_m$, the distance that the movable mirror moved toward the beam-splitter as you
    turned the micrometer knob. Remember, each division on the micrometer knob corresponds
    to one micron ($1\unit{\mu m}$) of mirror movement.
    Record $m$, the number of fringes that crossed your reference mark.
%    \[ d_m = \blank \hspace{1cm} m = \blank \]
%    \hfill \mbox{$d_m$ = \blank} \\
%    \hfill \mbox{$m$ = \blank} %\\[0cm]
\item Calculate the wavelength of the laser light $\left(\lambda = \boxfrac{2d_m}{m}\right)$.
\item Calculate the percentage difference between your measured value for the wavelength of the
    laser light and the value recorded in the laser specifications.
    %(Check with your teacher for the laser specifications.)
\end{enumerate}

\rmfamily

\subsection{Questions}
\begin{question}
\item Why does the screen display bright spots and dim spots?
\item When you move the $M_2$ mirror, why does the pattern change?
\item Draw the path that light follows to get from the laser, off $M_1$ to the screen.  Separately, draw the path that light follows to get from the laser, off $M_2$ to the screen.  Describe where you would find the difference in the lengths of the two paths.
\item Does the pattern ever give a dark spot at the very center?  Comment on the difference in path-length that would produce this pattern.
\item If the laser had a different color (different wavelength), then how would that affect the pattern on the screen, if at all?
\end{question}






%\end{document}









\section[Human Forearm]{Conditions of Equilibrium -- Model of a Human Forearm}
\revised{(Nov 22, 2011)}
\revised{(Aug 18, 2011)}

\subsection{Introduction}

Objects that are not accelerating are said to be in a state of equilibrium.  If the object is moving at a constant velocity, then it is in equilibrium.  If the object is at rest, then it is in ``static equilibrium.''  These principles apply to many physical examples in engineering, architecture, and biophysics.   In particular, these principles allow one to be able to analyze and calculate the forces on the beams or the cables in a bridge or the forces at work in the muscles and bones in the human body.

The two conditions for equilibrium can be stated in equation form: First, if the body's center of mass is in translational equilibrium then it will not accelerate in any direction.
\[ \sum \vec F = 0 \]
Secondly, if the body is in rotational equilibrium then it will not rotate about any point or axis of rotation.
\[ \sum \tau = 0\]

For all systems such as these there is a special point called the center of mass or center of gravity of the system.  The center of mass calculation is a weighted average of the individual masses (giving more emphasis to those positions where there is more mass).  The location of this special point can be useful in determining whether the system will be in equilibrium.  For an object with uniform density, such as a half-meter stick, the center of mass is at the center of the stick.

\subsection{Experimental Objectives}

You will verify the equations for translational and rotational equilibrium by experimentally balancing the system and comparing the measurements to the results from these equations.

\vfill

\subsection{Prelab Work}

\begin{lablist}
\item Define the following terms:  torque, lever arm, and center of mass.
\item State in a sentence, the first and second conditions of equilibrium.
%\item For a 2-dimensional system, these two conditions of equilibrium can be written in three equations,  write these 3 equations.
\item For a linear function, the slope can be determined from knowing the values of any two points, $(x_1,y_1)$ and $(x_2,y_2)$.   Using these two points write out a formula for the slope of the line.
\item Draw a force diagram for a horizontal forearm as outlined in Section~\ref{ss:setup}.  Label all of the appropriate forces.
\end{lablist}

\newpage


\subsection{Procedure}

\subsubsection{Determine the Spring Constant of the Bicep Muscle}

\noindent
Before creating the forearm model, we need an equation for the force that the bicep exerts.  Since the spring models the bicep, we can use Hooke's law: $F = k(x-x_0)$, where $x_0$ is the equilibrium position.  To find the force as a function of position, $F(x) = k x + b$, you will make two measurements of position and force and then determine the spring constant $k$ and the intercept $b$.
\begin{lablist}
\item Hang the spring from the vertical support so that it is easy to measure the position of the bottom of the spring.
\item Measure the position of the bottom hook of the spring, $x_1$, with about $m_1=250\unit{g}$ placed on the spring.
    Measure the position of the bottom hook of the spring, $x_2$, with about $m_2=850\unit{g}$ placed on the spring.
    (If you measure from the floor, which is more convenient, then $k$ will be negative.)
\begin{lablist}
\item Parallax errors can be very significant with this measurement.  With the use of a mirror, this error can be reduced.  Hold the mirror vertically along the track and align your line of sight until the bottom hook of the spring and its mirror image are at the same level.  Then take the position readings.
\item Each person in the group should carry out these measurements without looking at the other group members' results.  The average of these locations should be used.
\end{lablist}
\item With these two measurements, calculate the spring constant $k$ as the slope of the graph of $F$ versus $x$.
    You do not have to -- and should not -- actually create a graph of $F$ versus $x$ to do this, {\it if} you remember your prelab work for slope.
\begin{lablist}
%\item $A$ is the slope and can be found from $\displaystyle \frac{m_2-m_1}{x_2-x_1}$.
\item $b$ is the intercept.  Using one pair of data and your knowledge of $k$, find $b$.
\item Determine the uncertainty in $k$ and $b$.
\item Now, with values and uncertainties for $k$ and $b$, you can use $F=kx+b$ to find the force that the spring thinks it is supporting.
\end{lablist}
\end{lablist}


\skipthis{
We can use a new version of Hooke's law to calculate the amount of force that the spring exerts by measuring the stretch of the spring (rather than by measuring the actual force).
\begin{lablist}
\item Measure the position of the bottom hook of the spring, $x_1$, with about $m_1=250\unit{g}$ placed on the spring.  Measure the position of the bottom hook of the spring, $x_2$, with $m_2=850\unit{g}$ placed on the spring.
\begin{lablist}
\item Parallax errors can be very significant with this measurement.  With the use of a mirror, this error can be reduced.  Hold the mirror vertically along the track and align your line of sight until the bottom hook of the spring and its mirror image are at the same level.  Then take the position readings.
\item Each person in the group should carry out these measurements without looking at the other group members' results.  The average of these locations should be used.
\end{lablist}
\item With these two measurements, we can create an equation\footnote{The equation is easier to calculate when you compare mass to position; but, it is easier to understand if you compare the weight (force on the spring) versus the position.  The equation above relates the position to the {\it mass}.} that calculates the mass on the spring from the stretch of the spring:  $m(x) = A x + B$. Remember your prelab work for slope.
\begin{lablist}
\item $A$ is the slope and can be found from $\displaystyle \frac{m_2-m_1}{x_2-x_1}$.
\item $B$ is the intercept.  Using one pair of data, $m_1 = A x_1 +B$, solve this for $B$.
\item Now, with values of $A$ and $B$, you can use $m=Ax+B$ to find the mass that the spring thinks it is supporting, $m$, from the location, $x$, of the spring when stretched.
\end{lablist}
\end{lablist}
}


\subsubsection{Experimental Setup for the Equilibrium Experiment}\label{ss:setup}

For our purposes, the forearm can be considered to jut out forwards from a vertical upper arm with the hand and the weight of the forearm itself pulling down while the bicep holds the forearm up.
The elbow joint is assumed to be a nearly frictionless pivot point for the forearm, allowing the forearm to rotate about the elbow.  The upper arm (the humerus) is connected to the forearm (the ulna) at the elbow, exerting a downward force on the forearm at the joint.

\begin{center}
\begin{tabular}{lll}
Human Arm & Model & Force Location \\ \hline
Forearm & half-meter stick & center of mass \\
\hfill (ulna and radius) && \\
Hand & clamp and hanger  & about $1\unit{cm}$ from the far end \\
& \hfill or plastic cup & \\
Bicep Muscle & clamp and spring & $8.6\%$ of the length of the forearm from the elbow \\
& \hfill (pulling up) & \hfill (about $4\unit{cm}$ from the elbow) \\
Elbow (humerus) & clamp, hanger, weights & as near as possible to the elbow end \\
&& \hfill (about $1\unit{cm}$ from the elbow-end)
\end{tabular}
\end{center}

\noindent
Be sure to include the weight of the clamps at each location where one is used.
%
\begin{lablist}
\item Set-up a force diagram for this model of a horizontal forearm.
\item Set-up the equilibrium equations for this system.
\end{lablist}


\subsubsection{Equilibrium Experiment}
\label{ss:equilib}

You are going to do the following procedure for two different locations of the bicep muscle:  First set the bicep, as mentioned above, in the human location of about $4\unit{cm}$ from the elbow.  Second, set the bicep at about $10\unit{cm}$ from the elbow.  You should consider which location gives more leverage to lift the hand.

\begin{lablist}
\item With your model completely set up, place a $50 \unit{g}$ mass in the ``hand.'' {\it Experimentally} determine the force at the elbow necessary to balance the system and make it horizontal.  You may also change the position of the hand by small amounts if you find it easier to balance the system.  However, be sure to record the correct distances.
\item To check that the system is level, use another meter stick to measure the height from the floor {\it at each end} of the forearm half-meter stick.
\item Determine the force in the bicep muscle using the formula derived in the previous section as appropriate for the new position of the bottom of the hook of the spring (again using the mirror for accuracy).  Be sure to include the uncertainties!
\item Record all force locations as read directly from the ``forearm'' meter stick. {\bf Do not subtract in your head} so that we can reproduce the locations later, if need be.
\item {\bf Experimentally} determine the sensitivity of your values for force at the elbow by checking how many grams can be added or removed at the elbow while maintaining the horizontal equilibrium.
\item Since the hand is so sensitive, you should estimate roughly the sensitivity of the value of force at the hand.
\end{lablist}

With this data, you will verify the equations of equilibrium.

\subsubsection{Comparative Anatomy}
\label{ss:elephant}

Create a table that allows you to compare the values of force for the elbow, bicep, and hand for three different attachment-locations.
\begin{lablist}
\item Enter the data from Sec.~\ref{ss:equilib}.  Record each position and each force.
\item Predict how the forces will change if the hand is moved a few millimeters closer to the bicep.  (Hint: Imagine carrying grocery bags at different locations on your lower arm while holding your arm in an L-shape.)
\item Move the hand in and measure the forces that make it balance.  Compare to your predictions.
\item Predict how the forces will change if the bicep is moved a few millimeters further from the elbow.  (Hint: Imagine sitting at different locations on a teeter-totter.)
\item Move the bicep away from the elbow and measure the forces that make it balance.  Compare to your predictions.
\end{lablist}

\subsection{Analysis}

\begin{lablist}
\item For the forces measured or calculated in Sec.~\ref{ss:equilib}, test each of the conditions of equilibrium:
\begin{lablist}
\item Translational Equilibrium: Show that the up forces equal the down forces to within your uncertainty.
\item Rotational Equilibrium: Show that the clockwise torques equal the counterclockwise torques to within your uncertainty.
\item To show that it does not matter which point you choose to calculate the summation of the torques about, choose the far end of the meter stick (near the hand) as the zero or the rotation point.
\item If your conditions of equilibrium are not consistent, then calculate the necessary mass that should be in the hand and test this value experimentally. Discuss how well this value works.  (If you cannot get this to balance, you might check to be sure you included the weight of the forearm itself\ldots)
\end{lablist}
\item Consider the forces in Sec.~\ref{ss:elephant}.
\begin{lablist}
\item If you have three shopping bags of food at the grocery store that you want to hang from your arm as you walk to your car, should you hang the heaviest bag closest to your elbow or closest to your hand?  Explain.
\item The location of the bicep relative to the elbow determines the leverage in pulling your hand to your chest (or for a quadruped move their feet forward for the next step).  Do you expect fast animals to have their leg muscles attached close to the joint or far from the joint?  Explain.
\end{lablist}
\end{lablist}


\subsection{Questions}
\begin{enumerate} \itemsep 0cm
\item Is it necessary to have the meter stick horizontal for the system to be in equilibrium?   Why did we want to keep the arm horizontal?
\item How was the tension in the muscle affected as the position of attached muscle was moved further from the joint?   Keep the mass in the hand constant.
\skipthis{\item Consider an elephant and a gazelle.  Is either animal likely have their leg muscles connected at 8.6\% of their foreleg distance from the joint, like in humans?  Do you expect their bicep to be connected more closely to or further from their elbow?

    Note that large animals like an elephant are built for strength and not for speed.   Animals like a gazelle, are built for speed.  These properties are related to the animal's mass and the size of their bones.  But these properties are also related to where the muscles are attached to the bones.  Use your data and results to compare an elephant's leg to a gazelle's leg to explain the ``strength'' versus ``speed'' for these two animals.  For comparison, this human forearm system ``strength'' can be interpreted to mean that there is a greater ability to lift a larger mass (in the hand) with a smaller force applied by the muscle.  ``Speed'' can be interpreted to mean that a large change in the muscle tension force needed to move the limb (which is a small mass) in a short time period.
    }
\end{enumerate}


%-------------------------------------------------------------------------------------

\onecolumn
\newcommand{\iftwoweek}[2]{#2}


\end{document} 